\documentclass[10.5bp, jafontscale=1.00]{ltjsarticle}
\usepackage{stix,amsmath,cite,citesort,eclbkbox,listings,jlisting}
\usepackage{graphicx}
\usepackage{titlesec}
\usepackage{booktabs}
\usepackage{caption}
\captionsetup{justification=centering, skip=2pt}
\usepackage{tikz}
\usetikzlibrary{shapes,arrows,positioning}
\usepackage[no-math,expert,haranoaji]{luatexja-preset}
\setmainfont{Times New Roman}[BoldFont=TimesNewRomanPS-BoldMT]
\setsansfont{Times New Roman}[BoldFont=TimesNewRomanPS-BoldMT]
\setlength{\oddsidemargin}{-0.in}
\setlength{\evensidemargin}{-0.in}
\setlength{\topmargin}{-0.in}
\setlength{\headheight}{0cm}
\setlength{\headsep}{0mm}
\setlength{\textheight}{43.5\baselineskip}
\setlength{\columnsep}{9mm}
\setlength{\topsep}{0pt}
\setlength{\partopsep}{0pt}
\setlength{\parsep}{0pt}
\titleformat*{\section}{\normalsize\bfseries}
\titleformat*{\subsection}{\normalsize\bfseries}
\titleformat*{\subsubsection}{\normalsize\bfseries}
\makeatletter
\renewcommand\section{%
  \@startsection{section}%
    {1}%
    {0pt}%
    {-0\baselineskip plus -1ex minus -.2ex}%
    {0\baselineskip plus.2ex}%
    {\normalfont\normalsize\bfseries}%
  }
\renewcommand{\subsection}{%
  \@startsection{subsection}%
    {2}%
    {0pt}%
    {-0\baselineskip plus -1ex minus -0.2ex}%
	{0\baselineskip plus 0.2ex}%
    {\normalsize\bfseries}%
}
\renewcommand{\subsubsection}{%
  \@startsection{subsubsection}%
    {3}%
    {0pt}%
    {-0.\baselineskip plus -1ex minus -0.2ex}%
	{0.\baselineskip plus 0.2ex}%
    {\normalsize\bfseries}%
}
\def\linesperpage#1{%
\baselineskip=\textheight
\divide\baselineskip by #1}
\def\fitwidth#1#2{\leavevmode{%
  \setbox0=\hbox{#2}%
  \ifdim\wd0>#1 \resizebox{#1}{\height}{#2}%
  \else
    \def\@tempa{#2}%
    \ifx\@tempa\empty\hbox to#1{\hss}%
    \else\relax\hbox to#1\bgroup\hfil\@fitwidth#2{}\end@fitwidth\fi
  \fi}}
\def\@fitwidth#1#2\end@fitwidth{\def\@tempa{#2}#1%
  \ifx\@tempa\empty\let\next\end@fitwidth
  \else\hfill\def\next{\@fitwidth#2\end@fitwidth}\fi
  \next}
\def\end@fitwidth{\hfil\egroup}
\makeatother
\begin{document}
\linesperpage{48}
\pagestyle{empty}
\twocolumn[
\fontsize{14bp}{\baselineskip}\selectfont
\begin{flushleft}
{\bf Eー3}
\end{flushleft}
\vspace*{-\baselineskip}
\begin{center}
\textbf{授業評価の数値に表れない学生の本音}
\end{center}
\vspace*{0.5\baselineskip}
\fontsize{10.5bp}{\baselineskip}\selectfont
\begin{center}
22M11178 藺牟田 晃弘
\end{center}
\begin{flushright}
指導教員 佐藤 大輔
\end{flushright}
\hspace*{1cm}
\vspace*{-0.\baselineskip}
]
\fontsize{10.5bp}{\baselineskip}\selectfont
\section{はじめに}
現在,大学では教育の質向上のために,学生による授業評価は,教育改善のための重要な指標とされている.多くの大学では,学期末に授業評価アンケートを実施し,その結果を教員にフィードバックしている.一般的にアンケートは,多段階のスコアと自由記述で構成されている.現状の課題として,一個人がどのような項目に重きを置いて授業評価スコアを判断しているかが不明である.また,評価スコアだけでは捉えきれない学生の本音が存在していると考えられる.

そこで,学生の本音を見ることができる自由記述に着目する.自由記述には,学生個人の授業や教員に対する感情を反映していると考えられる.本研究では,授業ごとの自由記述から読み取れる感情スコアと,授業評価スコアの関係性を分析し,授業評価の要因を明らかにする.
\section{手法}
\subsection{用いるデータセット}
本研究では,福岡工業大学の授業評価システムの期間6年分を対象としたデータを使用した.対象期間は2018年度から2023年度までであり,授業数は3,268件である.
授業アンケートの構成として,授業評価スコアと自由記述がある.
授業評価スコアは,択一式質問の点数化による単純平均である.自由記述は,「先生に向けてこの授業の感想や学んだこと,意見や要望を記述してください」と「次期履修者に向けて,この授業についてのアドバイスを記述してください」の2つの質問から構成されている.
モデル構築のために,ランダム抽出した1,000件を手動でラベリングし,教師あり学習のデータとして使用した.自由記述を手動で分類した結果,ネガティブ191件(感情スコア-1),ポジティブ180件(感情スコア+1),ニュートラル628件(感情スコア0)となった.
表\ref{tab:stats}に基本統計量を示す.感情スコアの平均は0.001でほぼニュートラルであり,標準偏差は0.260で適度なばらつきがある.授業評価スコアの平均は3.459点(4点満点)であり,全体的に比較的高い評価が多い傾向にある.
\begin{table}[b]
\centering
\caption{データの基本統計量}
\label{tab:stats}
\small
\begin{tabular}{lrr}
\toprule
\multicolumn{1}{c}{統計量} & \multicolumn{1}{c}{感情スコア} & \multicolumn{1}{c}{授業評価スコア} \\
\midrule\midrule
平均 & 0.001 & 3.459 \\
標準偏差 & 0.260 & 0.216 \\
最小値 & -1.000 & 2.000 \\
中央値 & 0.000 & 3.480 \\
最大値 & 1.000 & 4.000 \\
\bottomrule
\end{tabular}
\end{table}
\subsection{モデル}
本研究では,日本語の事前学習済みBERTモデルをベースモデルとして使用した.BERTは,Googleが2018年に開発した双方向Transformerベースの言語モデルである.本研究では,このモデルを作成した教師データ(1,000件)で微調整(fine-tuning)した.

本研究では,感情スコア予測と授業評価スコア予測を同時に学習するマルチタスクモデルを構築した.マルチタスク学習は,複数のタスクを同時に学習することで,タスク間の共通した特徴を効率的に学習できる手法である.これにより,感情スコアと授業評価スコアの両方に影響する要因と,それぞれに特有の要因を分離する\cite{mtl_survey}.

さらに,授業評価スコアを順序変数として扱う順序回帰モデルを構築した\cite{coral}.授業評価スコアは1点から4点までの順序尺度であり,単純な回帰や分類では順序関係を適切に扱えないため,順序回帰モデルを用いた.順序回帰モデルでは,各評価スコア(1点,2点,3点,4点)の確率(P1,P2,P3,P4)を同時に予測する.これにより,中低評価確率(P2:2点を減らす要因)と高評価確率(P4:4点を増やす要因)を同時に分析することが可能となる.本研究では,SHAP分析によりP2とP4の要因を解明することで,授業評価スコアで2点と4点に強く影響する要因を特定する.

図\ref{fig:architecture}にモデルアーキテクチャを示す.マルチタスクモデルでは,BERTエンコーダの共有層から感情分析ヘッドと評価スコアヘッドが分岐し,2つのタスクを同時に学習する.

SHAP\cite{shap}は,機械学習モデルの予測結果に対する各特徴量の寄与度を定量的に算出する説明可能AI手法である.SHAP値が正であれば予測を高める方向に,負であれば低める方向に寄与する.本研究では,単一タスクモデル,マルチタスクモデル,および順序回帰モデルの各モデルに対してSHAP分析を実施した.

\begin{figure}[t]
\centering
\includegraphics[width=\columnwidth]{test2.png}
\caption{モデル構造(マルチタスク+順序回帰)}
\label{fig:architecture}
\end{figure}
\section{結果}
\subsection{単一タスクモデルのSHAP分析}
表\ref{tab:single_shap}に,ポジティブ判定に寄与する重要語TOP10を示す.分析には,層化サンプリングにより抽出した5,000件を使用した.「やす」(SHAP値0.27),「良かっ」(0.25),「おもしろ」(0.24)などの単語が上位にランクインした.特に「やす」は「わかりやす」「取りやす」などの語幹であり,授業の理解しやすさを表している.この結果から,理解しやすさと面白さが学生の満足度に大きく寄与することが明らかになった.
\begin{table}[t]
\centering
\caption{ポジティブ判定に寄与する重要語TOP10}
\label{tab:single_shap}
\small
\begin{tabular}{crrr}
\toprule
 & \multicolumn{1}{c}{単語} & \multicolumn{1}{c}{SHAP値} & \multicolumn{1}{c}{出現回数} \\
\midrule\midrule
1 & やす & 0.2660 & 337 \\
2 & 良かっ & 0.2466 & 207 \\
3 & おもしろ & 0.2438 & 10 \\
4 & よかっ & 0.2251 & 195 \\
5 & 面白 & 0.2178 & 100 \\
6 & 楽しい & 0.1959 & 67 \\
7 & 楽しめる & 0.1876 & 6 \\
8 & ありが & 0.1760 & 19 \\
9 & 楽し & 0.1642 & 192 \\
10 & 面白い & 0.1518 & 37 \\
\bottomrule
\end{tabular}
\end{table}
\subsection{マルチタスク学習のSHAP分析}
マルチタスク学習のSHAP分析により,4つの要因タイプを特定した(表\ref{tab:multitask_factors}).共通要因(満足度要因)は577語彙(18.0\%),感情特化要因は1,200語彙(37.5\%),評価特化要因は532語彙(16.6\%)が特定された.\begin{table}[t]
\centering
\caption{SHAP分析による要因(マルチタスク)}
\label{tab:multitask_factors}
\small
\begin{tabular}{lrrl}
\toprule
\multicolumn{1}{c}{要因タイプ} & \multicolumn{1}{c}{語彙数} & \multicolumn{1}{c}{割合} & \multicolumn{1}{c}{影響} \\
\midrule\midrule
共通要因(満足度) & 577 & 18.0\% & 両方 \\
感情特化要因 & 1,200 & 37.5\% & 感情のみ \\
評価特化要因 & 532 & 16.6\% & 評価のみ \\
低重要度要因 & 889 & 27.8\% & 影響が小さい \\
\bottomrule
\end{tabular}
\end{table}
表\ref{tab:common_factors}に,共通要因(満足度要因)の代表例TOP5を示す.「学ぶ」「理解」「総括」「推奨」などが含まれている.これらの要因は,感情スコアと評価スコアの両方に正の影響を与えるため,1つの施策で学生満足度と教育効果の両方を向上させることができる.そのため,教育改善における投資効果が高い.
\begin{table}[t]
\centering
\caption{共通要因(満足度要因)TOP5}
\label{tab:common_factors}
\small
\begin{tabular}{crrr}
\toprule
\multicolumn{1}{c}{順位} & \multicolumn{1}{c}{単語} & \multicolumn{1}{c}{感情重要度} & \multicolumn{1}{c}{評価重要度} \\
\midrule\midrule
1 & 学ぶ & 0.001278 & 0.001386 \\
2 & 理解 & 0.001073 & 0.000833 \\
3 & 総括 & 0.000974 & 0.000952 \\
4 & 推奨 & 0.001132 & 0.000755 \\
5 & 人数 & 0.001195 & 0.000704 \\
\bottomrule
\end{tabular}
\end{table}
\section{おわりに}
本研究では,自由記述の感情分析により,授業評価の要因を明らかにした.単一タスクモデルのSHAP分析では,理解しやすさと面白さが満足度に大きく寄与することが示された.さらに,マルチタスク学習により,共通要因(満足度要因)への投資が最も効率的であることが明らかになった.これらは, 限られた教育リソースを効果的に配分するための指針となる.今後は,順序回帰モデルのSHAP分析により,授業評価スコアで2点と4点に強く影響する要因を解明していく.
\begin{thebibliography}{9}
\bibitem{mtl_survey} Zhang, Y., and Yang, Q., ``A Survey on Multi-Task Learning,'' \textit{IEEE Transactions on Knowledge and Data Engineering}, Vol. 34, No. 12, pp.~5586--5609, 2022.
\bibitem{coral} Cao, W., Mirjalili, V., and Raschka, S., ``Rank consistent ordinal regression for neural networks with application to age estimation,'' \textit{Pattern Recognition Letters}, Vol. 140, pp.~325--331, 2020.
\bibitem{shap} Lundberg, S. M., and Lee, S. I., ``A Unified Approach to Interpreting Model Predictions,'' \textit{Advances in Neural Information Processing Systems}, pp.~4765--4774, 2017.
\end{thebibliography}
\end{document}
