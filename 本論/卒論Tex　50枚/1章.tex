\chapter{はじめに}
\setcounter{page}{1}
\pagenumbering{arabic}

%%%%%%%%%%%%%%%%%%%%%%%%%%%%%%%%%%%%%%%%%%%%%%%%%%%%%%%%%%%%%%%%%%%%%%%%%%%%%%%
\section{研究背景}
%%%%%%%%%%%%%%%%%%%%%%%%%%%%%%%%%%%%%%%%%%%%%%%%%%%%%%%%%%%%%%%%%%%%%%%%%%%%%%%

\subsection{高等教育における質保証}
高等教育機関では,教育の質保証が重要な課題となっている.大学設置基準の改正や認証評価制度の導入により,各大学は教育活動の点検・評価を行い,その結果を教育改善に活かすことが求められている.この文脈において,学生による授業評価は,教育の質を測定する主要な手段として広く実施されている\cite{marsh2007,spooren2013}.

授業評価は,学生の視点から授業の質を把握し,教員へのフィードバックを通じて教育改善を促進する役割を担っている.多くの大学では,学期末にアンケート形式で授業評価を実施し,その結果を教員に還元するとともに,全学的な教育改善の資料として活用している.

\subsection{授業評価アンケートの構成}
授業評価アンケートは,一般に多段階の評価スコア(例:4段階や5段階の択一式回答)と自由記述から構成される.評価スコアは定量的な比較に適しており,授業間や教員間の比較,経年変化の把握などに活用される.一方,自由記述は学生の具体的な意見や感想を収集でき,評価スコアの背景にある理由や具体的な改善提案を把握できる利点がある.

しかしながら,評価スコアと自由記述はそれぞれ異なる特性を持つ.評価スコアは数値として集計・比較が容易であるが,なぜその評価に至ったかという理由は直接観測できない.自由記述には学生の本音や具体的な要望が含まれることが多いが,非構造データであるため大規模な分析には困難が伴う.

\subsection{自由記述分析の課題}
自由記述の分析は,従来,教員や職員による人的な読解に依存してきた.しかし,大規模な大学では学期ごとに数万件の自由記述が収集されることがあり,すべてを人力で読解することは現実的ではない.また,読解者の主観により解釈が異なる可能性があり,客観的な分析が困難である.

このような背景から,自然言語処理技術を用いた自由記述の自動分析への期待が高まっている\cite{gottipati2018,hujala2020}.特に,近年の深層学習技術の発展により,大規模なテキストデータから意味のある情報を抽出することが可能になってきた\cite{bert}.

\subsection{感情分析と教育分野への応用}
感情分析(Sentiment Analysis)は,テキストに含まれる肯定的・否定的・中立的な感情を自動的に推定する技術である\cite{liu2012}.ソーシャルメディアの分析や製品レビューの分析など,様々な分野で活用されている.教育分野の自由記述分析でも,自動分析の活用が進んでおり\cite{gottipati2018,hujala2020},感情情報の活用可能性が示唆されている.

自由記述を感情スコアとして数値化することで,評価スコアとの関係を統計的に検討できる可能性がある.感情スコアと評価スコアの相関を分析することで,学生の感情が授業評価にどのように関係しているかを明らかにできると考えられる.

\subsection{本研究の対象データ}
本研究では,福岡工業大学における2018年度から2023年度までの6年間の授業評価データを分析対象とする.対象データの規模を表\ref{tab:data_intro}に示す.

\begin{table}[t]
    \centering
    \caption{本研究の対象データ}
    \label{tab:data_intro}
    \resizebox{0.75\textwidth}{!}{
    \begin{tabular}{l r}
        \toprule
        項目 & 値 \\
        \midrule
        対象期間 & 2018年度〜2023年度(6年間) \\
        対象学科数 & 9学科 \\
        授業数 & 3,268件 \\
        自由記述総件数 & 83,851件 \\
        平均自由記述数/授業 & 25.2件 \\
        \bottomrule
    \end{tabular}
    }
\end{table}

このデータ規模は,機械学習モデルの構築および統計的分析を行う上で十分な量であり,得られた知見の信頼性を担保できると考えられる.

%%%%%%%%%%%%%%%%%%%%%%%%%%%%%%%%%%%%%%%%%%%%%%%%%%%%%%%%%%%%%%%%%%%%%%%%%%%%%%%
\section{課題の整理}
%%%%%%%%%%%%%%%%%%%%%%%%%%%%%%%%%%%%%%%%%%%%%%%%%%%%%%%%%%%%%%%%%%%%%%%%%%%%%%%

授業評価の活用には,以下の課題が存在する.

\subsection{評価要因の不可視性}
授業評価スコアは,授業内容,教員の説明力,教材の質,試験の難易度など,複数の要因に対する総合判断として付与される.しかし,どの要因がどの程度評価に影響したかは直接観測できない.例えば,同じ3点の評価であっても,「内容は良いが説明が分かりにくい」場合と「説明は分かりやすいが内容が物足りない」場合では,必要な改善策は異なる.

評価スコアのみでは,このような要因の内訳を把握することが困難であり,具体的な改善につなげにくいという課題がある.

\subsection{自由記述分析の困難性}
自由記述は非構造データであり,表記ゆれ,略語,誤字脱字など,多様な表現が含まれる.また,同じ内容でも学生によって表現方法が異なるため,単純な文字列マッチングでは十分な分析ができない.

人的読解に依存する場合,分析に膨大な時間がかかるだけでなく,読解者の主観や疲労により分析の一貫性が損なわれる可能性がある.本研究の対象データでは83,851件の自由記述が存在し,すべてを人力で読解することは現実的ではない.

\subsection{感情情報の未活用}
授業評価スコアは授業の「評価」を示すが,学生の「感情」や「満足度」を直接測定するものではない.同じ評価スコアであっても,学生の感情的な満足度には差異がある可能性がある.

例えば,「勉強になったが楽しくなかった」授業と「楽しかったが勉強にならなかった」授業は,評価スコアが同程度であっても,学生の感情的体験は大きく異なる.評価スコアのみでは,このような感情的側面を把握することが困難である.

\subsection{改善施策の優先順位付け}
教育改善に投入できる資源は限られており,すべての課題に同時に対応することは困難である.どの要因を優先的に改善すべきかを客観的に判断するためには,各要因の影響度を定量的に把握する必要がある.

しかし,従来の分析手法では,このような優先順位付けを行うための定量的な根拠を得ることが難しかった.

%%%%%%%%%%%%%%%%%%%%%%%%%%%%%%%%%%%%%%%%%%%%%%%%%%%%%%%%%%%%%%%%%%%%%%%%%%%%%%%
\section{研究目的}
%%%%%%%%%%%%%%%%%%%%%%%%%%%%%%%%%%%%%%%%%%%%%%%%%%%%%%%%%%%%%%%%%%%%%%%%%%%%%%%

本研究の目的は,授業評価アンケートの自由記述から感情スコアを推定し,授業評価スコアとの関係を分析した上で,共通要因と特化要因を調査することにより,授業評価スコアの高さと関連する要因を定量的に特定することである.

具体的には,以下の3つのサブ目的を設定する.

\begin{enumerate}
\item \textbf{感情スコアと評価スコアの関係解明}: 自由記述の感情分析により感情スコアを算出し,授業単位で集計した感情スコアと授業評価スコアの関係を統計的に検討する.

\item \textbf{共通要因と特化要因の調査}: 感情スコアと評価スコアを同時に予測するマルチタスク学習モデルを構築し,両者に共通する要因(満足度要因)と,それぞれに特有の要因(感情特化要因・評価特化要因)を整理することで,授業評価スコアの高さと関連する要因を抽出する.

\item \textbf{要因の定量化と可視化}: SHAP分析を用いて単語レベルの寄与度を定量化し,高い授業評価と関連する語彙・要因を可視化する.
\end{enumerate}

%%%%%%%%%%%%%%%%%%%%%%%%%%%%%%%%%%%%%%%%%%%%%%%%%%%%%%%%%%%%%%%%%%%%%%%%%%%%%%%
\section{研究仮説}
%%%%%%%%%%%%%%%%%%%%%%%%%%%%%%%%%%%%%%%%%%%%%%%%%%%%%%%%%%%%%%%%%%%%%%%%%%%%%%%

本研究では,以下の2つの仮説を設定する.

\textbf{仮説1}: 授業単位で集約した感情スコアと授業評価スコアには正の相関関係がある.

この仮説は,学生の感情的満足度と授業評価が関連しているという前提に基づく.自由記述にポジティブな感情が多く表れる授業は,評価スコアも高い傾向があると予想される.

\textbf{仮説2}: 感情スコアと授業評価スコアの両方と関連する共通要因(満足度要因)が存在する.

感情と評価は異なる概念であるが,両者に共通して関連する要因が存在すると考えられる.例えば,「分かりやすさ」は感情的満足度と評価スコアの双方と関連する可能性がある.

%%%%%%%%%%%%%%%%%%%%%%%%%%%%%%%%%%%%%%%%%%%%%%%%%%%%%%%%%%%%%%%%%%%%%%%%%%%%%%%
\section{研究のアプローチと特徴}
%%%%%%%%%%%%%%%%%%%%%%%%%%%%%%%%%%%%%%%%%%%%%%%%%%%%%%%%%%%%%%%%%%%%%%%%%%%%%%%

\subsection{BERTによる感情分類}
本研究では,日本語の事前学習済みBERTモデル(Bidirectional Encoder Representations from Transformers)を基盤とした感情分類モデルを構築する\cite{bert}.BERTは双方向の文脈情報を考慮できる言語モデルであり,少量の教師データでも高精度な分類が可能である\cite{cl-tohoku}.

83,851件の自由記述に対して感情スコア(ポジティブ/ニュートラル/ネガティブ)を推定し,授業単位で集計することで,授業レベルの感情スコアを算出する.

\subsection{マルチタスク学習}
感情スコアと授業評価スコアを同時に予測するマルチタスク学習モデルを構築する\cite{mtl}.BERTエンコーダを共有表現として使用し,感情分類ヘッドと評価スコア予測ヘッドを分岐させる構成とする.

この構成により,両タスクに共通する特徴(共通要因)と,各タスクに固有の特徴(特化要因)を分離することが可能となる.

\subsection{SHAP分析による要因の可視化}
モデルの解釈可能性を高めるため,SHAP(SHapley Additive exPlanations)分析を実施する\cite{shap}.SHAPは協力ゲーム理論に基づく特徴量重要度の算出手法であり,単語レベルでの寄与度を定量化できる.

SHAP分析により,どの語彙が感情スコアや評価スコアに寄与しているかを明らかにし,授業改善に向けた具体的な示唆を得る.

%%%%%%%%%%%%%%%%%%%%%%%%%%%%%%%%%%%%%%%%%%%%%%%%%%%%%%%%%%%%%%%%%%%%%%%%%%%%%%%
\section{研究の意義}
%%%%%%%%%%%%%%%%%%%%%%%%%%%%%%%%%%%%%%%%%%%%%%%%%%%%%%%%%%%%%%%%%%%%%%%%%%%%%%%

\subsection{学術的意義}
本研究は,以下の点で学術的意義を有する.

第一に,評価スコアと自由記述を統合的に分析し,共通要因・特化要因の観点から高い授業評価と関連する要因を定量的に整理する点である.従来の研究では,評価スコアと自由記述を別々に分析することが多く,統合的な要因整理は限定的であった.

第二に,マルチタスク学習とSHAP分析を組み合わせ,授業評価スコアの高さと関連する語彙・要因を可視化する分析フレームワークを提示する点である.この方法論は,教育分野に限らず,複数の評価指標が存在する場面での要因分析に応用可能である.

\subsection{実践的意義}
本研究は,以下の点で実践的意義を有する.

第一に,大規模な自由記述データの効率的な分析手法を提供する点である.83,851件の自由記述を自動的に分類することで,人的コストを大幅に削減できる.

第二に,授業改善の優先順位を客観的に決定できる基盤を提供する点である.共通要因・感情特化要因・評価特化要因の区別により,高い授業評価と関連する要因に基づいた改善施策を設計できる.

第三に,データに基づく教育改善(Evidence-Based Education)の実現に貢献する点である.

%%%%%%%%%%%%%%%%%%%%%%%%%%%%%%%%%%%%%%%%%%%%%%%%%%%%%%%%%%%%%%%%%%%%%%%%%%%%%%%
\section{本研究の構成}
%%%%%%%%%%%%%%%%%%%%%%%%%%%%%%%%%%%%%%%%%%%%%%%%%%%%%%%%%%%%%%%%%%%%%%%%%%%%%%%

本研究は,全5章からなる.各章の概要を以下に示す.

\textbf{第2章 関連研究}: 授業評価研究,自然言語処理による感情分析,BERTと言語モデル,マルチタスク学習,解釈可能AI(SHAP)に関する関連研究を整理し,本研究の位置づけを明確にする.

\textbf{第3章 データと手法}: 本研究で使用するデータセット(3,268授業,83,851件自由記述)の概要,前処理手順,感情分類モデルの構成,マルチタスク学習モデルの構成,SHAP分析の手法,および評価指標を述べる.

\textbf{第4章 結果と考察}: 感情スコアと授業評価スコアの相関分析結果,感情分類モデルの性能,SHAP分析による要因抽出結果を報告し,得られた知見を考察する.

\textbf{第5章 おわりに}: 本研究の成果を総括し,研究の限界および今後の課題を述べる.

%%%%%%%%%%%%%%%%%%%%%%%%%%%%%%%%%%%%%%%%%%%%%%%%%%%%%%%%%%%%%%%%%%%%%%%%%%%%%%%
