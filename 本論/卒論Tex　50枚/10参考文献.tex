\begin{thebibliography}{99}
\addcontentsline{toc}{chapter}{参考文献}

% === 深層学習・言語モデル ===

\bibitem{bert} Devlin, J., Chang, M.-W., Lee, K., and Toutanova, K.: ``BERT: Pre-training of Deep Bidirectional Transformers for Language Understanding,'' \textit{Proceedings of the 2019 Conference of the North American Chapter of the Association for Computational Linguistics: Human Language Technologies (NAACL-HLT)}, pp. 4171--4186 (2019).

\bibitem{transformer} Vaswani, A., Shazeer, N., Parmar, N., Uszkoreit, J., Jones, L., Gomez, A. N., Kaiser, L., and Polosukhin, I.: ``Attention Is All You Need,'' \textit{Advances in Neural Information Processing Systems 30 (NeurIPS 2017)} (2017).

\bibitem{cl-tohoku} 東北大学乾・鈴木研究室: ``日本語BERT事前学習モデル,'' https://github.com/cl-tohoku/bert-japanese (2019).

% === 解釈可能AI ===

\bibitem{shap} Lundberg, S. M., and Lee, S.-I.: ``A Unified Approach to Interpreting Model Predictions,'' \textit{Advances in Neural Information Processing Systems 30 (NeurIPS 2017)} (2017).

% === マルチタスク学習 ===

\bibitem{mtl} Zhang, Y., and Yang, Q.: ``A Survey on Multi-Task Learning,'' \textit{IEEE Transactions on Knowledge and Data Engineering}, Vol. 34, No. 12, pp. 5586--5609 (2022).

% === 感情分析 ===

\bibitem{liu2012} Liu, B.: ``Sentiment Analysis and Opinion Mining,'' \textit{Synthesis Lectures on Human Language Technologies}, Vol. 5, No. 1, pp. 1--167 (2012).

% === 教育分野の自由記述分析 ===

\bibitem{gottipati2018} Gottipati, S., Shankararaman, V., and Lin, J. R.: ``Text Analytics Approach to Extract Course Improvement Suggestions from Students' Feedback,'' \textit{Research and Practice in Technology Enhanced Learning}, Vol. 13, Article 6 (2018).

\bibitem{hujala2020} Hujala, M., Knutas, A., Hynninen, T., and Arminen, H.: ``Improving the Quality of Teaching by Utilising Written Student Feedback: A Streamlined Process,'' \textit{Computers \& Education}, Vol. 157, Article 103965 (2020).

% === 授業評価研究 ===

\bibitem{marsh2007} Marsh, H. W.: ``Students' Evaluations of University Teaching: Dimensionality, Reliability, Validity, Potential Biases and Usefulness,'' \textit{The Scholarship of Teaching and Learning in Higher Education: An Evidence-Based Perspective}, pp. 319--383, Springer (2007).

\bibitem{spooren2013} Spooren, P., Brockx, B., and Mortelmans, D.: ``On the Validity of Student Evaluation of Teaching: The State of the Art,'' \textit{Review of Educational Research}, Vol. 83, No. 4, pp. 598--642 (2013).

\end{thebibliography}
