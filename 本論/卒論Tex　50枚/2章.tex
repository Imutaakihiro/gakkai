\chapter{関連研究}

本章では,本研究に関連する先行研究を整理する.授業評価研究,感情分析,BERT,マルチタスク学習,解釈可能AIについて概観し,本研究の位置づけを明確にする.

%%%%%%%%%%%%%%%%%%%%%%%%%%%%%%%%%%%%%%%%%%%%%%%%%%%%%%%%%%%%%%%%%%%%%%%%%%%%%%%
\section{授業評価研究}
%%%%%%%%%%%%%%%%%%%%%%%%%%%%%%%%%%%%%%%%%%%%%%%%%%%%%%%%%%%%%%%%%%%%%%%%%%%%%%%

\subsection{授業評価の意義と歴史}
大学における授業評価は,教育の質向上に向けた重要な指標として広く用いられている\cite{marsh2007}.授業評価は1920年代にアメリカの大学で始まり,1970年代以降に世界的に普及した\cite{spooren2013}.日本においても,1990年代後半から多くの大学で導入が進み,現在ではほぼすべての大学で実施されている.

授業評価の主な目的は,(1) 教員へのフィードバックによる授業改善,(2) 人事評価の参考資料,(3) 学生への授業選択情報の提供,の3点である\cite{spooren2013}.これらの目的の比重は大学の制度や運用方針によって異なる可能性がある.

\subsection{授業評価の構成要素}
多くの大学では,学期末のアンケートにより授業評価スコアと自由記述を収集し,教員へフィードバックを行っている.授業評価スコアは数量的に扱いやすい一方で,学生が評価に至った理由や具体的な改善要望は自由記述に含まれることが多い\cite{hujala2020}.

自由記述を定量的に分析し,授業評価スコアの背後にある要因を明らかにする研究が必要とされている.さらに,授業改善への活用を前提とした分析プロセスの標準化や,フィードバックの迅速化が課題として指摘されており,効率的な分析手法の整備が求められる\cite{hujala2020}.

\subsection{授業評価の信頼性と妥当性}
授業評価の信頼性・妥当性については多くの議論がある.Marshは,授業評価が多次元的な構造を持ち,異なる側面(例:明確さ,組織性,熱意)を測定していることを示した\cite{marsh2007}.一方で,評価が成績期待や授業難易度に影響される可能性も指摘されている\cite{spooren2013}.

%%%%%%%%%%%%%%%%%%%%%%%%%%%%%%%%%%%%%%%%%%%%%%%%%%%%%%%%%%%%%%%%%%%%%%%%%%%%%%%
\section{自由記述分析と感情分析}
%%%%%%%%%%%%%%%%%%%%%%%%%%%%%%%%%%%%%%%%%%%%%%%%%%%%%%%%%%%%%%%%%%%%%%%%%%%%%%%

\subsection{自由記述の特性}
自由記述は非構造テキストであり,従来は人的な読解に依存していた.しかし,大規模なデータでは人的読解に限界があり,分析者の主観による解釈のばらつきも問題となる.近年の自然言語処理技術の発展により,大規模な自由記述を自動的に解析し,感情や評価の傾向を抽出することが可能になっている\cite{liu2012}.

\subsection{感情分析の基礎}
感情分析(Sentiment Analysis)は,テキストに含まれる肯定的・否定的・中立的な感情を推定する技術である\cite{liu2012}.感情分析は,ソーシャルメディアの分析,製品レビューの分析,顧客満足度調査など,様々な分野で活用されている.

教育分野においても,自由記述の感情傾向を把握する試みが進められている.

\subsection{教育分野での自由記述分析}
教育分野の自由記述分析では,テキスト分析を通じた改善提案の抽出や意見整理が報告されている\cite{gottipati2018,hujala2020}.Gottipatiらは,学生のフィードバックから授業改善の提案を自動抽出するテキスト分析手法を提案した\cite{gottipati2018}.Hujalaらは,学生の自由記述を活用した授業改善プロセスの整理を報告している\cite{hujala2020}.

これらの研究は,従来のスコア中心の評価を補完する手段としての自由記述分析の重要性を示している.一方で,評価スコアとの統合的分析や改善施策への接続は限定的である可能性がある.

%%%%%%%%%%%%%%%%%%%%%%%%%%%%%%%%%%%%%%%%%%%%%%%%%%%%%%%%%%%%%%%%%%%%%%%%%%%%%%%
\section{感情分析の手法分類}
%%%%%%%%%%%%%%%%%%%%%%%%%%%%%%%%%%%%%%%%%%%%%%%%%%%%%%%%%%%%%%%%%%%%%%%%%%%%%%%

感情分析の手法は大きく,(1) 辞書型手法,(2) 古典的機械学習手法,(3) 深層学習手法に分類できる.本研究では深層学習手法を中心に扱い,他手法との差異を踏まえて位置づける.

\subsection{辞書型手法}
辞書型手法は,極性語彙(ポジティブ・ネガティブな単語のリスト)を用いて感情を推定する手法である.実装が容易で解釈しやすい利点がある一方,文脈依存の表現や否定表現に弱いという欠点がある.

\subsection{古典的機械学習手法}
古典的機械学習手法(SVM,ナイーブベイズ,ランダムフォレストなど)は,特徴量設計により一定の精度を得られる.しかし,語彙の多様性が大きい自由記述では特徴量設計の負担が大きい.

\subsection{深層学習手法}
深層学習手法(RNN,LSTM,Transformerなど)は,文脈を自動的に考慮できる利点がある.一方で,教師データの準備コストが高く,教育分野固有の語彙や表現への適応が課題となる\cite{bert}.近年は事前学習済みモデルの活用により,少量の教師データでも高精度な分類が可能になっている.

%%%%%%%%%%%%%%%%%%%%%%%%%%%%%%%%%%%%%%%%%%%%%%%%%%%%%%%%%%%%%%%%%%%%%%%%%%%%%%%
\section{BERTと事前学習済み言語モデル}
%%%%%%%%%%%%%%%%%%%%%%%%%%%%%%%%%%%%%%%%%%%%%%%%%%%%%%%%%%%%%%%%%%%%%%%%%%%%%%%

\subsection{BERTの概要と日本語モデル}
BERT(Bidirectional Encoder Representations from Transformers)は,Transformerのエンコーダ部分を用いた事前学習済み言語モデルである\cite{bert}.Vaswaniらが提案したTransformerアーキテクチャ\cite{transformer}を基盤とし,双方向の文脈情報を同時に考慮できる点が特長である.

BERTは,Masked Language Model(MLM)タスクとNext Sentence Prediction(NSP)タスクにより事前学習される.大規模コーパスで事前学習されたモデルを特定タスクに微調整することで,少量の教師データでも高精度な分類が可能である\cite{bert}.

日本語に対しても事前学習済みBERTモデルが複数提供されている.東北大学が公開した「cl-tohoku/bert-base-japanese」は,日本語Wikipediaで事前学習されたモデルであり,日本語NLPタスクで広く利用されている\cite{cl-tohoku}.

教育分野の自由記述に対して微調整を行うことで,文脈を考慮した感情分類が実現できる.一方で,学習データの領域差が大きい場合には汎化性能が低下する可能性があるため,教育分野に特化した微調整と評価設計が必要となる.

%%%%%%%%%%%%%%%%%%%%%%%%%%%%%%%%%%%%%%%%%%%%%%%%%%%%%%%%%%%%%%%%%%%%%%%%%%%%%%%
\section{マルチタスク学習}
%%%%%%%%%%%%%%%%%%%%%%%%%%%%%%%%%%%%%%%%%%%%%%%%%%%%%%%%%%%%%%%%%%%%%%%%%%%%%%%

\subsection{マルチタスク学習の概要と本研究への関係}
マルチタスク学習は,関連する複数のタスクを同時に学習し,共通の表現を共有する学習枠組みである\cite{mtl}.個別に学習する場合と比べて,データが少ない場面で学習が安定する可能性がある.

主な利点として,関連タスクからの情報共有による性能向上や過学習の抑制が挙げられる.一方で,タスク間の関連が弱い場合には効果が限定的になる可能性がある.

本研究では,感情スコア推定と授業評価スコア予測を同時に学習し,両者に共通する要因と特化要因を整理する手がかりを得ることを目的とする.この枠組みにより,授業評価スコアの高さと関連する語彙・要因を具体的に抽出できる可能性がある.

%%%%%%%%%%%%%%%%%%%%%%%%%%%%%%%%%%%%%%%%%%%%%%%%%%%%%%%%%%%%%%%%%%%%%%%%%%%%%%%
\section{解釈可能AIとSHAP}
%%%%%%%%%%%%%%%%%%%%%%%%%%%%%%%%%%%%%%%%%%%%%%%%%%%%%%%%%%%%%%%%%%%%%%%%%%%%%%%

\subsection{SHAPの概要とテキスト分析への適用}
機械学習モデルの予測根拠を明確化するため,解釈可能AI(Explainable AI: XAI)が注目されている.特に教育分野では,モデルの予測精度だけでなく説明可能性が重要であり,改善施策への翻訳可能性が求められる.

SHAP(SHapley Additive exPlanations)は,協力ゲーム理論のShapley値に基づき,特徴量の寄与度を定量化する手法である\cite{shap}.局所的な説明と大域的な説明の両方が可能であり,さまざまなモデルに適用できる点が特長である\cite{shap}.

テキスト分類では単語レベルの寄与度を算出できるため,授業評価に関連する語彙を具体的に提示できる.ただし,寄与度の解釈は文脈に依存するため,定性的な検討との併用が必要となる.

%%%%%%%%%%%%%%%%%%%%%%%%%%%%%%%%%%%%%%%%%%%%%%%%%%%%%%%%%%%%%%%%%%%%%%%%%%%%%%%
\section{教育分野への応用研究}
%%%%%%%%%%%%%%%%%%%%%%%%%%%%%%%%%%%%%%%%%%%%%%%%%%%%%%%%%%%%%%%%%%%%%%%%%%%%%%%

教育分野では,自由記述を対象としたテキスト分析や意見抽出の取り組みが報告されている\cite{gottipati2018,hujala2020}.しかし,評価スコアとの関係性を統合的に扱った研究は多くない.このため,感情分析・マルチタスク学習・解釈可能AIを組み合わせた総合的な分析枠組みの構築が求められている.

%%%%%%%%%%%%%%%%%%%%%%%%%%%%%%%%%%%%%%%%%%%%%%%%%%%%%%%%%%%%%%%%%%%%%%%%%%%%%%%
\section{既存研究の限界と課題}
%%%%%%%%%%%%%%%%%%%%%%%%%%%%%%%%%%%%%%%%%%%%%%%%%%%%%%%%%%%%%%%%%%%%%%%%%%%%%%%

既存研究には以下の課題がある.これらは授業改善に直結する知見の整理を難しくする要因となり得る.

第一に,評価スコアと自由記述の統合が不十分である.多くの研究は評価スコアの分析または自由記述の分析を別々に行っており,両者の関係を同時にモデル化した研究は限られている\cite{gottipati2018,hujala2020}.

第二に,感情分析結果の解釈が定性的で,教育改善に直結しづらい.感情をポジティブ・ネガティブに分類するだけでは,具体的な改善策の導出が困難である.

第三に,予測精度と説明可能性の両立が難しい.高精度な深層学習モデルはブラックボックス化しやすく,教育改善への翻訳が困難である.

%%%%%%%%%%%%%%%%%%%%%%%%%%%%%%%%%%%%%%%%%%%%%%%%%%%%%%%%%%%%%%%%%%%%%%%%%%%%%%%
\section{本研究の位置づけ}
%%%%%%%%%%%%%%%%%%%%%%%%%%%%%%%%%%%%%%%%%%%%%%%%%%%%%%%%%%%%%%%%%%%%%%%%%%%%%%%

本研究は,授業評価アンケートの自由記述に対し,BERTによる感情分類とマルチタスク学習を適用し,さらにSHAP分析によって評価要因を定量化する点に特徴がある.評価スコアと自由記述を同じ枠組みで扱う点に独自性がある.

既存研究との差異を表\ref{tab:comparison}に示す.比較項目は,評価スコア分析,自由記述分析,統合分析,要因の定量化に限定して整理した.

\begin{table}[t]
    \centering
    \caption{既存研究との比較}
    \label{tab:comparison}
    \resizebox{0.9\textwidth}{!}{
    \begin{tabular}{l c c c c}
        \toprule
        研究 & 評価スコア分析 & 自由記述分析 & 統合分析 & 要因の定量化 \\
        \midrule
        Gottipati et al. (2018) & − & ○ & − & − \\
        Hujala et al. (2020) & ○ & ○ & △ & − \\
        \textbf{本研究} & \textbf{○} & \textbf{○} & \textbf{○} & \textbf{○} \\
        \bottomrule
    \end{tabular}
    }
\end{table}

本研究の新規性は以下の3点である.以下では先行研究との比較に基づき整理する.

\begin{enumerate}
\item 感情スコアと授業評価スコアを同時に学習するマルチタスクモデルを構築し,両者の関係を統合的にモデル化する.
\item SHAP分析により,共通要因と特化要因を語彙レベルで整理し,改善施策の優先順位付けに利用できる定量的根拠を提供する.
\item 3,268授業,83,851件の自由記述という大規模データを用いて,統計的に信頼性の高い分析を行う.
\end{enumerate}

これにより,教育改善に資する具体的な知見を提供することを目指す.得られた知見は授業改善の意思決定に活用できる可能性がある.
