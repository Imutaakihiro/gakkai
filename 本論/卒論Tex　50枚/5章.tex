%-----------------------------
\chapter{おわりに}
%-----------------------------

本章では,本研究の成果を総括し,研究の限界,今後の課題,および研究の意義について述べる.

%%%%%%%%%%%%%%%%%%%%%%%%%%%%%%%%%%%%%%%%%%%%%%%%%%%%%%%%%%%%%%%%%%%%%%%%%%%%%%%
\section{結論}
%%%%%%%%%%%%%%%%%%%%%%%%%%%%%%%%%%%%%%%%%%%%%%%%%%%%%%%%%%%%%%%%%%%%%%%%%%%%%%%

\subsection{研究目的の達成}
本研究は,授業評価アンケートの自由記述から感情スコアを推定し,授業評価スコアとの関係を分析した上で,共通要因と特化要因を調査することにより,授業評価スコアの高さと関連する要因を定量的に特定することを目的とした.この目的に対し,以下の成果を得た.

第一に,授業単位で集約した感情スコアと授業評価スコアの相関分析を行った結果,ピアソン相関係数0.3097($p<0.000001$)の中程度の正の相関が確認された.スピアマン順位相関係数(0.2970)およびケンドール順位相関係数(0.2042)においても同様に統計的に有意な正の相関が得られ,複数の指標で一貫した結果となった.これにより,学生の自由記述に表れる感情と授業評価スコアには一定の関係があることが示された.

第二に,BERTを基盤とした感情分類モデルを用いて自由記述の感情スコアを推定し,授業単位で集約した指標を構築した.

第三に,感情スコアと授業評価スコアを同時に予測するマルチタスク学習モデルを構築し,SHAP分析により評価要因を定量化した.その結果,1,564語を4つの要因グループ(共通要因18.0\%,感情特化要因37.5\%,評価特化要因16.6\%,低重要度要因27.8\%)に分類することに成功した.

\subsection{仮説の検証}
本研究で設定した2つの仮説について,以下の結果が得られた.

\textbf{仮説1}「授業単位で集約した感情スコアと授業評価スコアには正の相関関係がある」については,相関分析により統計的に有意な正の相関が確認され,\textbf{支持された}.

\textbf{仮説2}「感情スコアと授業評価スコアの両方と関連する共通要因(満足度要因)が存在する」については,SHAP分析により577語(18.0\%)の共通要因が抽出され,「学ぶ」「理解」「総括」などの語彙が感情と評価の双方と関連することが示された.したがって,本仮説は\textbf{支持された}.

\subsection{主要な発見}
本研究における主要な発見は以下の通りである.

\begin{enumerate}
\item \textbf{理解しやすさの重要性}: ポジティブ判定に最も寄与する語彙は「やす」(分かりやすい,理解しやすいの語幹)であり,授業内容の理解しやすさが学生満足度の中心的要因であることが示唆された.

\item \textbf{興味・関心の影響}: 「面白」「おもしろ」「楽しい」といった語彙が上位に位置しており,授業への興味・関心が感情評価に強く影響することが明らかになった.

\item \textbf{共通要因の存在}: 感情スコアと評価スコアの双方に寄与する共通要因(18.0\%)が存在し,これらへの対応が効率的な授業改善につながる可能性が示された.

\item \textbf{要因の分離可能性}: マルチタスク学習とSHAP分析の組み合わせにより,満足感に関わる要因と評価に関わる要因を定量的に分離できることが実証された.
\end{enumerate}

%%%%%%%%%%%%%%%%%%%%%%%%%%%%%%%%%%%%%%%%%%%%%%%%%%%%%%%%%%%%%%%%%%%%%%%%%%%%%%%
\section{研究の限界}
%%%%%%%%%%%%%%%%%%%%%%%%%%%%%%%%%%%%%%%%%%%%%%%%%%%%%%%%%%%%%%%%%%%%%%%%%%%%%%%

本研究には以下の限界がある.

\subsection{因果関係の未検証}
本研究は相関関係の探索を目的としており,因果関係の検証は行っていない.例えば,「分かりやすい」という語彙がポジティブ判定に寄与することは示されたが,授業を「分かりやすく」すれば評価が向上するという因果的主張は本研究からは導けない.因果関係の検証には,介入研究や実験的デザインが必要である.

\subsection{データの限定性}
本研究のデータは福岡工業大学の1大学に限定されており,他大学への一般化可能性には限界がある.大学の規模,学部構成,学生層,教育文化などが異なる環境では,異なる結果が得られる可能性がある.

\subsection{教師データの制約}
教師データは1,000件と比較的少なく,ラベル付けは単一の評価者により行われたため,主観性が残る.また,クラス不均衡(ニュートラル62.8\%,ネガティブ19.1\%,ポジティブ18.0\%)が存在し,少数クラスの分類精度に影響を与えている可能性がある.

\subsection{モデルの解釈性}
BERTは高い性能を示す一方で,その予測根拠は必ずしも人間にとって直感的ではない.SHAP分析により解釈可能性を高めたものの,サブワード単位の分析では語彙の意味を完全に捉えられない場合がある.

\subsection{時間的変化の未考慮}
本研究では2018年度から2023年度までの6年間のデータを一括して分析したが,教育環境や学生の価値観は時間とともに変化している可能性がある.特に2020年以降のCOVID-19の影響によるオンライン授業の増加は,評価傾向に変化をもたらした可能性がある.

%%%%%%%%%%%%%%%%%%%%%%%%%%%%%%%%%%%%%%%%%%%%%%%%%%%%%%%%%%%%%%%%%%%%%%%%%%%%%%%
\section{今後の課題}
%%%%%%%%%%%%%%%%%%%%%%%%%%%%%%%%%%%%%%%%%%%%%%%%%%%%%%%%%%%%%%%%%%%%%%%%%%%%%%%

本研究の結果を踏まえ,以下の課題が今後の研究として挙げられる.

\subsection{因果関係の検証}
本研究で示唆された要因(理解しやすさ,面白さなど)が実際に評価向上に寄与するかを検証するため,介入研究が必要である.具体的には,特定の授業に対して共通要因に基づく改善を施し,その前後での評価変化を測定する準実験的デザインが考えられる.

\subsection{一般化可能性の検証}
複数大学のデータを用いた比較分析により,本研究の知見の一般性を検証する必要がある.また,学部・学科ごとの分析を行うことで,専門分野による評価要因の違いを明らかにすることも重要である.

\subsection{モデルの精度向上}
教師データの拡充や半教師あり学習の導入により,感情分類モデルの精度向上が期待される.また,ドメイン適応技術を用いて,教育分野に特化した言語モデルを構築することも有効であると考えられる.

\subsection{縦断的分析}
年度ごとの評価傾向の変化や,同一教員の授業における経年変化を分析することで,教育改善の効果測定や長期的なトレンドの把握が可能になると考えられる.

%%%%%%%%%%%%%%%%%%%%%%%%%%%%%%%%%%%%%%%%%%%%%%%%%%%%%%%%%%%%%%%%%%%%%%%%%%%%%%%
\section{研究の意義}
%%%%%%%%%%%%%%%%%%%%%%%%%%%%%%%%%%%%%%%%%%%%%%%%%%%%%%%%%%%%%%%%%%%%%%%%%%%%%%%

本研究は,以下の点で学術的・実践的意義を有する.

\subsection{学術的意義}
\begin{itemize}
\item \textbf{統合分析の提示}: 評価スコアと自由記述を同一の枠組みで扱い,授業評価の要因構造を定量的に整理した.

\item \textbf{要因タイプの整理}: 共通要因と特化要因の存在を語彙レベルで示し,評価と感情の関係を具体的に記述した.

\item \textbf{知見の蓄積}: 授業評価における感情要因の役割について,データに基づく知見を蓄積した.
\end{itemize}

\subsection{実践的意義}
\begin{itemize}
\item \textbf{改善の優先順位付け}: 共通要因・感情特化要因・評価特化要因の区別により,授業改善の優先順位を客観的に決定できる基盤を提供した.

\item \textbf{効率的な資源配分}: 共通要因への投資により,限られた資源で感情と評価の双方を向上させる戦略を提示した.
\end{itemize}

%%%%%%%%%%%%%%%%%%%%%%%%%%%%%%%%%%%%%%%%%%%%%%%%%%%%%%%%%%%%%%%%%%%%%%%%%%%%%%%

本研究では,授業評価アンケートの自由記述に対して感情分析を適用し,授業評価スコアとの関係性を分析した.その結果,感情スコアと授業評価スコアに統計的に有意な正の相関があること,理解しやすさや面白さが満足度に強く寄与すること,マルチタスク学習により要因を整理できることを明らかにした.これらの知見は,授業評価の要因理解に資する基盤を提供するものである.
