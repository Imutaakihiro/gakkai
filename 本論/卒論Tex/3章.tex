\chapter{GNLモデル}
本節では,まずGNLモデルの選択確率を示す.次に,GNLモデルに基づく集計問題の発生例を示し,集計ルールが満たすべき条件について論じる.その後,ブランド選択モデルにおけるネストと選択肢,それぞれの集計ルールを示す.

\section{選択確率}
本研究では,GNLモデルをブランド選択モデルに適用し,データ匿名化が選択モデルに与える影響を検証する.GNLモデルは,選択行動を分析するための確率モデルの一つであり,特に選択肢間の相関性を考慮した分析が可能な点で優れている.このモデルを使用し,匿名化によるデータ構造の変化が消費者行動やモデルの性能に及ぼす影響を詳細に評価することを目的とする.

GNLモデルでは,ネスティングルール(選択肢を階層的に構造化する規則)を商品の属性に基づいて定義する.
また,本研究では消費者が同質であると仮定し,すべての消費者が同じ選択構造に従うものとする.
選択肢の構造としては,商品$l$が特定の構成要素$m$を持つものとし,消費者はまず構成要素$m$を選択した後,その構成要素$m$を持つ商品$l$を選ぶものとする.
このプロセスは2段階で構成されており,第一段階では構成要素$m$の選択が行われ,第二段階でその構成要素を持つ商品が選ばれる.
このような選択プロセスは,消費者がまず商品カテゴリや属性(例えば「低価格」「大容量」など)に基づいて候補を絞り込み,
その後ブランド名やその他の要素を基に具体的な商品を選択する現実の購買行動に即している.

ここで,同一の商品$l$が複数の構成要素$m'$に属することは許されないとする.つまり,各ブランドは一意に特定の構成要素に紐付けられる.一方で,同一の構成要素$m'$には複数の商品が属することが可能である.このルールにより,商品と構成要素の関係は一対一ではなく,一対多の関係として表現される.このため,消費者が特定の構成要素$m$を選択すると,その構成要素に属する商品$l' \in \mathcal{L}_m$のみが選択肢として残ることになる.このようにして,消費者が選択するブランドの候補が構成要素の選択によって限定される点が,本モデルの重要な特徴である.
以上のモデル構造により,GNLモデルは選択肢間の相関性や階層的な選択プロセスを考慮した精密な選択行動の解析を可能にする.

構成要素$m$を通じて選択された商品$l$の効用$\hat U_{ml}$は,次のように表される:
\begin{align}
\hat U_{ml} &= V_m + \epsilon_{l}, \\
V_{l} &= \hat V_{m} + \hat V_{ml} + \hat V_{l}, \\
\epsilon_{l} &= \hat{\epsilon}_l + \hat{\epsilon}_{ml}.
\end{align}
ここで,$\hat V_m$は商品構成要素(ネスト)$m$を選択した場合に得られる確定的効用を表し,$\hat V_{ml}$は商品$l$を構成要素$m$の下で選択した場合に得られる確定的効用を示す.また,$\hat{V}_l$は商品$l$による確定的効用を表し,$V_l$はこれらの効用の和として計算される総合的な効用である.
同様に,確率的効用については,$\hat{\epsilon}_m$が商品構成要素$m$を選択したことに対応する確率的効用を表し,$\hat{\epsilon}_{ml}$が商品$l$の残る確率的効用を示す.
これらの確率的効用は,選択行動の不確実性や選択肢間の個別性をモデル化するために用いられる.
 
WenとKoppelman\cite {wk}に基づき,式(3.4)~(3.6)の条件を適用して整理を行う.この整理により,ログサム形式を採用したGNLモデルにおけるブランド$l$の選択確率$\hat{P}_l$は,以下の式で表される:

\textbf{[GNL-LS]}
\begin{align}
\hat{P}_l = \sum_m \hat{P}_m \hat{P}_{l|m},
\end{align}
\begin{align}
\hat{P}_m = \frac{\exp \left( \hat{V}_m + V'_{m} \right)}{\sum_m \exp \left( \hat{V}_m + V'_{m} \right)}, 
\end{align}
\begin{align}
\hat{P}_{l|m} = \frac{\left( \gamma_{lm} \exp V_{l}^h \right)^{1/\mu_m}}{\sum_{l' \in \mathcal{N}_l} \left( \gamma_{l'm} \exp V_{l'}^h \right)^{1/\mu_m}}.
\end{align}
$\hat{P}_m$はネスト$m$を選択する確率で,$\hat{P}_{l|m}$はネスト$m$の下で商品$l$を選択する確率である.ここで,
\begin{align}
\hat{V}'_m := \mu_m \ln \sum_{l' \in \mathcal{N}_m} \left( \gamma_{l'm} \exp \left( \hat{V}_{ml'} + \hat{V}_{l'} \right) \right)^{1/\mu_m}
\end{align}
であり,$\hat{V}'_m$ は通常ログサムと呼ばれる.$N_m$は商品構成要素集合,$\mu_m$はネスト間の非類似度パラメータであり,効用最大化と整合的であるためには,
\begin{align}
0 &< \mu_m \leq 1
\end{align}
を満たす必要がある.同様に$\gamma_{lm}$はアロケーションパラメータであり,
\begin{align}
\gamma_{lm} \geq 0, \hspace{2em} \forall {m,l}, \\
\sum_m \gamma_{lm} = 1,  \hspace{2em} \forall {l}
\end{align}
を満たす必要がある.GNLモデルにおいて,属性分割(Attribute Separation)というネスティングルールを採用した場合,消費者はまずブランドの構成要素を選択し,その後に具体的なブランドを選択すると解釈される.GNLモデルにおけるパラメータの推定は,一般的に最尤推定法によって行われる.
最尤推定は,観測データに基づいてモデルのパラメータを推定する統計的手法であり,尤度関数を最大化することで最も適切なパラメータを求めることが可能である.
しかし,ネストや選択肢の数が増加すると,それに伴い非類似度パラメータやアロケーションパラメータの数も増加する.特にアロケーションパラメータは,ネスト数や選択肢数のいずれが増加しても比例的に増加するため,計算負荷が大きくなるという課題がある.このような問題を軽減するために,$k$匿名化を適用することで選択肢やネスト数を統合し,計算時間の短縮を図ることが可能となる.

\section{GNLモデルにおける集計問題の生成例}
本節では,GNLモデルにおいて集計問題が発生する生成例を説明するとともに,直感的に考えられる集計ルールでは克服できないことを例(ブランド選択モデル:紙パック牛乳) を用いて示す.
使用するデータの詳細については付録Bを参照されたい.

選択肢数は13個,そしてその選択構造は図3.1左図に示すように,ネスト数6つの構造としよう.
ここでのネスティングルールは,商品が持つ属性(値)をネストとして分類する属性分割(Attribute Separation)の手法を用いる.
ブランド$l$の確定的効用については,以下に示す線形効用関数で表されるものとする:
\begin{align}
V_l = \sum_{i=1}^{6} \alpha_i X_{il}.
\end{align}
ここで,$X_{1l}$はブランド(イオン)・ダミー(イオン=1,その他=0),$X_{2l}$はブランド(古谷)・ダミー(古谷=1,その他=0),$X_{3l}$はブランド(雪印)・ダミー(雪印=1,その他=0),$X_{4l}$は$l$の価格,$X_{5l}$は$l$の容量,$X_{6l}$は低温殺菌・ダミー(低温殺菌=1,普通=0),
$\alpha_i$は$i$番目の変数に対応するパラメータである.
$k$匿名化を適用するにあたり,ブランド名を「イオン」「古谷」「雪印」「その他」の4つのカテゴリに統合し,購入日時を「2週間ごと」または「1か月ごと」に一般化する.
当初,購入日時を「1週間ごと」に一般化することを検討したが,ブランド名を統合した「雪印」において,「2001年第6週」の購買機会が1回のみとなり,一意的な個人識別が可能となるリスクが生じる.
そのため,「1週間ごと」の一般化は比較対象から除外する.
匿名化前後の選択構造の変化については図3.1に示すように,匿名化前は13個,匿名化後は4個とする.このように一般化を伴う匿名化を行うことで,選択モデルに及ぼす影響を詳細に検証する.

\begin{figure}[t]
\centering
\includegraphics[scale=0.7]{gnl.png}
\caption{$k$匿名化統合前後のGNL構造}
\end{figure}

\section{集計ルールが満たすべき条件}
本節では,集計ルールが満たすべき条件について示す.
Sweet\cite {sr}は集計が整合的に行われているために満たすべき条件として,次の2つの条件を示している. \\
\textbf{条件1 固定需要} \\
個人$h$が集計化された選択肢$L$を選ぶ周辺確率$\hat P_L^h$は,統合前の選択肢$l$の効用$\hat V_l^h$を共通の効用である$\hat V_L^h$に置き換えても変化がない.\\
\textbf{条件2 選択確率の整合性} \\
個人$h$が集計化された選択肢$L$を選ぶ周辺確率$\hat P_L^h$は,統合前の選択肢$l$($l \in L$) の選択確率と等しい:
\begin{align}
\hat P_l = \sum_{l \in L} \hat P_l.
\end{align}
これ以外に,アロケーションパラメータの条件が必要となる.ここでは,次の条件提示する: \\
\textbf{条件3 アロケーションパラメータの整合性}
\begin{align}
\gamma_{lM} = \sum_{m \in M} \gamma_{lm}, \hspace{2em} \forall l.
\end{align}
条件3は集計が式(3.10) と整合的であるための条件である.
ただし,これはネストが集計化される場合についてのみの条件であり,選択肢が集計化される場合には適用することができない \cite {io}.

\section{GNLモデルにおける集計ルールの導入:ネストの集計}
本節では,\UTF{9AD9}橋 \cite{kt} が示すネストの集計ルールについて説明する.
ここでは,集計の例と同様に,まずネストとして商品構成要素$m$を選択し,その後に具体的な商品$l$を選択するものとする.ネスト(商品構成要素)は$m \in M$という表現を用いる.ここで,$m$はブランド統合前のネストを指し,$M$は統合後のネストを意味している.
本研究で求めるべき集計ルールとは,$\hat{V}_M$および$\hat{V}'_M$に関する条件を指す.
この条件をネスト$m$に適用し,さらにGNLの選択確率式(3.5) に代入することで,モデル全体の計算過程を整理する.
\begin{align}
\exp \left(\hat V_M + \hat V_{M'} \right) = \sum_{m \in M} \exp \left(\hat V_m +\hat V_{m'} \right)
\end{align}
を得る.ここで,条件1より
\begin{align}
\hat V_{M'} = \ln \sum_{m \in M} \exp \hat V'_m +\hat V_{M'} \hspace{2em} \forall m
\end{align}
であるため,式(3.14) は,
\begin{align}
\exp \left(\hat V_M +\hat V_{M'} \right) = \sum_{m \in M} \exp \left(\hat V_m +\hat V_{M'} \right)
\end{align}
と書き直せる.ここで,式(3.16) より$\hat V_M$に関する部分を消去し,
\begin{align}
\hat V_{M} = \ln \sum_{m \in M} \exp \hat V'_m
\end{align}
を得る.式(3.17) を式(3.16) に代入し,これを整理することにより,
\begin{align}
\hat V_{M} = \ln \sum_{m \in M} \exp \left(\hat V_m +\hat V_{m'} \right)  - \ln \sum_{m \in M} \exp \hat V'_m
\end{align}
となる.
これらの式のうち,式(3.17) と(3.18)がGNLにおけるブランド選択モデルでネストを集計した場合の満たすべき集計ルールとなる.
 
次に,ネストを集計した場合の集計ルール,ブランド統合前のネスト$\hat V_m$と統合後のネスト$\hat V_M$について示す.
\begin{align}
\hat{V}_{m} &= \sum_{m' \in N_m} \left( \gamma_{l'm} \exp \left( \hat{V}_{ml'} + \hat{V}_{l'} \right) \right)^{1/\mu_m},
\end{align}
\begin{align}
\hat{V}_{M} &= \ln \sum_{m \in M} \exp \hat{V}_m.
\end{align}
$V_m$と$\hat V_m$,$\hat V'_m$の関係は,式(3.2) 及び付録Aを参照せよ.

また,$V_m$についてはSweetでいうセントラル・ログサムにあたる.
従ってSweetと同様に,集計による各確定的効用項の範囲については次のとおりになる:
\begin{align}
\min_{m \in{M}} \hat V_{m} \leq \hat V_M \leq \max_{m \in{M}} \hat V_m.
\end{align}
$\hat V'_M$についても同様に,
\begin{align}
\min_{m \in{M}} \left(\hat V_{m}+\hat V'_{m} \right) - \max_{m \in{M}}\hat V'_m \leq \hat V_M \leq \max_{m \in{M}} \left(\hat V_{m}+\hat V'_{m} \right) - \min_{m \in{M}}\hat V'_m
\end{align}
を得る.$\hat V_Ml$については,$\hat V'_M$は式(3.22) に示すとおりであり,$\hat V_l$は集計と無関係であるため定数である.
従って,$\hat V_{Ml}$の上限,下限は最後の項$\Omega$に左右される.
$\Omega$は,$\hat{P}_{l|m}$を$\hat{P}m$で重み付けし、それを$\mu_m$乗したものを$\gamma{lM}$で除し、対数を取ったものである:
\begin{align}
\Omega = \ln \left( \frac{1}{\sum_{m \in M}\gamma_{lm}} \left(\frac{\sum_{m \in m}\hat P_{l|m} \hat P_m}{\sum_{m' \in M}\hat P_{m'}}\right)^{\mu_m}\right)
\end{align}
ここで,式(3.9) ,(3.10) より,
\begin{align}
0 \leq \sum_{m \in M}\gamma_{lm}=\gamma_{lM} \leq 1
\end{align}
である.一般的に$\gamma_lM$が大きい場合,つまり多くのネストを束ねた場合, $\Omega$はマイナスになりやすくなる.
また,式(3.8),$0 \leq \hat P_m \leq 1$,$0 \leq \hat P_{l|m} \leq 1$より,
\begin{align}
0 \leq  \left(\frac{\sum_{m \in M} \hat P_m \hat P_{l|m}}{\sum_{m' \in M}\hat P_{m'}}\right)^{\mu_m} \leq 1
\end{align}
である.一般的に,$\hat P_{l|m}$が大きい場合,すなわちネスト$m$に属する選択肢が魅力的である場合,$\Omega$は正の値をとりやすくなる.
最終的な$\Omega$の正負は,式(3.24)および式(3.25)の大小関係によって決定される.
しかし,$\Omega$の最大値や最小値を厳密に決定することはできず,同様に$V_{Ml}$についてもその範囲を確定することは困難である.
なお,本研究では,ブランド統合の前後においてネストの種類に変化がないことから,これらの計算は省略するものとする.

\section{GNLモデルにおける集計ルールの導入:選択肢の集計}
本節では,\UTF{9AD9}橋が示す選択肢の集計ルールについて説明する.
まずネストとして商品構成要素$m$を選択し,次に商品$l$を選択するものとする.選択肢(ブランド)を$l \in L$という表現を用いよう.
ここで$l$はブランド統合前の選択肢,$L$は統合後の選択肢を意味している.

ここで求めるべき集計ルールとは,$\hat V_{mL}$,$\hat V_{L}$に関しての条件を指す.条件2を$\hat P_{l|m}$,$\hat P_{L|m}$に適用することにより,
\begin{align}
\left(\gamma_{Lm}\exp \left(\hat V_{mL} +\hat V_{L}\right)\right)^{1/\mu_m} = \sum_{l \in L}\left(\gamma_{lm}\exp \left(\hat V_{ml} +\hat V_{l}\right)\right)^{1/\mu_m}
\end{align}
を得る.ここで条件1より,
\begin{align}
\left(\gamma_{Lm}\exp \left(\hat V_{mL} +\hat V_{L}\right)\right)^{1/\mu_m} = \sum_{l \in L}\left(\gamma_{lm}\exp \left(\hat V_{mL} +\hat V_{l}\right)\right)^{1/\mu_m}
\end{align}
となる.式(3.27) を$\hat V_L$について解くと,
\begin{align}
\hat V_L = \mu_m \ln \left(\frac{1}{\gamma_{Lm}}\sum_{l \in L}\left(\gamma \exp \hat V_l \right)^{1/ \mu_m} \right)
\end{align}
を得る.ちなみに式(3.28) の条件は,$\hat P'_{m}$に条件1を適用しても同様に得られる.次に,式(3.28) を式(3.26) に代入し,整理すると次の条件を得る:
\begin{align}
\hat V_{mL} = \mu_m \ln \left(\frac{1}{\gamma_{Lm}}\sum_{l \in L}\left(\gamma_{lm} \exp \hat V_{ml} \right)^{1/ \mu_m} \right) - \mu_m \ln \left(\frac{1}{\gamma_{Lm}}\sum_{l \in L}\left(\gamma_{lm} \exp \hat V_l \right)^{1/ \mu_m} \right)
\end{align}
これらの式のうち,式(3.28) と(3.29)が,GNLにおけるブランド選択モデルでネストを集計した場合の満たすべき集計ルールとなる.

また,集計による各確定的効用項の範囲については次のとおりになる:
\begin{align}
\min_{l \in{L}} \gamma_{lm} \hat V_{l} \leq \gamma_{Lm} \hat V_L \leq \max_{l \in{L}} \gamma_{lm} \hat V_l,
\end{align}
これらは,データのばらつきを考慮した最適化問題や確率モデルにおいて,変数が取りうる範囲を制限するために使用される.
\clearpage