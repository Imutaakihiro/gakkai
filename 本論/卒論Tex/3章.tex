\chapter{データと手法}
\section{データセット}
本研究では,福岡工業大学の授業評価システムにおける2018年度から2023年度までのデータを使用した.対象は9学科,授業数は3,268件である.自由記述の総件数は83,851件であり,各授業に対して複数の自由記述が付随する.

データセットの概要を表\ref{tab:dataset}に示す.
\begin{table}[t]
    \centering
    \caption{データセット概要}
    \label{tab:dataset}
    \resizebox{0.75\textwidth}{!}{
    \begin{tabular}{l r}
        \toprule
        項目 & 値 \\
        \midrule
        対象期間 & 2018年度〜2023年度 \\
        対象学科数 & 9 \\
        授業数 & 3,268 \\
        自由記述総件数 & 83,851 \\
        平均自由記述数/授業 & 25.2 \\
        \bottomrule
    \end{tabular}
    }
\end{table}

\section{授業評価アンケートの構成}
授業評価アンケートは,(1) 択一式質問の点数化による授業評価スコア,(2) 自由記述の2種類の情報から構成される.自由記述は以下の2つの質問からなる.
\begin{enumerate}
\item 先生に向けてこの授業の感想や学んだこと,意見や要望を記述してください
\item 次期履修者に向けて,この授業についてのアドバイスを記述してください
\end{enumerate}
授業評価スコアは4段階(1〜4点)であり,授業ごとに単一のスコアが付与される.一方,自由記述は授業単位で複数件存在するため,授業単位で集約して分析する必要がある.

\section{前処理}
自由記述は,日本語形態素解析によりトークン化を行い,記号や不要な空白の除去などの正規化を施した.その後,モデル入力に適した形式へ変換し,学習データとして利用した.

\section{教師データ}
感情分類モデルの構築のため,ランダム抽出した1,000件の自由記述を手動でラベリングし,教師データとして用いた.ラベルはネガティブ,ポジティブ,ニュートラルの3クラスとし,分類結果はネガティブ191件,ポジティブ180件,ニュートラル628件であった.

教師データのラベル分布を表\ref{tab:labeldist-method}に示す.
\begin{table}[t]
    \centering
    \caption{教師データのラベル分布(1,000件)}
    \label{tab:labeldist-method}
    \resizebox{0.6\textwidth}{!}{
    \begin{tabular}{l r}
        \toprule
        ラベル & 件数 \\
        \midrule
        ネガティブ & 191 \\
        ニュートラル & 628 \\
        ポジティブ & 180 \\
        \bottomrule
    \end{tabular}
    }
\end{table}

\section{モデル構成}
\subsection{感情分類モデル}
感情分類には,日本語の事前学習済みBERTモデルを用いた.BERTの出力表現に分類器を接続し,3クラス分類として微調整を行った.これにより,自由記述ごとの感情スコア(ネガティブ=-1,ニュートラル=0,ポジティブ=+1)を推定した.

\subsection{マルチタスク学習モデル}
感情スコア予測と授業評価スコア予測を同時に学習するマルチタスクモデルを構築した.BERTエンコーダを共有し,感情分類ヘッドと評価スコア予測ヘッドを分岐させる構成とした.これにより,共通要因と特化要因の分離が可能となる.

\subsection{順序回帰モデル}
授業評価スコアは順序尺度であるため,順序回帰モデルを構築する.順序回帰モデルでは,評価スコア1〜4点の確率分布(P1,P2,P3,P4)を同時に推定し,評価段階ごとの特徴を分析する.本節の詳細は,追加実験の完了後に追記する.

\section{学習設定}
モデル学習は,教師データを訓練用・検証用に分割して実施した.学習率やバッチサイズは予備実験により調整し,過学習を防ぐために早期終了などの設定を用いた.

\section{授業単位集約と相関分析}
自由記述ごとの感情スコアを授業単位で平均し,授業単位の感情スコアを算出した.この感情スコアと授業評価スコアの関係性を検討するため,ピアソン相関係数,スピアマン順位相関係数,ケンドール順位相関係数を用いて相関分析を行った.

\section{SHAP分析}
モデルの解釈可能性を高めるため,SHAP分析を実施した.単一タスクモデルおよびマルチタスクモデルに対し,層化サンプリングで5,000件を抽出し,出現回数5回以上の単語を対象として寄与度を算出した.最終的に1,564語を対象に分析を行った.

\section{評価指標}
感情分類の性能評価には正解率およびF1スコアを用いた.授業評価スコア予測の性能評価には決定係数($R^2$)を用いた.相関分析では有意確率を併記し,統計的有意性を確認した.

\section{分析フロー}
分析全体の流れを図\ref{fig:flow}に示す.
\begin{figure}[t]
\centering
\fbox{\rule{0pt}{45mm}\rule{0.6\textwidth}{0pt}}
\caption{分析フローの概略}
\label{fig:flow}
\end{figure}
