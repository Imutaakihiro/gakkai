\chapter{データと手法}

本章では,本研究で使用したデータセットの概要,前処理,教師データ,モデル構成,学習設定,分析手順を述べる.

%%%%%%%%%%%%%%%%%%%%%%%%%%%%%%%%%%%%%%%%%%%%%%%%%%%%%%%%%%%%%%%%%%%%%%%%%%%%%%%
\section{データセット}
%%%%%%%%%%%%%%%%%%%%%%%%%%%%%%%%%%%%%%%%%%%%%%%%%%%%%%%%%%%%%%%%%%%%%%%%%%%%%%%

\subsection{データ概要}
本研究では,福岡工業大学の授業評価システムにおける2018年度から2024年度までの7年間のデータを使用した.対象は9学科で,授業数は3,268件,自由記述総件数は83,851件である.データセット概要を表\ref{tab:dataset}に示す.

\begin{table}[t]
    \centering
    \caption{データセット概要}
    \label{tab:dataset}
    \resizebox{0.75\textwidth}{!}{
    \begin{tabular}{l r}
        \toprule
        項目 & 値 \\
        \midrule
        対象期間 & 2018年度〜2024年度(7年間) \\
        対象学科数 & 9 \\
        授業数 & 3,268 \\
        自由記述総件数 & 83,851 \\
        平均自由記述数/授業 & 25.2 \\
        自由記述の平均文字数 & 約41文字 \\
        \bottomrule
    \end{tabular}
    }
\end{table}

\subsection{アンケート構成と尺度}
授業評価アンケートは,(1) 択一式質問の点数化による授業評価スコア,(2) 自由記述の2種類から構成される.自由記述は以下の2問である.

\begin{enumerate}
\item 先生に向けてこの授業の感想や学んだこと,意見や要望を記述してください
\item 次期履修者に向けて,この授業についてのアドバイスを記述してください
\end{enumerate}

授業評価スコアは4段階(1〜4点)であり,平均3.459点(標準偏差0.216)である.自由記述は授業単位で複数件存在するため,授業単位で集約して分析する必要がある.

%%%%%%%%%%%%%%%%%%%%%%%%%%%%%%%%%%%%%%%%%%%%%%%%%%%%%%%%%%%%%%%%%%%%%%%%%%%%%%%
\section{前処理と教師データ}
%%%%%%%%%%%%%%%%%%%%%%%%%%%%%%%%%%%%%%%%%%%%%%%%%%%%%%%%%%%%%%%%%%%%%%%%%%%%%%%

\subsection{テキスト前処理}
自由記述には以下の前処理を施した.

\begin{enumerate}
\item \textbf{Unicode正規化}: 全角・半角の統一と正規化
\item \textbf{記号除去}: 絵文字や特殊記号の除去
\item \textbf{形態素解析}: MeCabによる分かち書き
\item \textbf{最大長制限}: BERTの入力長(512トークン)に合わせて切り詰め
\end{enumerate}

\subsection{教師データの作成}
感情分類モデルの構築のため,自由記述からランダムに1,000件を抽出し,ネガティブ(−1),ニュートラル(0),ポジティブ(+1)の3クラスで手動ラベリングを行った.ラベル分布を表\ref{tab:labeldist-method}に示す.

\begin{table}[t]
    \centering
    \caption{教師データのラベル分布(1,000件)}
    \label{tab:labeldist-method}
    \resizebox{0.6\textwidth}{!}{
    \begin{tabular}{l r r}
        \toprule
        ラベル & 件数 & 割合 \\
        \midrule
        ネガティブ & 191 & 19.1\% \\
        ニュートラル & 628 & 62.8\% \\
        ポジティブ & 180 & 18.0\% \\
        \midrule
        合計 & 1,000 & 100.0\% \\
        \bottomrule
    \end{tabular}
    }
\end{table}

教師データは訓練用800件(80\%)と検証用200件(20\%)に層化分割した.

%%%%%%%%%%%%%%%%%%%%%%%%%%%%%%%%%%%%%%%%%%%%%%%%%%%%%%%%%%%%%%%%%%%%%%%%%%%%%%%
\section{モデル構成}
%%%%%%%%%%%%%%%%%%%%%%%%%%%%%%%%%%%%%%%%%%%%%%%%%%%%%%%%%%%%%%%%%%%%%%%%%%%%%%%

\subsection{BERTの概要}
BERTはTransformerに基づく事前学習済み言語モデルであり\cite{bert},双方向の文脈情報を同時に考慮できる点が特徴である\cite{transformer}.本研究では日本語の事前学習済みBERTを用いた.基本構成を表\ref{tab:bert_arch}に示す.

\begin{table}[t]
    \centering
    \caption{BERTモデルの基本構成}
    \label{tab:bert_arch}
    \resizebox{0.7\textwidth}{!}{
    \begin{tabular}{l r}
        \toprule
        項目 & 値 \\
        \midrule
        エンコーダ層数 & 12 \\
        隠れ層次元数 & 768 \\
        アテンションヘッド数 & 12 \\
        パラメータ数 & 約1.1億 \\
        最大入力トークン数 & 512 \\
        \bottomrule
    \end{tabular}
    }
\end{table}

\subsection{感情分類モデル}
感情分類モデルは,BERTエンコーダの[CLS]ベクトルに分類ヘッドを接続し,3クラスの確率分布を出力する構成とした.損失関数はクロスエントロピー損失であり,クラス不均衡に対応するため重み付けを適用した.

\subsection{マルチタスク学習モデル}
感情スコア予測と授業評価スコア予測を同時に学習するマルチタスク学習モデルを構築した\cite{mtl}.BERTエンコーダを共有し,感情分類ヘッドと評価スコア予測ヘッドを分岐させる構成とした.アーキテクチャを図\ref{fig:mtl_model}に示す.

\begin{figure}[t]
\centering
\begin{tabular}{c}
\hline
\textbf{マルチタスク学習モデルの構成} \\
\hline
入力テキスト \\
$\downarrow$ \\
BERTエンコーダ(共有) \\
$\downarrow$ \\
{[}CLS{]}トークンの出力ベクトル(768次元) \\
$\swarrow$ \hspace{2cm} $\searrow$ \\
感情分類ヘッド \hspace{1cm} 評価スコア予測ヘッド \\
$\downarrow$ \hspace{2.5cm} $\downarrow$ \\
3クラス確率 \hspace{1.5cm} 評価スコア(回帰値) \\
\hline
\end{tabular}
\caption{マルチタスク学習モデルのアーキテクチャ}
\label{fig:mtl_model}
\end{figure}

\subsection{順序回帰モデル}
授業評価スコアは順序尺度であるため,順序回帰モデルを検討した\cite{coral}.評価スコアが$k$以上となる累積確率から,評価段階の確率を算出する.

\begin{equation}
P(Y = k) = P(Y \geq k) - P(Y \geq k+1)
\label{eq:ordinal}
\end{equation}

%%%%%%%%%%%%%%%%%%%%%%%%%%%%%%%%%%%%%%%%%%%%%%%%%%%%%%%%%%%%%%%%%%%%%%%%%%%%%%%
\section{学習設定と分析手順}
%%%%%%%%%%%%%%%%%%%%%%%%%%%%%%%%%%%%%%%%%%%%%%%%%%%%%%%%%%%%%%%%%%%%%%%%%%%%%%%

\subsection{学習設定}
モデル学習のハイパーパラメータを表\ref{tab:hyperparams}に示す.最適化にはAdamWを使用し\cite{bert},早期終了を適用した.

\begin{table}[t]
    \centering
    \caption{学習のハイパーパラメータ}
    \label{tab:hyperparams}
    \resizebox{0.75\textwidth}{!}{
    \begin{tabular}{l r l}
        \toprule
        パラメータ & 値 & 選定理由 \\
        \midrule
        バッチサイズ & 16 & GPUメモリ制約を考慮 \\
        学習率 & $5 \times 10^{-6}$ & 微調整に適した低学習率 \\
        エポック数 & 5 & 早期終了で過学習を抑制 \\
        最大トークン長 & 512 & BERTの最大入力長 \\
        ドロップアウト率 & 0.1 & 過学習抑制の標準値 \\
        \bottomrule
    \end{tabular}
    }
\end{table}

\subsection{授業単位集約と相関分析}
感情分類モデルで推定した感情スコアを授業単位で平均し,授業評価スコアとの関係を分析した.授業$j$の感情スコア平均$\bar{S}_j$は次式で算出する.

\begin{equation}
\bar{S}_j = \frac{1}{n_j}\sum_{i=1}^{n_j} s_{ij}
\label{eq:agg}
\end{equation}

相関係数としてピアソン,スピアマン,ケンドールの3指標を算出した.

\subsection{SHAP分析}
モデル解釈のため,SHAP分析を実施した\cite{shap}.計算負荷を考慮し,層化サンプリングで5,000件を抽出した.分析設定を表\ref{tab:shap_setting}に示す.

\begin{table}[t]
    \centering
    \caption{SHAP分析の設定}
    \label{tab:shap_setting}
    \resizebox{0.7\textwidth}{!}{
    \begin{tabular}{l r}
        \toprule
        項目 & 値 \\
        \midrule
        分析サンプル数 & 5,000件 \\
        サンプリング手法 & 層化サンプリング \\
        最小出現回数閾値 & 5回 \\
        分析対象語彙数(単一タスク) & 1,564語 \\
        分析対象語彙数(マルチタスク) & 3,198語 \\
        \bottomrule
    \end{tabular}
    }
\end{table}

マルチタスクモデルのSHAP分析では,感情重要度と評価重要度の閾値(0.0005)に基づき,共通要因・感情特化要因・評価特化要因・低重要度要因の4群に分類した.

\subsection{評価指標}
感情分類モデルの評価にはAccuracy,Precision,Recall,F1を用い,クラス不均衡を考慮してマクロ平均と重み付き平均を併記した.授業評価スコア予測では$R^2$,RMSE,MAEを使用した.

%%%%%%%%%%%%%%%%%%%%%%%%%%%%%%%%%%%%%%%%%%%%%%%%%%%%%%%%%%%%%%%%%%%%%%%%%%%%%%%
\section{分析フローと実装環境}
%%%%%%%%%%%%%%%%%%%%%%%%%%%%%%%%%%%%%%%%%%%%%%%%%%%%%%%%%%%%%%%%%%%%%%%%%%%%%%%

\subsection{分析フロー}
本研究の分析フローを図\ref{fig:flow}に示す.

\begin{figure}[t]
\centering
\begin{tabular}{|l|}
\hline
\textbf{【データ収集】} \\
授業評価アンケート(3,268授業,83,851件自由記述) \\
\hline
$\downarrow$ \\
\hline
\textbf{【前処理】} \\
テキスト正規化,トークナイザによる分割 \\
\hline
$\downarrow$ \\
\hline
\textbf{【教師データ作成】} \\
1,000件の手動ラベリング(3クラス) \\
\hline
$\downarrow$ \\
\hline
\textbf{【モデル構築】} \\
感情分類モデル(BERT + 分類ヘッド) \\
マルチタスク学習モデル(BERT + 2ヘッド) \\
\hline
$\downarrow$ \\
\hline
\textbf{【感情スコア推定】} \\
全自由記述に対する感情スコア推定 \\
\hline
$\downarrow$ \\
\hline
\textbf{【授業単位集約】} \\
授業ごとの感情スコア平均を算出 \\
\hline
$\downarrow$ \\
\hline
\textbf{【相関分析】} \\
感情スコアと授業評価スコアの相関を検証 \\
\hline
$\downarrow$ \\
\hline
\textbf{【SHAP分析】} \\
5,000件サンプル(単一タスク: 1,564語) \\
マルチタスク: 3,198語,4グループ分類 \\
\hline
\end{tabular}
\caption{分析フローの概略}
\label{fig:flow}
\end{figure}

\subsection{実装環境}
本研究の実装環境を表\ref{tab:env}に示す.

\begin{table}[t]
    \centering
    \caption{実装環境}
    \label{tab:env}
    \resizebox{0.7\textwidth}{!}{
    \begin{tabular}{l l}
        \toprule
        項目 & 内容 \\
        \midrule
        プログラミング言語 & Python 3.10 \\
        深層学習フレームワーク & PyTorch 2.0 \\
        Transformersライブラリ & Hugging Face Transformers 4.30 \\
        SHAP分析ライブラリ & SHAP 0.42 \\
        統計分析ライブラリ & SciPy 1.11 \\
        \bottomrule
    \end{tabular}
    }
\end{table}

本章では,データセットの概要,前処理,モデル構成,学習設定,分析手順を説明した.次章では結果を報告する.
