\chapter{結論}
%----------------------------
\section{本研究のまとめ}
%----------------------------
本研究では,複雑な選択肢間の異質性を考慮できる Generalized Nested Logit (GNL) モデルについて
基本的な性質を明らかにし,マーケティング分野において適用,拡張,及び効率的利用のためのツールの開発を行なった.
本研究で明らかになった成果は次のとおりである.

2 章では,GNL モデルが持つ,集計的,非集計的な側面からの情報理論との対応付けを行なった.
その結果,次のことがわかった:
\begin{enumerate}
\item GNL モデルは集計的に,エントロピー制約付き社会的効用最大化問題として解釈可能,
\item 集計的なモデルにおいて,サンプル数を $1$ とすると個人レベルでの効用最大化問題として解釈可能,
\item GNL モデルの段階的最尤推定問題は,その各段階がそれぞれ共役なエントロピー最大化(情報量最小化)問題と等価,
\item 一般的に効用最大化と整合的であるためとされたパラメータの制約条件は,情報整合的な制約と,効用最大化と整合的であるための条件があることが判明.
\end{enumerate}
これらの事実により,新たなパラメータ推定の方法や,パラメータの制約条件の付加の仕方が提案され,情報理論的な GNL モデルの解釈が可能となった.

3 章では,従来ほとんど例がなかった GNL モデルのマーケティング分野への適用を行なった.
具体的な GNL モデルの構造を決定するネスティング・ルールとして,要素分割を提案し,商品カテゴリーとしてコーラに対し,適用,検証を行なった.
その結果,次のことがわかった:
\begin{enumerate}
\item MNL,NL,GNL の各モデルを比較し,各基本統計量において,GNL モデルが優位,
\item 集計した選択割合においても,GNL モデルが優位,
\item 上記のモデル間には,効用関数パラメータの大きな相違が見られる.
\end{enumerate}
4 章では,3 章で提案した GNL モデルの適用を,消費者に異質性について考慮できるように拡張した LGNL モデルを提案した.
そして 3 章と同様に商品カテゴリーとしてコーラに対し,LGNL モデルの適用,検証を行った.
\begin{enumerate}
\item MNL,LML,GNL,LGNL の各モデルを比較し,各基本統計量において,LGNL モデルが優位,
\item 集計した選択割合においても,LGNL モデルが優位.
\end{enumerate}
3,4 章の実証結果により,統計学的な側面から LGNL,GNL モデルの妥当性が示されたといえる.

5 章では,効用最大化と矛盾するとされる心理的効果について,GNL モデルを用いて表現することを試みた.
その結果,GNL モデルのある特定のネスティング構造を用いることで,次に示す結果を得た:
\begin{enumerate}
\item 弱妥協効果,強妥協効果については表現可能,
\item 弱妥協効果については,その表現可能な大きさは Simonson ans Tversky (1992) \cite{ST92} の結果と比較し,十分,
\item 弱魅力効果についても表現可能であるが,その大きさはわずか,
\item 効用最大化と矛盾するとされた現象が,ランダム効用最大化と整合的に生起.
\end{enumerate}
これらの結果は,GNL モデルの妥当性を心理学的に裏付けるものであるといえる.

6 章では,実務面での GNL モデルの適用を考え,集計問題を克服できる集計ルールを提案した.
その結果,次に示す結果を得た:
\begin{enumerate}
\item ネストを集計した場合,満たすべきルールとして三つのルールを導出,
\item 選択肢を集計した場合,満たすべきルールとして二つのルールを導出,
\item GNL モデルの集計ルールが,MNL,NL の各モデルのルールと整合的であることを検証,
\item 集計した場合の効用の変化について限定的ながら上限,下限を導出.
\end{enumerate}
これらの結果は,GNL モデルを実務で適用する際に有用であり,GNL モデルの工学的な性質を明らかにしたものであるといえる.

本研究では,経済学的なモデルである GNL モデルについて,その性質,妥当性を多方面から提示,実証した.
離散選択モデルについて画期的な研究を行ない,本研究でもその著作の多くを引用した McFadden のノーベル経済学賞受賞記念原稿(reprinted to \cite{McF01})では,
10 人の学者に対し謝辞を冒頭で贈っている,共同受賞者の James Heckman, それ以外に Zvi Griliches, L. L. Thurstone, Jacob Marschak, Duncan Luce, Amos Tversky, Danny Kahneman, Mashe Ben-Akiva, Charlles Manski, Kenneth Train である.
これらのうち,純粋な経済学者と目されるのは,Zvi Grilliches, Chalies Manski, Kenneth Train だけである.
James Heckman と Jacob Marschak は統計学者としての側面が大きい.
残りの L. L. Thurstone, Duncan Luce, Amos Tversky, Danny Kahneman は心理学者,Ben-Akiva は工学者である.
本研究では,これら全ての方面から,経済学的なモデルである GNL モデルについて検討したこととなる.

%----------------------------
\section{今後の展望}
%----------------------------
前節に示したように,本研究では,様々な方面から GNL モデルについてその性質や意味づけを研究した.
しかし,なお若干の理論的課題,技術的課題と,様々な場面での実証が必要である.

理論的課題として,確率的顕示選好の弱公理と強公理の明確な対応を示すことが挙げられる.
近年,Fosgerau et al. (2010) \cite{FMB10} が示しているように,GNL モデルの構造とある種のモデルのクラスとの関係性が明らかになりつつあり,
今後の発展が期待される.

技術的課題としては,GNL モデルのパラメータ推定方法の改良が挙げられる.
これは,GNL モデルの同時最尤推定を行なう際の対数尤度関数が凸とならず,安定的にパラメータ推定ができないこと \cite{ZFY07} を克服しようとするものである.
この問題に対し,メタヒューリスティクスを活用したパラメータ推定方法 \cite{RMK06,FJDB06,KM08} が GNL モデル以外の離散選択モデルで提案されている.
その多くが実数値 Genetic Algorithm (例えば Davis, 1991 \cite{Dav91}) を用いたものであるが,筆者は特に,複数の NL モデルの線形和的に構築される GNL モデルの構造を考えると,Particle Swarm Optimization (PSO) \cite{KE95} が有力な方法論の基本となると考えている.
また,推定対象のパラメータ数を減らし,凸性を減少させ,計算不可を低減する上では,パラメータをデモグラフィック要因や商品の構成要素に起因する要因で説明する,パラメータの構造化(例えば,Prashker and S. Bekhor, 1998 \cite{PB98}; Adachi et al., 2011 \cite{ATSO11}) が有力な手法となり得るだろう.

実証面での課題は,本研究では行われていない POS データからの心理的効果の実証が挙げられる.
また,現在,心理的効果を考えずに,ヒューリスティクスに行われているライン拡張 \cite{DK93} が行われた際のデータでの効果の検証も実務に適用する際には必要となるだろう. 
