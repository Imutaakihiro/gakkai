%-----------------------------
\chapter{おわりに}
%-----------------------------

本章では,本研究の成果を総括し,研究の限界と今後の課題を述べる.

%%%%%%%%%%%%%%%%%%%%%%%%%%%%%%%%%%%%%%%%%%%%%%%%%%%%%%%%%%%%%%%%%%%%%%%%%%%%%%%
\section{まとめ}
%%%%%%%%%%%%%%%%%%%%%%%%%%%%%%%%%%%%%%%%%%%%%%%%%%%%%%%%%%%%%%%%%%%%%%%%%%%%%%%

\subsection{研究目的と方法}
本研究は,授業評価アンケートの自由記述から感情スコアを推定し,授業評価スコアとの関係を分析することで,授業評価に影響する要因を定量的に特定することを目的とした.BERTによる感情分類とマルチタスク学習を組み合わせ,SHAP分析により語彙寄与度を可視化した.

\subsection{主な成果}
授業単位で集約した感情スコアと授業評価スコアの相関分析では,ピアソン0.3097,スピアマン0.2970,ケンドール0.2042の正の相関が得られた.感情分類モデルは検証データ200件で正解率0.770,マクロ平均F1スコア0.706を示した.マルチタスクモデルのSHAP分析により,3,198語を共通要因18.0\%,感情特化要因37.5\%,評価特化要因16.6\%,低重要度要因27.8\%に分類した.

%%%%%%%%%%%%%%%%%%%%%%%%%%%%%%%%%%%%%%%%%%%%%%%%%%%%%%%%%%%%%%%%%%%%%%%%%%%%%%%
\section{研究の限界}
%%%%%%%%%%%%%%%%%%%%%%%%%%%%%%%%%%%%%%%%%%%%%%%%%%%%%%%%%%%%%%%%%%%%%%%%%%%%%%%

\subsection{データと一般化}
本研究のデータは単一大学(2018年度〜2024年度)に限定されており,他大学や異なる教育環境への一般化可能性は検証されていない.また,本研究は相関・寄与度に基づく分析であり,因果関係の検証は行っていない.

\subsection{モデルと解釈}
教師データは1,000件であり,単一評価者によるラベリングとクラス不均衡が存在する.さらに,SHAP分析はサブワード単位であり,語彙の文脈的意味の解釈には限界がある.

研究の限界を表\ref{tab:limitations}に整理する.

\begin{table}[t]
    \centering
    \caption{研究の限界と対応の整理}
    \label{tab:limitations}
    \resizebox{0.85\textwidth}{!}{
    \begin{tabular}{l l l l}
        \toprule
        観点 & 内容 & 影響 & 対応の方向性 \\
        \midrule
        因果関係 & 相関・寄与度に基づく分析に限定 & 因果的主張は困難 & 介入研究・準実験デザイン \\
        一般化 & 単一大学・2018--2024年度に限定 & 外的妥当性が限定的 & 複数大学の比較分析 \\
        教師データ & 1,000件・不均衡 & 少数クラスの推定が不安定 & 追加ラベリング・複数評価者 \\
        解釈性 & サブワード単位の寄与度 & 文脈解釈の難しさ & フレーズ単位の再分析 \\
        \bottomrule
    \end{tabular}
    }
\end{table}

%%%%%%%%%%%%%%%%%%%%%%%%%%%%%%%%%%%%%%%%%%%%%%%%%%%%%%%%%%%%%%%%%%%%%%%%%%%%%%%
\section{今後の課題}
%%%%%%%%%%%%%%%%%%%%%%%%%%%%%%%%%%%%%%%%%%%%%%%%%%%%%%%%%%%%%%%%%%%%%%%%%%%%%%%

\subsection{検証研究の拡充}
共通要因に基づく改善施策が実際に評価向上に寄与するかを確認するため,介入研究や準実験デザインが必要である.また,複数大学の比較や学部・学科別の分析により,知見の一般性を検証する必要がある.

\subsection{実装・運用の発展}
教師データの拡充やドメイン適応により感情分類精度の向上が期待される.順序回帰モデルの本格導入によって評価段階ごとの要因分析を深化できる可能性がある.さらに,結果を教育現場で活用するため,分析結果を提示するダッシュボード等の設計が課題である.

%%%%%%%%%%%%%%%%%%%%%%%%%%%%%%%%%%%%%%%%%%%%%%%%%%%%%%%%%%%%%%%%%%%%%%%%%%%%%%%

本研究は,授業評価の自由記述を感情分析し,評価スコアとの関係を定量的に示した.得られた知見は,データに基づく教育改善の検討に資する基盤となる.
