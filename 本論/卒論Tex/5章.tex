%-----------------------------
\chapter{おわりに}
%-----------------------------
\section{結論}
本研究の結果として,$k$匿名化を適用することでプライバシー保護の強度が向上する一方で,データの一般化による情報損失がブランド選択モデルの精度に影響を及ぼすことが明らかとなった.具体的には,$k$の値が大きくなるにつれて,データ内の個人識別性が低下し,プライバシー保護の強度が増すが,選択確率の推定精度が低下する傾向が見られた.この影響は特に,高次元の特徴量を含むデータセットにおいて顕著であり,匿名化による情報の欠損がブランド選択の予測精度や識別性能の低下につながる可能性が示唆された.

また,GNLモデルにおけるパラメータ推定への影響を分析した結果,匿名化の適用によって一部の選択構造が変化し,ブランド間の相対的な選好関係に変動が生じる可能性が確認された.特に,選択肢間の相関構造を考慮するGNLモデルにおいては,データの一般化による影響が単なる精度低下にとどまらず,選択行動の解釈にも影響を与えることが示された.\\


%***************************************************************************************************************************
\section{今後の課題}
今後の課題として,まず他の匿名化手法との比較が挙げられる.本研究では$k$匿名化を中心に検討したが,$l$多様性や$t$近接性など他の手法と比較し,プライバシー保護とデータ有用性のトレードオフを評価することが重要である.

また,ネスト構造の妥当性についての検討が挙げられる.ブランド属性や選択肢の類似性を基にしたネスト設計がモデルの精度に与える影響を評価し,異なる基準でのネスト設計がモデルの適合性や予測精度をどう変化させるかを検討し,選択モデル全体の適合性向上の模索を今後の課題とする. 