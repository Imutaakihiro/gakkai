%-----------------------------
\chapter{おわりに}
%-----------------------------

本章では,本研究の成果を総括し,実践的示唆,研究の限界,および今後の課題について述べる.

%%%%%%%%%%%%%%%%%%%%%%%%%%%%%%%%%%%%%%%%%%%%%%%%%%%%%%%%%%%%%%%%%%%%%%%%%%%%%%%
\section{結論}
%%%%%%%%%%%%%%%%%%%%%%%%%%%%%%%%%%%%%%%%%%%%%%%%%%%%%%%%%%%%%%%%%%%%%%%%%%%%%%%

\subsection{研究目的の達成}
本研究は,授業評価アンケートの自由記述から感情スコアを推定し,授業評価スコアとの関係性を分析することで,授業評価に影響する要因を定量的に特定することを目的とした.この目的に対し,以下の成果を得た.

第一に,授業単位で集約した感情スコアと授業評価スコアの相関分析を行った結果,ピアソン相関係数0.3097($p<0.000001$)の中程度の正の相関が確認された.スピアマン順位相関係数(0.2970)およびケンドール順位相関係数(0.2042)においても同様に統計的に有意な正の相関が得られ,複数の指標で一貫した結果となった.これにより,学生の自由記述に表れる感情と授業評価スコアには一定の関係があることが示された.

第二に,BERTを基盤とした感情分類モデルを構築し,検証データ200件に対して正解率77\%,マクロ平均F1スコア0.71を達成した.この性能は,教育分野の自由記述に対する感情分析として実用的な水準であると考えられる.

第三に,感情スコアと授業評価スコアを同時に予測するマルチタスク学習モデルを構築し,SHAP分析により評価要因を定量化した.その結果,1,564語を4つの要因グループ(共通要因18.0\%,感情特化要因37.5\%,評価特化要因16.6\%,低重要度要因27.8\%)に分類することに成功した.

\subsection{仮説の検証}
本研究で設定した3つの仮説について,以下の結果が得られた.

\textbf{仮説1}「授業単位で集約した感情スコアと授業評価スコアには正の相関関係がある」については,相関分析により統計的に有意な正の相関が確認され,\textbf{支持された}.

\textbf{仮説2}「感情スコアと授業評価スコアの両方に影響する共通要因(満足度要因)が存在する」については,SHAP分析により577語(18.0\%)の共通要因が抽出され,「学ぶ」「理解」「総括」などの語彙が感情と評価の双方に寄与することが示された.したがって,本仮説は\textbf{支持された}.

\textbf{仮説3}「マルチタスク学習により,共通要因と特化要因を分離できる」については,マルチタスクモデルのSHAP分析により,共通要因・感情特化要因・評価特化要因・低重要度要因の4グループへの分離に成功した.したがって,本仮説は\textbf{支持された}.

\subsection{主要な発見}
本研究における主要な発見は以下の通りである.

\begin{enumerate}
\item \textbf{理解しやすさの重要性}: ポジティブ判定に最も寄与する語彙は「やす」(分かりやすい,理解しやすいの語幹)であり,授業内容の理解しやすさが学生満足度の中心的要因であることが示唆された.

\item \textbf{興味・関心の影響}: 「面白」「おもしろ」「楽しい」といった語彙が上位に位置しており,授業への興味・関心が感情評価に強く影響することが明らかになった.

\item \textbf{共通要因の存在}: 感情スコアと評価スコアの双方に寄与する共通要因(18.0\%)が存在し,これらへの対応が効率的な授業改善につながる可能性が示された.

\item \textbf{要因の分離可能性}: マルチタスク学習とSHAP分析の組み合わせにより,満足感に関わる要因と評価に関わる要因を定量的に分離できることが実証された.
\end{enumerate}

%%%%%%%%%%%%%%%%%%%%%%%%%%%%%%%%%%%%%%%%%%%%%%%%%%%%%%%%%%%%%%%%%%%%%%%%%%%%%%%
\section{実践的示唆}
%%%%%%%%%%%%%%%%%%%%%%%%%%%%%%%%%%%%%%%%%%%%%%%%%%%%%%%%%%%%%%%%%%%%%%%%%%%%%%%

\subsection{教育改善への応用}
本研究の結果は,教育改善において以下の示唆を提供する.

\textbf{共通要因への優先投資}: 共通要因(18.0\%)は,感情スコアと評価スコアの双方を同時に向上させる可能性がある.限られた資源で授業改善を行う際には,これらの共通要因に対応することで投資効率を高められると考えられる.具体的には,「学ぶ」「理解」「総括」といった語彙が示唆するように,学習成果の実感や授業内容の理解促進に注力することが有効であると考えられる.

\textbf{目的に応じた施策設計}: 感情特化要因(37.5\%)と評価特化要因(16.6\%)を区別することで,目的に応じた施策を設計できる.学生の満足感を高めたい場合は感情特化要因(授業の楽しさ,雰囲気など)に注力し,評価スコアの改善を優先する場合は評価特化要因(授業の有用性,学習成果など)に注力する戦略が考えられる.

\subsection{授業設計への具体的提案}
SHAP分析の結果から,以下の具体的な授業設計の方向性が示唆される.

\begin{enumerate}
\item \textbf{内容の明確化}: 「分かりやすさ」に関連する語彙の重要度が高いことから,授業内容の構造化や説明の明確化が満足度向上に寄与すると考えられる.

\item \textbf{興味喚起の工夫}: 「面白さ」「楽しさ」に関連する語彙が上位に位置することから,学生の興味・関心を引く教材設計や説明方法の工夫が有効であると考えられる.

\item \textbf{学習成果の可視化}: 共通要因に「学ぶ」「理解」「総括」が含まれることから,学習成果を学生自身が実感できる機会(小テスト,振り返りなど)を設けることが有効であると考えられる.
\end{enumerate}

\subsection{データ活用の可能性}
本研究で構築した感情分類モデルは,以下の場面での活用が期待される.

\begin{itemize}
\item \textbf{自動分類システム}: 大量の自由記述を自動的にポジティブ・ネガティブ・ニュートラルに分類し,教員へのフィードバックを効率化できる.
\item \textbf{早期警告システム}: 授業中や授業後の感想をリアルタイムで分析し,問題のある授業を早期に検知できる可能性がある.
\item \textbf{要因分析ダッシュボード}: SHAP分析の結果を可視化し,各授業の強み・弱みを定量的に把握できるツールの開発が考えられる.
\end{itemize}

%%%%%%%%%%%%%%%%%%%%%%%%%%%%%%%%%%%%%%%%%%%%%%%%%%%%%%%%%%%%%%%%%%%%%%%%%%%%%%%
\section{研究の限界}
%%%%%%%%%%%%%%%%%%%%%%%%%%%%%%%%%%%%%%%%%%%%%%%%%%%%%%%%%%%%%%%%%%%%%%%%%%%%%%%

本研究には以下の限界がある.

\subsection{因果関係の未検証}
本研究は相関関係の探索を目的としており,因果関係の検証は行っていない.例えば,「分かりやすい」という語彙がポジティブ判定に寄与することは示されたが,授業を「分かりやすく」すれば評価が向上するという因果的主張は本研究からは導けない.因果関係の検証には,介入研究や実験的デザインが必要である.

\subsection{データの限定性}
本研究のデータは福岡工業大学の1大学に限定されており,他大学への一般化可能性には限界がある.大学の規模,学部構成,学生層,教育文化などが異なる環境では,異なる結果が得られる可能性がある.

\subsection{教師データの制約}
教師データは1,000件と比較的少なく,ラベル付けは単一の評価者により行われたため,主観性が残る.また,クラス不均衡(ニュートラル62.8\%,ネガティブ19.1\%,ポジティブ18.0\%)が存在し,少数クラスの分類精度に影響を与えている可能性がある.

\subsection{モデルの解釈性}
BERTは高い性能を示す一方で,その予測根拠は必ずしも人間にとって直感的ではない.SHAP分析により解釈可能性を高めたものの,サブワード単位の分析では語彙の意味を完全に捉えられない場合がある.

\subsection{時間的変化の未考慮}
本研究では2018年度から2023年度までの6年間のデータを一括して分析したが,教育環境や学生の価値観は時間とともに変化している可能性がある.特に2020年以降のCOVID-19の影響によるオンライン授業の増加は,評価傾向に変化をもたらした可能性がある.

%%%%%%%%%%%%%%%%%%%%%%%%%%%%%%%%%%%%%%%%%%%%%%%%%%%%%%%%%%%%%%%%%%%%%%%%%%%%%%%
\section{今後の課題}
%%%%%%%%%%%%%%%%%%%%%%%%%%%%%%%%%%%%%%%%%%%%%%%%%%%%%%%%%%%%%%%%%%%%%%%%%%%%%%%

本研究の結果を踏まえ,以下の課題が今後の研究として挙げられる.

\subsection{因果関係の検証}
本研究で示唆された要因(理解しやすさ,面白さなど)が実際に評価向上に寄与するかを検証するため,介入研究が必要である.具体的には,特定の授業に対して共通要因に基づく改善を施し,その前後での評価変化を測定する準実験的デザインが考えられる.

\subsection{一般化可能性の検証}
複数大学のデータを用いた比較分析により,本研究の知見の一般性を検証する必要がある.また,学部・学科ごとの分析を行うことで,専門分野による評価要因の違いを明らかにすることも重要である.

\subsection{モデルの精度向上}
教師データの拡充や半教師あり学習の導入により,感情分類モデルの精度向上が期待される.また,ドメイン適応技術を用いて,教育分野に特化した言語モデルを構築することも有効であると考えられる.

\subsection{順序回帰モデルの発展}
授業評価スコアは順序尺度であるため,順序回帰モデルの導入により予測精度の向上が期待される.評価段階ごとの寄与要因を分析することで,「普通」から「良い」へ,「良い」から「非常に良い」への評価向上に寄与する要因を個別に特定できる可能性がある.

\subsection{実践への実装}
本研究の成果を教育現場で活用するため,感情分類の自動化システムや要因分析ダッシュボードの開発が課題である.教員が直感的に結果を理解し,授業改善に活用できるインターフェースの設計が重要となる.

\subsection{縦断的分析}
年度ごとの評価傾向の変化や,同一教員の授業における経年変化を分析することで,教育改善の効果測定や長期的なトレンドの把握が可能になると考えられる.

%%%%%%%%%%%%%%%%%%%%%%%%%%%%%%%%%%%%%%%%%%%%%%%%%%%%%%%%%%%%%%%%%%%%%%%%%%%%%%%
\section{研究の意義}
%%%%%%%%%%%%%%%%%%%%%%%%%%%%%%%%%%%%%%%%%%%%%%%%%%%%%%%%%%%%%%%%%%%%%%%%%%%%%%%

本研究は,以下の点で学術的・実践的意義を有する.

\subsection{学術的意義}
\begin{itemize}
\item \textbf{方法論の確立}: BERTを用いた感情分類とマルチタスク学習を組み合わせ,SHAP分析により要因を定量化する分析フレームワークを確立した.この方法論は,他の教育データ分析にも応用可能である.

\item \textbf{要因分離の実証}: マルチタスク学習により,共通要因と特化要因を定量的に分離できることを実証した.これは,複数の評価指標が存在する場面での要因分析に新たな視点を提供する.

\item \textbf{知見の蓄積}: 授業評価における感情要因の役割について,データに基づく知見を蓄積した.
\end{itemize}

\subsection{実践的意義}
\begin{itemize}
\item \textbf{改善の優先順位付け}: 共通要因・感情特化要因・評価特化要因の区別により,授業改善の優先順位を客観的に決定できる基盤を提供した.

\item \textbf{効率的な資源配分}: 共通要因への投資により,限られた資源で感情と評価の双方を向上させる戦略を提示した.

\item \textbf{データ駆動型教育改善}: 自由記述の自動分析により,大規模データに基づく教育改善の可能性を示した.
\end{itemize}

%%%%%%%%%%%%%%%%%%%%%%%%%%%%%%%%%%%%%%%%%%%%%%%%%%%%%%%%%%%%%%%%%%%%%%%%%%%%%%%

本研究では,授業評価アンケートの自由記述に対して感情分析を適用し,授業評価スコアとの関係性を分析した.その結果,感情スコアと授業評価スコアに統計的に有意な正の相関があること,理解しやすさや面白さが満足度に強く寄与すること,マルチタスク学習により要因を分離できることを明らかにした.これらの知見は,データに基づく教育改善の実現に向けた基盤を提供するものである.
