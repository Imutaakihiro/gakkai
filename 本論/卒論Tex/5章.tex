%-----------------------------
\chapter{おわりに}
%-----------------------------
\section{結論}
本研究では,授業評価アンケートの自由記述に対して感情分析を適用し,授業評価スコアとの関係性を分析した.授業単位で集約した感情スコアと授業評価スコアには統計的に有意な正の相関が確認され,学生の感情が授業評価と一定の関係を持つことが示された.

さらに,BERTを用いた感情分類モデルとマルチタスク学習モデルを構築し,SHAP分析により要因を定量化した.その結果,理解しやすさや面白さに関わる語彙がポジティブ判定に強く寄与することが明らかになった.また,共通要因として「学ぶ」「理解」「総括」「推奨」などが抽出され,感情スコアと授業評価スコアの双方を高める要因であることが示唆された.これらの結果は,授業改善における重点領域の把握に有用である.

\section{実践的示唆}
本研究で抽出された共通要因は,教育改善における優先度の高い領域を示す.授業内容の理解を促進する工夫や,興味・関心を高める設計が,満足度向上に寄与する可能性がある.また,評価特化要因や感情特化要因を切り分けることで,授業改善の施策をより精緻に設計できる.

\section{限界}
本研究は相関関係の探索を目的としており,因果関係の検証は行っていない.また,データは単一大学に限定されているため,他大学への一般化可能性には限界がある.教師データは1,000件と比較的少なく,ラベル付けの主観性も残る.

\section{今後の課題}
今後は,介入研究や実験的検証により因果的関係の解明が必要である.複数大学のデータを用いた比較や,学部・学科ごとの分析を行うことで一般性を高める必要がある.また,教師データの拡充や半教師あり学習の導入により感情分類の精度向上が期待される.さらに,感情特化要因・評価特化要因の活用方法を教育実践に落とし込むための検討が今後の課題である.順序回帰モデルの追加実験を通じて,評価段階ごとの要因差を明らかにすることも重要な課題である.
