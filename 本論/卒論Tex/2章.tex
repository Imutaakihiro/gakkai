\chapter{関連研究}

本章では,本研究に関連する先行研究を整理する.授業評価研究,自由記述分析と感情分析,BERT,マルチタスク学習,解釈可能AI,順序回帰について概観し,本研究の位置づけを明確にする.

%%%%%%%%%%%%%%%%%%%%%%%%%%%%%%%%%%%%%%%%%%%%%%%%%%%%%%%%%%%%%%%%%%%%%%%%%%%%%%%
\section{授業評価研究}
%%%%%%%%%%%%%%%%%%%%%%%%%%%%%%%%%%%%%%%%%%%%%%%%%%%%%%%%%%%%%%%%%%%%%%%%%%%%%%%

\subsection{授業評価の意義と活用}
授業評価(SET)は教育の質保証と改善に活用される指標であり,大学教育で広く実施されている\cite{marsh2007,spooren2013}.評価スコアは定量比較に適する一方,評価理由の把握には自由記述が重要となる.

\subsection{信頼性・妥当性}
授業評価は多面的な構造を持つことが示されており,信頼性・妥当性に関する議論が蓄積されている\cite{marsh2007,spooren2013}.一方で,評価が授業条件や学生側の要因に影響される可能性も指摘されているため,評価結果の解釈には注意が必要である.

%%%%%%%%%%%%%%%%%%%%%%%%%%%%%%%%%%%%%%%%%%%%%%%%%%%%%%%%%%%%%%%%%%%%%%%%%%%%%%%
\section{自由記述分析と感情分析}
%%%%%%%%%%%%%%%%%%%%%%%%%%%%%%%%%%%%%%%%%%%%%%%%%%%%%%%%%%%%%%%%%%%%%%%%%%%%%%%

\subsection{自由記述の特性}
自由記述は非構造テキストであり,人手による読解は大規模データでは困難である.教育分野では学習分析・教育データマイニングの文脈でテキスト分析が行われている\cite{romero2020}.

\subsection{感情分析の基礎と教育応用}
感情分析はテキストの感情極性を推定する技術であり,レビュー分析などで広く利用されている\cite{liu2012}.教育分野でも,学習者の自由記述から満足度や不満の傾向を把握する手段として位置づけられる\cite{romero2020}.

%%%%%%%%%%%%%%%%%%%%%%%%%%%%%%%%%%%%%%%%%%%%%%%%%%%%%%%%%%%%%%%%%%%%%%%%%%%%%%%
\section{感情分析手法の発展}
%%%%%%%%%%%%%%%%%%%%%%%%%%%%%%%%%%%%%%%%%%%%%%%%%%%%%%%%%%%%%%%%%%%%%%%%%%%%%%%

\subsection{辞書・特徴量ベースの手法}
辞書ベースや特徴量設計に依存する手法は実装が容易で解釈しやすいが,文脈依存表現への対応に限界がある\cite{liu2012}.

\subsection{深層学習と事前学習モデル}
深層学習は文脈情報を自動的に捉える利点を持つ.Transformerに基づく事前学習モデルの登場により,少量の教師データでも高精度な分類が可能になった\cite{transformer,bert}.

%%%%%%%%%%%%%%%%%%%%%%%%%%%%%%%%%%%%%%%%%%%%%%%%%%%%%%%%%%%%%%%%%%%%%%%%%%%%%%%
\section{BERTと事前学習言語モデル}
%%%%%%%%%%%%%%%%%%%%%%%%%%%%%%%%%%%%%%%%%%%%%%%%%%%%%%%%%%%%%%%%%%%%%%%%%%%%%%%

\subsection{BERTの特徴}
BERTは双方向の文脈情報を同時に考慮できる言語モデルであり,Masked Language Model等の事前学習により汎用表現を獲得する\cite{bert}.

\subsection{日本語モデルとドメイン適応}
日本語テキストに対しても事前学習済みモデルが利用可能であり,教育分野の自由記述に対してはドメイン適応を意識した微調整が求められる.

%%%%%%%%%%%%%%%%%%%%%%%%%%%%%%%%%%%%%%%%%%%%%%%%%%%%%%%%%%%%%%%%%%%%%%%%%%%%%%%
\section{マルチタスク学習}
%%%%%%%%%%%%%%%%%%%%%%%%%%%%%%%%%%%%%%%%%%%%%%%%%%%%%%%%%%%%%%%%%%%%%%%%%%%%%%%

\subsection{原理と利点}
マルチタスク学習は複数タスクを同時に学習し,共通表現を獲得することで性能向上と正則化効果を得る手法である\cite{mtl}.

\subsection{本研究への適用}
感情スコアと授業評価スコアは自由記述に基づく関連タスクであるため,マルチタスク学習により共通要因と特化要因を分離できる可能性がある\cite{mtl}.

%%%%%%%%%%%%%%%%%%%%%%%%%%%%%%%%%%%%%%%%%%%%%%%%%%%%%%%%%%%%%%%%%%%%%%%%%%%%%%%
\section{解釈可能AIとSHAP}
%%%%%%%%%%%%%%%%%%%%%%%%%%%%%%%%%%%%%%%%%%%%%%%%%%%%%%%%%%%%%%%%%%%%%%%%%%%%%%%

\subsection{XAIの背景}
モデルの予測根拠を説明するために解釈可能AI(XAI)が重視されており,教育分野でも説明可能性は重要である\cite{arrieta2020}.

\subsection{SHAPの特徴}
SHAPはShapley値に基づいて特徴量の寄与度を算出する手法であり,局所・大域の両方の説明が可能である\cite{shap}.テキスト分類では語彙レベルの寄与度を提示できる.

%%%%%%%%%%%%%%%%%%%%%%%%%%%%%%%%%%%%%%%%%%%%%%%%%%%%%%%%%%%%%%%%%%%%%%%%%%%%%%%
\section{順序回帰}
%%%%%%%%%%%%%%%%%%%%%%%%%%%%%%%%%%%%%%%%%%%%%%%%%%%%%%%%%%%%%%%%%%%%%%%%%%%%%%%

\subsection{順序尺度への適用}
授業評価スコアは順序尺度であり,順序性を考慮した回帰手法が求められる\cite{coral}.

\subsection{ニューラルネットワークとの統合}
ニューラルネットワークと順序回帰を統合する手法により,順序一貫性を保った予測が可能となる\cite{coral}.

%%%%%%%%%%%%%%%%%%%%%%%%%%%%%%%%%%%%%%%%%%%%%%%%%%%%%%%%%%%%%%%%%%%%%%%%%%%%%%%
\section{教育分野での分析枠組み}
%%%%%%%%%%%%%%%%%%%%%%%%%%%%%%%%%%%%%%%%%%%%%%%%%%%%%%%%%%%%%%%%%%%%%%%%%%%%%%%

\subsection{学習分析・教育データマイニング}
教育分野では学習ログやアンケートを用いた分析が進展しており,テキスト分析もその重要な要素である\cite{romero2020}.

\subsection{統合分析の課題}
評価スコアと自由記述を統合的に扱った研究は多くなく,両者を同時にモデル化する枠組みの整備が課題である.

%%%%%%%%%%%%%%%%%%%%%%%%%%%%%%%%%%%%%%%%%%%%%%%%%%%%%%%%%%%%%%%%%%%%%%%%%%%%%%%
\section{本研究の位置づけ}
%%%%%%%%%%%%%%%%%%%%%%%%%%%%%%%%%%%%%%%%%%%%%%%%%%%%%%%%%%%%%%%%%%%%%%%%%%%%%%%

本研究は,BERTによる感情分類とマルチタスク学習を組み合わせ,SHAP分析によって要因を定量化する点に特徴がある.既存研究との位置づけを表\ref{tab:comparison}に示す.

\begin{table}[t]
    \centering
    \caption{関連研究の焦点と本研究の位置づけ}
    \label{tab:comparison}
    \resizebox{0.85\textwidth}{!}{
    \begin{tabular}{l c c c c}
        \toprule
        研究タイプ & 評価スコア分析 & 自由記述分析 & 統合分析 & 要因の定量化 \\
        \midrule
        評価スコア中心研究 & ○ & − & − & △ \\
        自由記述中心研究 & − & ○ & − & △ \\
        統合分析研究 & ○ & ○ & △ & △ \\
        \textbf{本研究} & \textbf{○} & \textbf{○} & \textbf{○} & \textbf{○} \\
        \bottomrule
    \end{tabular}
    }
\end{table}

本研究の新規性は,(1) 感情と評価を同時に学習する枠組みの導入,(2) SHAPによる語彙寄与度の定量化,(3) 3,268授業・83,851件の自由記述に基づく大規模分析,の3点にある.

