\chapter{関連研究}
\section{授業評価研究}
大学における授業評価は,教育の質向上に向けた重要な指標として用いられている.多くの大学では,学期末のアンケートにより授業評価スコアと自由記述を収集し,教員へフィードバックを行っている.授業評価スコアは数量的に扱いやすい一方で,学生が評価に至った理由や具体的な改善要望は自由記述に含まれることが多い.このため,自由記述を定量的に分析し,授業評価スコアの背後にある要因を明らかにする研究が必要とされている\\cite{hujala2020,santhanam2018}.
さらに,授業改善への活用を前提とした分析プロセスの標準化や,フィードバックの迅速化が課題として指摘されており,効率的な分析手法の整備が求められる\\cite{hujala2020}.

\section{自由記述分析と感情分析}
自由記述は非構造テキストであり,従来は人的な読解に依存していた.しかし,近年の自然言語処理技術の発展により,大規模な自由記述を自動的に解析し,感情や評価の傾向を抽出することが可能になっている.感情分析は,テキストに含まれる肯定的・否定的・中立的な感情を推定する技術であり,教育分野においても学生の満足度や不満の把握に活用できる\\cite{rajput2016,sindhu2019}.
教育分野の自由記述分析では,テキスト分析を通じた改善提案の抽出や意見整理が報告されており\\cite{gottipati2018,misuraca2021},従来のスコア中心の評価を補完する手段としての重要性が高まっている.

\section{自由記述分析の手法分類}
自由記述分析は大きく,(1) 辞書型手法,(2) 古典的機械学習手法,(3) 深層学習手法に分類できる.
辞書型手法は,極性語彙を用いて感情を推定するため実装が容易であるが,文脈依存の表現や否定表現に弱い\\cite{rajput2016}.
古典的機械学習手法は,特徴量設計により一定の精度を得られるが,語彙の多様性が大きい自由記述では特徴量設計の負担が大きい\\cite{santhanam2018}.
深層学習手法は文脈を考慮できる一方で,教師データの準備コストが高く,教育分野固有の語彙や表現への適応が課題となる\\cite{bert}.

\section{BERTと日本語感情分析}
BERT(Bidirectional Encoder Representations from Transformers)は,Transformerを基盤とした事前学習済み言語モデルである\\cite{transformer}.双方向の文脈情報を利用できる点が特長であり,少量の教師データでも高精度な分類が可能であると報告されている\\cite{bert}.日本語に対しても事前学習済みモデルが提供されており,教育分野の自由記述に対して微調整を行うことで,文脈を考慮した感情分類が実現できる.
一方で,学習データの領域差が大きい場合には汎化性能が低下する可能性があるため,教育分野に特化した微調整と評価設計が必要となる.

\section{マルチタスク学習}
マルチタスク学習は,複数の関連するタスクを同時に学習し,共通表現を獲得することで各タスクの性能を向上させる手法である\\cite{mtl}.感情分析と授業評価スコア予測は,いずれも自由記述に基づく評価理解という共通の目的を持つため,マルチタスク学習の適用が有効であると考えられる.
特に,感情スコアと評価スコアを同時に学習することで,共通要因と特化要因の分離が可能となり,教育改善の示唆をより具体化できる点が期待される.

\section{解釈可能AIとSHAP}
機械学習モデルの予測根拠を明確化するため,解釈可能AI(XAI)が注目されている.SHAP(SHapley Additive exPlanations)は,協力ゲーム理論に基づき,特徴量の寄与度を定量化する手法である\\cite{shap}.テキスト分類においては,単語レベルの寄与度を算出できるため,授業評価に影響する要因を具体的な語彙として提示できる.
教育分野では,モデルの予測精度だけでなく,説明可能性が重要であり,改善施策への翻訳可能性が求められる.SHAPはこの点で有用であるが,寄与度の解釈は文脈に依存するため,定性的な検討との併用が必要となる\\cite{shap}.

\section{順序回帰}
授業評価スコアは1点から4点までの順序尺度であり,単純な回帰や分類では順序関係を適切に扱えない.順序回帰は,評価段階の順序性を考慮して確率分布を推定する手法であり,近年はニューラルネットワークと組み合わせた手法も提案されている\\cite{coral}.授業評価スコアの分析においても,順序回帰の導入は妥当である.

\section{教育分野への応用研究の整理}
教育分野では,学習ログやアンケートデータを用いた分析が進んでいる.自由記述を対象としたテキスト分析や意見抽出の取り組みは報告されているが\\cite{gottipati2018,misuraca2021,hujala2020},評価スコアとの関係性を統合的に扱った研究は多くない.また,教育改善に直結する語彙や要因を整理するために,語彙辞書の拡張や分析手法の整理が試みられている\\cite{santhanam2018}.このため,感情分析・マルチタスク学習・解釈可能AIを組み合わせた総合的な分析枠組みの構築が求められている.

\section{既存研究の限界と課題}
既存研究は,(1) 評価スコアと自由記述の統合が不十分,(2) 感情分析結果の解釈が定性的で,教育改善に直結しづらい,(3) 予測精度と説明可能性の両立が難しい,という課題を持つ.
特に,評価スコアと感情情報の関係を同時にモデル化した研究は限られており,改善施策を定量的に導出するための枠組みが不足している\\cite{gottipati2018,misuraca2021}.

\section{本研究の位置づけ}
本研究は,授業評価アンケートの自由記述に対し,BERTによる感情分類とマルチタスク学習を適用し,さらにSHAP分析によって評価要因を定量化する点に特徴がある.特に,感情スコアと授業評価スコアを同時に学習することで,共通要因と特化要因を分離し,改善施策の優先順位付けに利用できることを目指す.これにより,教育改善に資する具体的な知見を提供することを目指す.
