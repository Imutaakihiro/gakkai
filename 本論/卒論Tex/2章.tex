\chapter{関連研究}

本章では,本研究に関連する先行研究を整理する.授業評価研究,感情分析,BERT,マルチタスク学習,解釈可能AI,順序回帰について概観し,本研究の位置づけを明確にする.

%%%%%%%%%%%%%%%%%%%%%%%%%%%%%%%%%%%%%%%%%%%%%%%%%%%%%%%%%%%%%%%%%%%%%%%%%%%%%%%
\section{授業評価研究}
%%%%%%%%%%%%%%%%%%%%%%%%%%%%%%%%%%%%%%%%%%%%%%%%%%%%%%%%%%%%%%%%%%%%%%%%%%%%%%%

\subsection{授業評価の意義と構成}
大学における授業評価(Student Evaluation of Teaching: SET)は,教育の質向上に向けた重要な指標として広く用いられている\cite{marsh2007}.授業評価は1920年代にアメリカの大学で始まり,1970年代以降に世界的に普及した\cite{spooren2013}.日本においても,1990年代後半から多くの大学で導入が進み,現在ではほぼすべての大学で実施されている.

授業評価の主な目的は,(1) 教員へのフィードバックによる授業改善,(2) 人事評価の参考資料,(3) 学生への授業選択情報の提供,の3点である\cite{spooren2013}.

多くの大学では,学期末のアンケートにより授業評価スコアと自由記述を収集し,教員へフィードバックを行っている.授業評価スコアは数量的に扱いやすい一方で,学生が評価に至った理由や具体的な改善要望は自由記述に含まれることが多い\cite{hujala2020,santhanam2018}.

自由記述を定量的に分析し,授業評価スコアの背後にある要因を明らかにする研究が必要とされている.さらに,授業改善への活用を前提とした分析プロセスの標準化や,フィードバックの迅速化が課題として指摘されており,効率的な分析手法の整備が求められる\cite{hujala2020}.

\subsection{授業評価の信頼性と妥当性}
授業評価の信頼性・妥当性については多くの議論がある.Marshは,授業評価が多次元的な構造を持ち,異なる側面(例:明確さ,組織性,熱意)を測定していることを示した\cite{marsh2007}.一方で,評価が成績期待や授業難易度に影響される可能性も指摘されている\cite{spooren2013}.

%%%%%%%%%%%%%%%%%%%%%%%%%%%%%%%%%%%%%%%%%%%%%%%%%%%%%%%%%%%%%%%%%%%%%%%%%%%%%%%
\section{自由記述分析と感情分析}
%%%%%%%%%%%%%%%%%%%%%%%%%%%%%%%%%%%%%%%%%%%%%%%%%%%%%%%%%%%%%%%%%%%%%%%%%%%%%%%

\subsection{自由記述の特性と感情分析の基礎}
自由記述は非構造テキストであり,従来は人的な読解に依存していた.しかし,大規模なデータでは人的読解に限界があり,分析者の主観による解釈のばらつきも問題となる.近年の自然言語処理技術の発展により,大規模な自由記述を自動的に解析し,感情や評価の傾向を抽出することが可能になっている\cite{liu2012}.

感情分析(Sentiment Analysis)は,テキストに含まれる肯定的・否定的・中立的な感情を推定する技術である\cite{liu2012}.感情分析は,ソーシャルメディアの分析,製品レビューの分析,顧客満足度調査など,様々な分野で活用されている.教育分野においても,学生の満足度や不満の把握に活用できる\cite{rajput2016,sindhu2019}.Sinduらは,学生のフィードバックからアスペクトベースの意見マイニングを行い,教員の教育パフォーマンス評価に活用した\cite{sindhu2019}.

教育分野の自由記述分析では,テキスト分析を通じた改善提案の抽出や意見整理が報告されている\cite{gottipati2018,misuraca2021}.Gottipatiらは,学生のフィードバックから授業改善の提案を自動抽出するテキスト分析手法を提案した\cite{gottipati2018}.Misuracaらは,意見マイニングを教育分析に応用し,学生フィードバックの統合的分析戦略を提案した\cite{misuraca2021}.

これらの研究は,従来のスコア中心の評価を補完する手段としての自由記述分析の重要性を示している.

%%%%%%%%%%%%%%%%%%%%%%%%%%%%%%%%%%%%%%%%%%%%%%%%%%%%%%%%%%%%%%%%%%%%%%%%%%%%%%%
\section{感情分析の手法分類}
%%%%%%%%%%%%%%%%%%%%%%%%%%%%%%%%%%%%%%%%%%%%%%%%%%%%%%%%%%%%%%%%%%%%%%%%%%%%%%%

\subsection{感情分析手法の比較}
感情分析の手法は大きく,(1) 辞書型手法,(2) 古典的機械学習手法,(3) 深層学習手法に分類できる.

辞書型手法は,極性語彙(ポジティブ・ネガティブな単語のリスト)を用いて感情を推定する手法である.実装が容易で解釈しやすい利点がある一方,文脈依存の表現や否定表現に弱いという欠点がある\cite{rajput2016}.Rajputらは,教員評価における辞書ベースの感情分析を行い,その有効性と限界を報告した\cite{rajput2016}.

古典的機械学習手法(SVM,ナイーブベイズ,ランダムフォレストなど)は,特徴量設計により一定の精度を得られる.しかし,語彙の多様性が大きい自由記述では特徴量設計の負担が大きい\cite{santhanam2018}.Santhanamらは,学生フィードバックのテキスト分析において,共通語彙の適応と拡張を行った\cite{santhanam2018}.

深層学習手法(RNN,LSTM,Transformerなど)は,文脈を自動的に考慮できる利点がある.一方で,教師データの準備コストが高く,教育分野固有の語彙や表現への適応が課題となる\cite{bert}.近年は事前学習済みモデルの活用により,少量の教師データでも高精度な分類が可能になっている.

%%%%%%%%%%%%%%%%%%%%%%%%%%%%%%%%%%%%%%%%%%%%%%%%%%%%%%%%%%%%%%%%%%%%%%%%%%%%%%%
\section{BERTと事前学習済み言語モデル}
%%%%%%%%%%%%%%%%%%%%%%%%%%%%%%%%%%%%%%%%%%%%%%%%%%%%%%%%%%%%%%%%%%%%%%%%%%%%%%%

\subsection{BERTと日本語事前学習モデル}
BERT(Bidirectional Encoder Representations from Transformers)は,Transformerのエンコーダ部分を用いた事前学習済み言語モデルである\cite{bert}.Vaswaniらが提案したTransformerアーキテクチャ\cite{transformer}を基盤とし,双方向の文脈情報を同時に考慮できる点が特長である.

BERTは,Masked Language Model(MLM)タスクとNext Sentence Prediction(NSP)タスクにより事前学習される.大規模コーパスで事前学習されたモデルを特定タスクに微調整することで,少量の教師データでも高精度な分類が可能である\cite{bert}.

日本語に対しても事前学習済みBERTモデルが複数提供されている.東北大学が公開した「cl-tohoku/bert-base-japanese」は,日本語Wikipediaで事前学習されたモデルであり,日本語NLPタスクで広く利用されている\cite{cl-tohoku}.

教育分野の自由記述に対して微調整を行うことで,文脈を考慮した感情分類が実現できる.一方で,学習データの領域差が大きい場合には汎化性能が低下する可能性があるため,教育分野に特化した微調整と評価設計が必要となる.

%%%%%%%%%%%%%%%%%%%%%%%%%%%%%%%%%%%%%%%%%%%%%%%%%%%%%%%%%%%%%%%%%%%%%%%%%%%%%%%
\section{マルチタスク学習}
%%%%%%%%%%%%%%%%%%%%%%%%%%%%%%%%%%%%%%%%%%%%%%%%%%%%%%%%%%%%%%%%%%%%%%%%%%%%%%%

\subsection{マルチタスク学習の原理と適用}
マルチタスク学習は,複数の関連するタスクを同時に学習し,共通表現を獲得することで各タスクの性能を向上させる手法である\cite{mtl}.Zhangらは,マルチタスク学習の包括的なサーベイを行い,ハードパラメータ共有とソフトパラメータ共有の2つのアプローチを整理した\cite{mtl}.

マルチタスク学習の利点として,(1) 関連タスクからの情報転移による性能向上,(2) 過学習の抑制(正則化効果),(3) 共通表現の学習による解釈可能性の向上,が挙げられる.

自然言語処理分野では,感情分析と関連タスク(アスペクト抽出,意見ターゲット識別など)を同時に学習するマルチタスクモデルが提案されている\cite{ruder2019}.これらの研究は,関連タスクの同時学習が個別タスクの性能を向上させることを示している.

感情分析と授業評価スコア予測は,いずれも自由記述に基づく評価理解という共通の目的を持つため,マルチタスク学習の適用が有効であると考えられる.特に,感情スコアと評価スコアを同時に学習することで,共通要因と特化要因の分離が可能となり,教育改善の示唆をより具体化できる点が期待される.

%%%%%%%%%%%%%%%%%%%%%%%%%%%%%%%%%%%%%%%%%%%%%%%%%%%%%%%%%%%%%%%%%%%%%%%%%%%%%%%
\section{解釈可能AIとSHAP}
%%%%%%%%%%%%%%%%%%%%%%%%%%%%%%%%%%%%%%%%%%%%%%%%%%%%%%%%%%%%%%%%%%%%%%%%%%%%%%%

\subsection{解釈可能AIとSHAPの原理}
機械学習モデルの予測根拠を明確化するため,解釈可能AI(Explainable AI: XAI)が注目されている\cite{arrieta2020}.特に教育分野では,モデルの予測精度だけでなく,説明可能性が重要であり,改善施策への翻訳可能性が求められる.

SHAP(SHapley Additive exPlanations)は,協力ゲーム理論のShapley値に基づき,特徴量の寄与度を定量化する手法である\cite{shap}.SHAPは,(1) 局所的な説明と大域的な説明の両方が可能,(2) 理論的に一貫した寄与度を算出,(3) 様々なモデルに適用可能,という利点を持つ\cite{shap}.テキスト分類においては,単語レベルの寄与度を算出できるため,授業評価に影響する要因を具体的な語彙として提示できる.ただし,寄与度の解釈は文脈に依存するため,定性的な検討との併用が必要となる.

%%%%%%%%%%%%%%%%%%%%%%%%%%%%%%%%%%%%%%%%%%%%%%%%%%%%%%%%%%%%%%%%%%%%%%%%%%%%%%%
\section{順序回帰}
%%%%%%%%%%%%%%%%%%%%%%%%%%%%%%%%%%%%%%%%%%%%%%%%%%%%%%%%%%%%%%%%%%%%%%%%%%%%%%%

\subsection{順序回帰の特性とニューラルネットワークとの統合}
授業評価スコアは1点から4点までの順序尺度であり,単純な回帰(連続値として扱う)や分類(カテゴリとして扱う)では順序関係を適切に扱えない.順序回帰は,評価段階の順序性を考慮して確率分布を推定する手法である\cite{coral}.

近年は,ニューラルネットワークと順序回帰を組み合わせた手法が提案されている.Caoらは,CORAL(Consistent Rank Logits)を提案し,順序一貫性を保証しながらニューラルネットワークで順序回帰を行う手法を示した\cite{coral}.授業評価スコアの分析においても,順序回帰の導入は妥当である.

%%%%%%%%%%%%%%%%%%%%%%%%%%%%%%%%%%%%%%%%%%%%%%%%%%%%%%%%%%%%%%%%%%%%%%%%%%%%%%%
\section{教育分野への応用研究}
%%%%%%%%%%%%%%%%%%%%%%%%%%%%%%%%%%%%%%%%%%%%%%%%%%%%%%%%%%%%%%%%%%%%%%%%%%%%%%%

教育分野では,学習ログやアンケートデータを用いた分析(Learning Analytics, Educational Data Mining)が進んでいる\cite{romero2020}.自由記述を対象としたテキスト分析や意見抽出の取り組みは報告されているが\cite{gottipati2018,misuraca2021,hujala2020},評価スコアとの関係性を統合的に扱った研究は多くない.

また,教育改善に直結する語彙や要因を整理するために,語彙辞書の拡張や分析手法の整理が試みられている\cite{santhanam2018}.このため,感情分析・マルチタスク学習・解釈可能AIを組み合わせた総合的な分析枠組みの構築が求められている.

%%%%%%%%%%%%%%%%%%%%%%%%%%%%%%%%%%%%%%%%%%%%%%%%%%%%%%%%%%%%%%%%%%%%%%%%%%%%%%%
\section{既存研究の限界と課題}
%%%%%%%%%%%%%%%%%%%%%%%%%%%%%%%%%%%%%%%%%%%%%%%%%%%%%%%%%%%%%%%%%%%%%%%%%%%%%%%

既存研究には以下の課題がある.

第一に,評価スコアと自由記述の統合が不十分である.多くの研究は評価スコアの分析または自由記述の分析を別々に行っており,両者の関係を同時にモデル化した研究は限られている\cite{gottipati2018,misuraca2021}.

第二に,感情分析結果の解釈が定性的で,教育改善に直結しづらい.感情をポジティブ・ネガティブに分類するだけでは,具体的な改善策の導出が困難である.

第三に,予測精度と説明可能性の両立が難しい.高精度な深層学習モデルはブラックボックス化しやすく,教育改善への翻訳が困難である.

%%%%%%%%%%%%%%%%%%%%%%%%%%%%%%%%%%%%%%%%%%%%%%%%%%%%%%%%%%%%%%%%%%%%%%%%%%%%%%%
\section{本研究の位置づけ}
%%%%%%%%%%%%%%%%%%%%%%%%%%%%%%%%%%%%%%%%%%%%%%%%%%%%%%%%%%%%%%%%%%%%%%%%%%%%%%%

本研究は,授業評価アンケートの自由記述に対し,BERTによる感情分類とマルチタスク学習を適用し,さらにSHAP分析によって評価要因を定量化する点に特徴がある.

既存研究との差異を表\ref{tab:comparison}に示す.

\begin{table}[t]
    \centering
    \caption{既存研究との比較}
    \label{tab:comparison}
    \resizebox{0.9\textwidth}{!}{
    \begin{tabular}{l c c c c}
        \toprule
        研究 & 評価スコア分析 & 自由記述分析 & 統合分析 & 要因の定量化 \\
        \midrule
        Gottipati et al. (2018) & − & ○ & − & − \\
        Misuraca et al. (2021) & − & ○ & − & △ \\
        Hujala et al. (2020) & ○ & ○ & △ & − \\
        Sindhu et al. (2019) & − & ○ & − & △ \\
        \textbf{本研究} & \textbf{○} & \textbf{○} & \textbf{○} & \textbf{○} \\
        \bottomrule
    \end{tabular}
    }
\end{table}

本研究の新規性は以下の3点である.

\begin{enumerate}
\item 感情スコアと授業評価スコアを同時に学習するマルチタスクモデルを構築し,両者の関係を統合的にモデル化する.
\item SHAP分析により,共通要因と特化要因を語彙レベルで分離し,改善施策の優先順位付けに利用できる定量的根拠を提供する.
\item 3,268授業,83,851件の自由記述という大規模データを用いて,統計的に信頼性の高い分析を行う.
\end{enumerate}

これにより,教育改善に資する具体的な知見を提供することを目指す.
