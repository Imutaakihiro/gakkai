\chapter{心理的効果の GNL モデルによる表現}
%%%%%%%%%%%%%%%%

\section{6 章の目的}
効用最大化行動は,個々の主体の行動を記述するミクロ経済学における消費者の重要な行動原理であり,
マーケティング・サイエンスの分野においても,心理学を基本とした記述的行動原理に対し,効用最大化行動は規範的行動原理\cite{Gil09}として捉えられている.
特にマーケティング・サイエンスの分野において広く用いられている離散選択モデルは,その多くが
効用最大化を緩和したランダム効用最大化行動(RUM)\cite{Man77}より導くことができる.
しかし,実際の選択行動において,この効用最大化と矛盾するとされる現象が現実の選択行動においていくつか示されている.

効用最大化と矛盾する現象として,妥協効果 \cite{Sim89},魅力効果 \cite{HPP82},類似性効果 \cite{Tve72a} が挙げられる.
類似性効果については,McFadden (1984)が Nested Logit (NL) モデルを用いることで
効用最大化と対応し,その生起を POS データ等の選好結果のみから非集計的に推定可能であることを示している.
類似性効果は,I.I.A. (Independence from Irrelevant Alternatives) 特性を犯しており,多くの製品カテゴリー,特に交通市場において顕著である.
マーケティング・サイエンスにおいても,特に需要予測,プロモーション効果の測定といった
目的でモデルを構築する際には,これらの現象をモデル内において表現できることは重要である.
プロダクト・ラインの拡張を考えた場合,属性空間上のどこに新製品を投入したらシェアを増すことが出来るのかということが分析可能となる.
また,店頭においてもどの商品を陳列したらよいかということにつながるだろう.

妥協効果,魅力効果については,多くの製品カテゴリーや分野において観測されている~\cite{HP83,BT02,DS03,KNVb04,NDS07}.
妥協効果,魅力効果については,ランダム効用最大化行動に基づくモデルで具体的に表現可能とした研究は存在しない.
また,これらの効果は,効用最大化と対応して生起することは具体的には示されていない.
妥協効果については,Rieskamp et al. (2006)\cite{RBM06} において効用最大化と整合的な GEV (Generalized Extreme Value) モデルにおいて表現可能と記されているが,具体的なモデルや研究の言及はない.
Rieskamp et al. では,魅力効果は Regularity を犯しているため,効用最大化と整合的な GEV モデルでは表現できないとしている.

複数のこのような現象を同じモデルで表現できることは,市場で起こりうる事象を模写できるという意味において,そのモデルの妥当性を示しているといえる.
Tversky and Simonson (1993) \cite{TS93}では,同じモデルで妥協効果と類似性効果を表現できるとしている.
また,Roe et al. (2001) \cite{RBT01}では,妥協効果,魅力効果,類似性効果の三つ全てが同じモデルで表現できるとしている.
これ以外にも,妥協効果を表現可能としている心理学のモデルは数多い\cite{DS03,KZK11,OJR08}.
ただし,これらのモデルは心理学的なモデルであり,効用最大化と整合的ではない.
また,魅力効果については,表現可能であるとしたモデルの数自体が少ない.

本章では,効用最大化と矛盾するとされている三つの現象,類似性効果,魅力効果,妥協効果,全てが効用最大化と整合的に,GNL モデルのある一つの構造で起こり得ることを示す.
GNL モデルは NL モデルを内包しているため,このモデルを用いることにより,妥協効果,魅力効果だけではなく類似性効果を併せ,三つ全ての効果を一つのモデルで説明可能となる.
妥協効果についてはいくつかの定義が考えられるが,このうち文脈依存ではない定義についてはいずれも GNL モデルにより説明することができる.
魅力効果についてもいくつかの定義が考えられるが,このうち相対的な確率のもとで定義した弱魅力効果について,GNL モデルにより説明できることを示す.

GNL モデルは非集計モデルであるため,今まで集計的な確率で述べられていたこれらの効果について,異なる状況設定下での心理的効果の
検証が非集計的に可能となることを意味している.
つまり,実験として行われている心理的効果の検証が,POS データ等の実際の購買行動から直接,非集計的に観測可能になることを意味している.
このことは,今までその大半が実験的環境下で生起することが確かめられていた心理的効果が,広く市場で生起するか観測可能になることに繋がる.

本章の構成は次のとおりである.まず $2$ 節において,合理的意思決定と心理的効果について既存研究を紹介し,それらの問題点を整理する.
次に,$3$ 節では心理的効果の定義を数式を用い行ない,その関係性を示す.
$4$ 節では,$3$ 節で定義された弱妥協効果,強妥協効果それぞれが GNL モデルを用いて効用最大化と対応して生起することを示す.また,その効果の表現の限界を示す.
同じく $5$ 節において$3$ 節で定義された弱魅力効果が GNL モデルを用いて効用最大化と対応して生起することを示す.
最後に $6$ 節で本章の結論を示す.
%%%%%%%%%%%%%%%%%%%%%%%%%%%%%%%%%%%%%%%%%%%%%%%%%%%%%%%%%%%%%%%%%%%%%%%%%%%%%%%%%%%%%%%%%%%%%%%%%%%%%%%%%%%%%%%%%%%%%%%%%%%%%%%%%%%%%%%%%%%%%%%%%%%%%%%%%%%%%%%%%%%%%%%%%%%%%%%%%
\section{心理的効果を表現可能なモデル}
心理的効果が表現できるモデルとしては,妥協効果について説明している Wernerfelt (1995) \cite{Wer95},Kiverz et al. (2004a) \cite{KNVa04},
類似性効果について説明しているMcFadden (1984) \cite{McF84} が挙げられる.
ただし, 妥協効果,魅力効果,類似性効果 ,計三つの効果を合理的意思決定のもとで同時に表現できるとした既存研究はない.
これら三つの効果はいずれも市場もしくは実験により実際に観測された事実である.
すなわちこれら 三つの効果を同じモデルで表現可能であることは,そのモデルの妥当性を示すこととなる \cite{RBT01}.

心理的効果について説明している離散選択モデルとして,Wernerfelt (1995)\cite{Wer95},Kiverz et al. (2004a) \cite{KNVa04},Rooderkerk, et al. \cite{RHB11} について説明しよう. 
Wernerfelt (1995)\cite{Wer95} は消費者に異質性を仮定しランクオーダー意思決定ルールに従うとすると,妥協効果が生起するとしている.
このルールのもとでは,消費者は選択肢が多い場合には十分比較できることから合理的選択が可能であり,
少ない場合には比較対象も少なくなるため合理的選択ができなくなるとしている.
このルールは選択肢集合の大きさに意思決定が依存するある種の文脈依存を表現しており,効用最大化行動とは異なるものである.
また,各選択肢の効用値をスキャン・パネル・データ等から推定するには多くのデータを必要とし,現実問題への適用も困難である.

Kiverz et al. (2004a) \cite{KNVa04}は Contextual Concavity Model (CCM),Normalized Contextual Concavity Model (NCCM),Relative Advantage Model (RAM),Loss-Aversion Model (LAM) という四つの離散選択モデルを提案し,四つのモデルそれぞれで妥協効果が生起するとしている.
それぞれのモデルにおける確定的効用は次のとおりである:
\begin{align}
&{\sf [CCM]}\notag\\
&~V_{k|\K}^h = \theta \sum_{m^{\prime}=1}^{N_m} \left( \upsilon_{m^{\prime}k}^h - \underline{\upsilon}_{m^{\prime}k}^{\K h} \right)^{\theta_i},\label{eq6-1}\\
&{\sf[NCCM]}\notag\\
&V_{k|\K}^h= \theta \sum_{m^{\prime}=1}^{N_m} \left( \overline{\upsilon}_{m^{\prime}k}^{\K h} - \underline{\upsilon}_{m^{\prime}k}^{\K h} \right) \left[ \frac{\upsilon_{m^{\prime}k}^h - \underline{\upsilon}_{m^{\prime}k}^{\K h}}{\overline{\upsilon}_{m^{\prime}k}^{\K h}-\underline{\upsilon}_{m^{\prime}k}^{\K h}} \right]^{\theta_i},\label{eq6-2}\\
&{\sf[RAM]}\notag\\
&~V_{k|\K}^h= \theta \left( \theta_1 \upsilon_k + \theta _2 \sum_{l \not= k} \frac{\sum \limits_{m^{\prime}=1}^{N_m} (\upsilon_{m^{\prime}k} - \upsilon_{m^{\prime}l}) {\boldsymbol 1}_{\{ \upsilon_{m^{\prime}k} > \upsilon_{m^{\prime}l} \}} }{\sum \limits_{m^{\prime}=1}^{N_m} \! (\upsilon_{m^{\prime}k} - \upsilon_{m^{\prime}l}) {\boldsymbol 1}_{\{ \upsilon_{m^{\prime}k} >\upsilon_{m^{\prime}l} \}} + \sum \limits_{m^{\prime}=1}^{N_m} \! (\upsilon_{m^{\prime}l} - \upsilon_{m^{\prime}k}) {\boldsymbol 1}_{\{ \upsilon_{m^{\prime}k} \leq \upsilon_{m^{\prime}l} \}}} \right), \label{eq6-3}\\
&{\sf[LAM]}\notag\\
&~V_{k|\K}= \theta \sum_{m^{\prime}=1}^{N_m} \left[ \left( \upsilon_{m^{\prime}k}- \upsilon_{m^{\prime}kR}^{\K} \right) {\boldsymbol 1}_{\{\upsilon_{m^{\prime}k} \geq \upsilon_{m^{\prime}kR}^{\K} \}} + \theta_m \left( \upsilon_{m^{\prime}k}- \upsilon_{m^{\prime}kR}^{\K} \right) {\boldsymbol 1}_{\{\upsilon_{m^{\prime}k} < \upsilon_{m^{\prime}kR}^{\K}\}} \right].\label{eq6-4}
\end{align}
ここで,$V_k^{\K h}$ は選択肢集合 $\K$ が提示されたとき,消費者 $h$ の選択肢 $k \in \K$ の確定的効用,$\upsilon_{mk}^h$ は選択肢 $k$ の $m$ 番目の属性に関する部分確定的効用,$\overline{\upsilon}_{mk}^{\K h}$,$\underline{\upsilon}_{mk}^{\K h}$はそれぞれ選択肢集合$\K$ のもとでの $m$ 番目の属性に関する部分確定的効用の最大値,最小値であり,$\upsilon_{mkR}^{\K h}$ は選択肢集合 $\K$ のもとでの $m$ 番目の属性の参照点である.
また,$\1_{\{\cdot \}}$ は,$\{\cdot\}$ 内の条件のとき $1$,それ以外は $0$ となる指示関数である.そして,$\theta$,$\theta_1$,$\theta_2$,$\theta_k$ はそれぞれパラメータである.
なお,簡便化のため消費者に関する添字は省略している.
選択肢集合 $\K$ が提示され,式\eqref{eq6-1}--\eqref{eq6-4}の確定的効用が与えられた場合の各モデルにおける選択肢 $k$ 選択確率 $P_k^{\K}$ は,MNL モデルのそれ:
\begin{align}
Q_k^{\K h}= \frac{\exp V_{k|\K}^h}{\sum \limits_{k^{\prime}=1}^{N_k} \exp V_{k^{\prime}|{\K}}^{h}} \label{eq6-5}
\end{align}
で与えられる.

これら四つのモデルは全て効用最大化と矛盾している.これは,端的に述べると確定的効用 $V_k^{\K}$ が選択肢集合 $\K$ に依存し,他の選択肢の影響が反映されているためである. 
CCM,NCCM は,選択肢の各属性に関する(確定的)部分効用が提示された選択肢の最低値を基底とするため,確定的効用が他の選択肢の影響を受けている.RAM  は確定的効用項に累積アドバンテージ,ディスアドバンテージ項が入っているため,他の選択肢 ($l \not = k$) の効用の影響を受けている.
LAM も参照点 $\upsilon_{mkR}^{\K}$ が選択肢集合 ${\K}$ に依存しているため,確定的効用が他の選択肢の影響を受けている.
つまり,距離空間が歪んでおり,効用最大化と整合的ではない.またこのことは,パラメータ推定時に異なる選択肢集合において複数回の推定が必要となることを意味している.
これは,実行可能性の面が大きく損なわれているといえるだろう.

Rooderkerk, et al. \cite{RHB11} は Kiverz et al. (2004a) \cite{KNVa04} と同様に確定的効用関数を提示されている選択肢集合 $\K$ に依存するようにし,
妥協効果だけではなく,魅力効果も表現できるとしている.

類似性効果については,McFadden (1984) \cite{McF84} が GEV モデルの一種である Nested Logit モデルを用いて,効用最大化と整合的に類似性効果は説明できることを述べている.
もっとも,全ての効果を一つの合理的意思決定のモデルで表現した研究は皆無である.
%%%%%%%%%%%%%%%%%%%%%%%%%%%%%%%%%%%%%%%%%%%%%%%%%%%%%%%%%%%%%%%%%%%%%%%%%%%%%%%%%%%%%%%%%%%%%%%%%%%%%%%%%%%%%%%%%%%%%%%%%%%%%%%%%%%%%%%%%%%%%%%%%%%%%%%%%%%%%%%%%%%%%%%%%%%%%%%%%%
\section{心理的効果の再定義}
本節では,Simonson (1989) \cite{Sim89},Huber et al. (1982) \cite{HPP82} により示された妥協効果,魅力効果の数式を用いた再定義を行なう.
前者については,Simonson (1989) \cite{Sim89} が厳密な定義を行なわなかったために,論文により複数の定義が用いられているためである.
後者については,数式による厳密な定義が未だなされていないためである.
いずれの場合においても,選択肢集合の提示順序が選択確率に影響を与える場合(文脈依存効果)が考えられるが,本研究ではそれを排除する.

\subsection{妥協効果の再定義}
妥協効果については,条件の強弱により弱妥協効果と強妥協効果二つの定義を行なう.
Simonson (1989) \cite{Sim89} は三つの選択肢集合で実験を行ない,妥協効果を示している.
妥協効果とは,言葉で書き表わすならば,二つの属性による属性空間において $\Ar$,$\Br$ いずれの選択肢も支配的でない状況でその中間の属性を持つ新たな選択肢 $\Cr$ を加えた場合,中庸な新たな選択肢 $\Cr$ の選択確率が一番高くなるというものである(図\ref{fig:6-1}).
\begin{figure}[t]
  \centering
  \psbox[width=0.55 \linewidth]{clip6-1}
\caption{妥協効果,魅力効果が生起する状況}
\label{fig:6-1}
\end{figure}
または,同様の空間,選択肢において $\Ar,\Cr$ が提示されている状況で新たな選択肢 $\Br$ を加えることにより,選択肢 $\Cr$ が極端な選択肢ではなくなり,選択肢 $\Cr$ の選択確率が上昇するというものである.

\begin{difinition} {\bf 弱妥協効果 }
$2$ 属性空間において,互いに支配的でない選択肢 $\Ar$,$\Br$ が存在し,その中間に選択肢 $\Cr$ が存在するとしよう:
	\begin{eqnarray}
 	X_{\Ar 1} > X_{\Cr 1} > X_{\Br 1}, \label{eq6-6}\\
 	X_{\Br 2} > X_{\Cr 2} > X_{\Ar 2}. \label{eq6-7}
	\end{eqnarray}
ここで,$X_{\Ar 1}$ は選択肢 $\Ar$ の属性 $1$ の属性値を表わす.
このとき,各選択肢集合における各選択確率が以下の条件を満たす場合,弱妥協効果 $\C_{\W}$ が生起しているという:
	\begin{eqnarray}
	\frac{Q_{\Cr| \left\{ \Ar, \Br, \Cr \right\}}^h}{Q_{\Cr| \left\{ \Ar, \Br, \Cr \right\}}^h+Q_{\Ar| \left\{ \Ar,\Br,\Cr \right\}}^h}>Q_{\Cr| \left\{ \Ar, \Cr \right\}}^h,\label{eq6-8}\\
 	\frac{Q_{\Cr| \left\{ \Ar, \Br, \Cr \right\}}^h}{Q_{\Cr| \left\{ \Ar, \Br, \Cr \right\}}^h+Q_{\Br| \left\{ \Ar, \Br,\Cr \right\}}^h}>Q_{\Cr| \left\{ \Br, \Cr \right\}}^h.\label{eq6-9}
	\end{eqnarray}
ここで,$Q_{\Cr| \left\{ \Ar,\Br,\Cr \right\}}^h$ は選択肢集合 ${\Ar,\Br,\Cr}$ が提示されている場合に選択肢 $\Cr$ を選択する確率を表わす.
上式は,新たな選択肢を加えることにより,選択肢 $\Cr$ が極端な選択肢ではなくなり,相対的確率が上昇することを意味している.
弱妥協効果は,{\rm Simonson,Simonson and Tversky (1992) \cite{ST92},Tversky and Simonson (1993) \cite{TS93},Wernerfelt (1995) \cite{Wer95},Kivetz et al. (2004a) \cite{KNVa04},Kivetz et al. (2004b) \cite{KNVb04}}で用いられている定義に相当する.
なお,式{\rm (\ref{eq6-8}),(\ref{eq6-9})}いずれかしか満たしていない場合は,ポラリゼーションと呼び,妥協効果と区別するものとする.
\end{difinition}

\begin{difinition} {\bf 弱妥協効果の大きさ }
弱妥協効果の大きさ $\delta_{\C}$ は,式{\rm (\ref{eq6-8}),(\ref{eq6-9})}それぞれの左辺から右辺を引いたものの最小値とする:
	\begin{eqnarray}
	\delta_{\C} := \min \left( \frac{Q_{\Cr| \left\{ \Ar, \Br, \Cr \right\}}^h}{Q_{C| \left\{ \Ar, \Br, \Cr \right\}}^h+Q_{\Ar| \left\{ \Ar, \Br, \Cr \right\}}^h}-Q_{\Cr| \left\{ \Ar,\Cr \right\}}^h, \frac{Q_{\Cr| \left\{ \Ar, \Br, \Cr \right\}}^h}{Q_{\Cr| \left\{ \Ar, \Br, \Cr \right\}}^h+Q_{\Br| \left\{ \Ar, \Br, \Cr \right\}}^h}-Q_{\Cr| \left\{ \Br, \Cr \right\}}^h \right).\label {eq6-10}
	\end{eqnarray}
	つまり,$\delta_{\C}>0$ ならば弱妥協効果が成立していることとなる.
\end{difinition}

一方,Roe et al. (2001) \cite{RBT01}では,相対的ではなく,絶対的確率に基づく妥協効果の定義をしている.ただし,これらの選択肢集合の提示順序については特に定めてはいない.

\begin{difinition} {\bf 強妥協効果 }
$2$ 属性以上の空間において,互いに支配的でない選択肢 $\Ar$,$\Br$ が存在し,その中間に選択肢 $\Cr$ が存在するとき,各選択肢集合における各選択確率は以下の条件を満たすとき強妥協効果 $\C_{\S}$ が生起しているという:
\begin{align}
Q_{\Ar| \left\{ \Ar, \Br \right\}}^h &=Q_{\Ar| \left\{ \Ar, \Cr \right\}}^h=Q_{\Br| \left\{ \Br, \Cr \right\}}^h=\frac{1}{2},\label{eq6-11}\\
Q_{\Cr| \left\{ \Ar, \Br, \Cr \right\}}^h&>Q_{\Ar| \left\{ \Ar, \Br, \Cr \right\}}^h,  \label {eq6-12}\\
Q_{\Cr| \left\{ \Ar, \Br ,\Cr \right\}}^h&>Q_{\Br | \left\{ \Ar, \Br, \Cr \right\}}^h. \label {eq6-13}
\end{align}
\end{difinition}

強妥協効果は,Roe et al. (2001) \cite{RBT01},Busmeyer et al. (2007) \cite{BBM07}で用いられている定義に相当する.Roe et al. (2001) とは条件({\ref{eq6-11})が異なる.
この点については,付録 \ref{sec:6-A-1} に示す.

\begin{lemma}
弱妥協効果は強妥協効果を包含する:
	\begin{equation}
	\C_{\S} \Longrightarrow  \C_{\W}. \label{eq6-14}
	\end{equation}
\end{lemma}

\begin{proof}
式{\rm(\ref{eq6-12}),式(\ref{eq6-13})}を足し合わせることにより,
	\begin{align}
	Q_{\Cr| \left\{ \Ar, \Br, \Cr \right\}}^h &> \frac{1}{2} \left(Q_{\Ar| \left\{ \Ar, \Br, \Cr \right\}}^h +Q_{\Br| \left\{ \Ar, \Br, \Cr \right\}}^h \right) = \frac{1}{2} \left(1-Q_{\Cr| \left\{ \Ar, \Br, \Cr \right\}}^h \right) \Longrightarrow Q_{\Cr| \left\{ \Ar, \Br, \Cr \right\}}^h >\frac{1}{3} \label{eq6-15}
	\end{align}
が導かれる.式{\rm(\ref{eq6-12}),式(\ref{eq6-15})}より,$Q_{\Ar| \left\{ \Ar, \Br,\Cr \right\}}<1/3$ であるため,
式{\rm(\ref{eq6-11}),式(\ref{eq6-15})}より,
	\begin{align}
	&\frac{Q_{\Cr| \left\{ \Ar, \Br, \Cr \right\}}^h}{Q_{\Cr| \left\{ \Ar, \Br, \Cr \right\}}^h+Q_{\Ar| \left\{ \Ar, \Br, \Cr \right\}}^h}>\frac{1}{2} =1-Q_{\Ar| \left\{ \Ar ,\Cr \right\}}^h=Q_{\Cr| \left\{ \Ar ,\Cr \right\}}^h \label{eq6-16}
	\end{align}
となり,式{\rm(\ref{eq6-8})}を満たす.選択肢 $\Br$ についても式{\rm(\ref{eq6-11}), (\ref{eq6-13}), (\ref{eq6-15})}より式{\rm(\ref{eq6-9})}を満たすことが同様に示せる.
また,$\C_{\S} \not\Leftrightarrow \C_{\W}$ であることは自明である.\QED
\end{proof}

\subsection{魅力効果}
魅力効果については,今まで数式を用いて定義されたことはない.
ここでは,弱妥協効果,強妥協効果と同様に,相対的,絶対的定義をそれぞれ弱魅力効果,強魅力効果として定義する.

魅力効果を言葉で書き表わすならば,二つの属性による属性空間において $\Ar $,$\Br$ いずれの選択肢も支配的でない状況で選択肢 $\Ar $ に類似しているが,魅力的ではない選択肢 $\Dr$ を加えた場合,新たな選択肢 $\Dr$ に類似している
選択肢 $\Ar $ の選択確率が高くなるというものである(図\ref{fig:6-1}).
魅力効果を最初に定義したHuber et al. (1982) \cite{HPP82} では,これは選択公理における Regularity を犯す現象として定義している.しかし,Hagerty (1983) \cite{Hag83},Ratneshwar et al.(1987) \cite{RSS87} では,絶対的選択確率が増すという定義を拡張し,相対的選択確率が増すということも含むようになっている.

\begin{difinition} {\bf 弱魅力効果 }
$2$ 属性の空間において,互いに支配的でない選択肢 $\Ar $,$\Br$ が存在し,選択肢$\Ar $に属性は類似し,選択肢 $\Ar $ に支配されている(選択肢 $\Br$ には支配されない)選択肢 $\Dr$ が存在するとしよう:
\begin{align}
X_{\Ar 1}&>X_{\Dr1}>X_{\Br 1},\label{eq6-17}\\
X_{\Br 2}&>X_{\Ar 2}>X_{\Dr2}. \label{eq6-18}
\end{align}
このとき,各選択肢集合における各選択確率が以下の条件を満たす場合,弱魅力効果 $\cA_{\W}$ が生起しているという:
	\begin{eqnarray}
	\frac{Q_{\Ar | \left\{ \Ar , \Br, \Dr \right\}}^h}{Q_{\Ar | \left\{ \Ar ,\Br, \Dr \right\}}^h + Q_{\Br| \left\{ \Ar ,\Br, \Dr \right\}}^h}>Q_{\Ar | \left\{ \Ar ,\Br \right\}}^h.\label{eq6-19}
	\end{eqnarray}
式{\rm(\ref{eq6-19})}は,新たな選択肢 $\Dr$ を加えることにより,選択肢 $\Dr$ に近い属性を持つ選択肢 $\Ar $ が魅力的になり,相対的確率が上昇することを意味している.
\end{difinition}
\begin{difinition} {\bf 弱魅力効果の大きさ }
弱魅力効果の大きさ $\delta_{\cA }$ は,式{\rm(\ref{eq6-19})}の左辺から右辺を引いたものとする:
	\begin{eqnarray}
	\delta_{\cA}:=\frac{Q_{\Ar | \left\{ \Ar ,\Br ,\Dr \right\}}^h}{Q_{\Ar | \left\{ \Ar ,\Br ,\Dr \right\}}^h + Q_{\Br | \left\{ \Ar ,\Br ,\Dr \right\}}^h}-Q_{\Ar | \left\{ \Ar ,\Br  \right\}}^h.\label{eq6-20}
	\end{eqnarray}
	つまり,$\delta_{\cA }>0$ ならば魅力効果が成立していることとなる.
\end{difinition}
\begin{difinition} {\bf 強魅力効果 }
$2$ 属性の空間において,互いに支配的でない選択肢 $\Ar $,$\Br $ が存在し,選択肢$\Ar $に属性は類似し,選択肢 $\Ar $ に支配されている(選択肢 $\Br $ には支配されない)選択肢 $\Dr$ が存在するとしよう:
\begin{align}
X_{\Ar 1}&>X_{\Dr1}>X_{\Br 1},\label{eq6-21}\\
X_{\Br 2}&>X_{\Ar 2}>X_{\Dr2}.\label{eq6-22}
\end{align}
このとき,各選択肢集合における各選択確率が以下の条件を満たす場合,強魅力効果 $\cA_{\S}$ が生起しているという:
	\begin{align}
	Q_{\Ar | \left\{ \Ar ,\Br ,\Dr \right\}}^h>Q_{\Ar | \left\{ \Ar ,\Br  \right\}}^h.\label{eq6-23}
	\end{align}
上式は,新たな選択肢 $\Dr$ を加えることにより,選択肢 $\Dr$ に近い属性を持つ選択肢 $\Ar $ が魅力的になり,絶対的確率が上昇することを意味している.
\end{difinition}

\begin{lemma}
弱魅力効果は強魅力効果を包含する:
	\begin{equation}
	\cA_{\S} \Longrightarrow  \cA_{\W}.\label{eq6-24}
	\end{equation}
\end{lemma}

\begin{proof}
$Q_{\Ar |\{\Ar ,\Br ,\Dr\}}^h$,$Q_{\Br |\{\Ar ,\Br ,\Dr\}}^h$は確率であるため,
\begin{align}
Q_{\Ar |\{ \Ar ,\Br ,\Dr \}}^h+Q_{\Br |\{ \Ar ,\Br ,\Dr \}}^h \leq 1\label{eq6-25}
\end{align}
である.従って,
\begin{align}
 \frac{Q_{\Ar | \left\{ \Ar ,\Br ,\Dr \right\}}^h}{Q_{\Ar | \left\{ \Ar ,\Br ,\Dr \right\}}^h+Q_{\Br | \left\{ \Ar ,\Br ,\Dr \right\}}^h} \geq Q_{\Ar | \left\{ \Ar ,\Br ,\Dr \right\}}^h > Q_{\Ar | \left\{ \Ar ,\Br  \right\}}^h\label{eq6-26}
\end{align}
となり,式\eqref{eq6-23}は\eqref{eq6-19}を満たしている.
また,$\cA_{\S} \not\Leftrightarrow \cA_{\W}$ であることは自明である.\QED
\end{proof}
%%%%%%%%%%%%%%%%%%%%%%%%%%%%%%%%%%%%%%%%%%%%%%%%%%%%%%%%%%%%%%%%%%%%%%%%%%%%%%%%%%%%%%%%%%%%%%%%%%%%%%%%%%%%%%%%%%%%%%%%%%%%%%%%%%%%%%%%%%%%%%%%%%%%%%%%%%%%%%%%%%%%%%%%%%%%%%%%%%
\section{妥協効果の GNL モデルにおける生起}
本章では,まず弱,強妥協効果それぞれが実際に GNL モデルを用いて生起することを示し,意味解釈を行なう.
また,各状況下での興味深い性質についても示す.

\subsection{弱妥協効果の数値例}
具体的な数値を示す前に,いくつかの仮定を置く.
対象商品を乗用車とし,属性を走行性能と燃費の二つとする.
そして,消費者の効用関数の確定項 $V_k^h$ を補償型とする:
\begin{equation}
V_k^h=\alpha_1 X_{1k} + \alpha_2 X_{2k}.\label{eq6-27}
\end{equation}
ここで,$V_k^h$ は選択肢 $k$ の確定的効用,$X_{1k}$ は選択肢 $k$ の走行性能,$X_{2k}$ は燃費,$\alpha_1$,$\alpha_2$ はそれぞれのパラメータである.
固有選好度はここでは意味をなさないため,省略する.
そして,GNL モデルのネスト数 $N_j$ を $3$ とし,それぞれに全ての選択肢 $\Ar $,$\Br $,$\Cr$ がネスティングされるものとする.具体的な構造を図\ref{fig:6-2}に示す.

以上の仮定のもと 表 \ref{tb:6-1}に示すパラメータを用い,選択肢 $\Cr$ の追加前後の選択確率を計算すると,
\begin{align}
Q_{\Ar |\left\{\Ar ,\Br \right\}}^h&=Q_{\Br |\left\{\Ar ,\Br \right\}}^h=0.5,\label {eq6-28}\\
Q_{\Ar |\left\{\Ar .\Br ,\Cr\right\}}^h&=Q_{\Br |\left\{\Ar , \Br ,\Cr\right\}}^h=0.332,\label {eq6-29}\\
Q_{\Cr|\left\{\Ar ,\Br ,\Cr\right\}}^h&=0.336,\label {eq6-30}
\end{align}
となる.式(\ref{eq6-28})--(\ref{eq6-30})は式(\ref{eq6-8}),式(\ref{eq6-9})を満たしており,GNLモデルにおいて弱妥協効果が生起していることがわかる.
ただし,強妥協効果については,$Q_{\Ar |\left\{\Ar ,\Cr\right\}}=Q_{\Br |\left\{\Br ,\Cr\right\}}\not=0.5$ であるため,この数値例では生起していない.
\begin{table} [tb:ex1]
\begin{center}
\caption{弱妥協効果の生起例パラメータ}  \label{tb:6-1}
\begin{tabular} {ccp{0.5zw}rp{0.8zw}ccp{0.5zw}r}
\toprule
\multicolumn{2}{c}{Variable}& && &\multicolumn{2}{c}{Variable}& &\\
\hline\hline
\multirow{6}{*}{Attribute Value}& $X_{1\Ar }$&& $3.0$ &&Similarity Parameter&$\mu_3$&&$0.9$\\
&$X_{1\Br }$&& $1.0$ &&\multirow{9}{*}{Allocation Parameter}&$\gamma_{\Ar 1}$&&$0.7$\\
&$X_{1\Cr}$&& $2.0$ &&&$\gamma_{\Ar 2}$&&$0.2$\\
&$X_{2\Ar }$&& $1.0$ &&&$\gamma_{\Ar 3}$&&$0.1$\\
&$X_{2\Br }$&& $3.0$ &&&$\gamma_{\Br 1}$&&$0.2$\\
&$X_{2\Cr}$&& $2.0$ &&&$\gamma_{\Br 2}$&&$0.6$\\
\multirow{2}{*}{Definite Utility Parameter}&$\alpha_1$&&$1.0$&&&$\gamma_{\Br 3}$&&$0.2$\\
&$\alpha_2$&&$1.0$&&&$\gamma_{\Cr1}$&&$0.1$\\
\multirow{2}{*}{Similarity Parameter}&$\mu_1$&&$0.9$&&&$\gamma_{\Cr2}$&&$0.2$\\
&$\mu_2$&&$0.7$&&&$\gamma_{\Cr3}$&&$0.7$\\
\bottomrule
\end{tabular}
\end{center}
\end{table}

\begin{figure}[t]
  \centering
  \psbox[width=0.55 \linewidth]{clip6-2}
\caption{弱妥協効果が生起するGNLモデルの構造}
\label{fig:6-2}
\end{figure}

妥協効果が生起するこの GNL モデルの構造についての意味解釈をしよう.
既存の適用分野である交通機関選択や経路選択において,GNL モデルのネスティングは交通機関や経路の重なりあいを表現している.
単に GNL モデルという場合にはその構造について意味し,そのネスティングの意味解釈には何らかの客観的基準が求められる.
ここでは,意味解釈を容易にするため次に示す仮定を置く:
\begin{align}
&\gamma_{\Ar 1}>\gamma_{\Cr1}>\gamma_{\Br 1},\gamma_{\Br 3}>\gamma_{\Cr3}>\gamma_{\Ar 3},\label {eq6-31}\\
&\gamma_{\Cr2}>\gamma_{\Ar 2},\gamma_{\Cr2}>\gamma_{\Br 2},\label {eq6-32}\\
&\gamma_{\Ar 1}>\gamma_{\Ar 2}>\gamma_{\Ar 3},\gamma_{\Br 3}>\gamma_{\Br 2}>\gamma_{\Br 1},\label {eq6-33}\\
&\gamma_{\Cr2}>\gamma_{\Cr1},\gamma_{\Cr2}>\gamma_{\Cr3}.\label {eq6-34}
\end{align}
式(\ref{eq6-31})は,ネスト $1,3$ への所属割合がネスト $1$ では $\Ar ,\Cr,\Br $ の順に高く,ネスト $3$ では逆であることを意味している.
同様に式(\ref{eq6-32})は,ネスト$2$への所属割合は$\Cr$が$\Ar $,$\Br $より高いことを示している.
また,式(\ref{eq6-33})は選択肢$\Ar ,\Br $の所属割合が選択肢$\Ar $では$1$,$2$,$3$の順に高く,選択肢 $\Br $ では逆であることを意味している.
最後に式(\ref{eq6-34})は,選択肢 $\Cr$ の所属割合が $2$ が $1$,$3$ より高いことを示している.

この仮定より,このモデルの構造は,ネスト $1$,$3$ は極端な属性を持つ選択肢を好むクラス,ネスト $2$ は中庸な属性を持つ選択肢を好むクラスといえる.
ここでは,ネスト $1$ は走行性能が高い選択肢を好むクラス,ネスト $3$ は燃費がよい選択肢を好むクラス,そしてネスト $2$ は両者のバランスを重視するクラスと解釈できる.
そして,消費者はこれらの潜在的なクラスを選択し,次に各クラスに属する選択肢を選択するという意思決定過程をたどっていると解釈できる.
表 \ref{tb:6-1}に示すパラメータは明らかに式(\ref{eq6-30})--(\ref{eq6-34})を満たしており,この意思決定過程をたどる場合,弱妥協効果が生起することがあるといえる.

\begin{lemma}
{\rm NL}モデルにおいてネスティング時にオーバラップを許さない場合,弱妥協効果を表現することはできない.
\end{lemma}
\begin{proof}
	{\rm NL} モデルにおいて選択肢$\Ar $, $\Br $, $\Cr$ をオーバーラップせずにネスティングする方法は図\ref{fig:6-3} に示す $3$ とおりである.
	このうち {\rm (I), (II)} は対称であるため,{\rm (I), (III)} の二つのついて考えればよい.
	まず{\rm (I)}の場合を考えよう.選択肢 $\Ar ,\Br $ の対称性より,$V_{\Ar}^h=V_{\Br }^h := V^h$ である.
	弱妥協効果の大きさは,
	\begin{align}
		\delta_{\C}= \min &\left\{ \frac{\exp{V_{\Cr}^h}^{1/\mu_1}\left( \left( \exp V^h \right)^{1/\mu_1}  + \left( \exp V_{\Cr}^h \right)^{1/\mu_1} \right)^{\mu_1-1}}{\exp{V}^h + \exp{V_{\Cr}^h}^{1/\mu_1}\left( \left( \exp V^h \right)^{1/\mu_1} + \left( \exp V_{\Cr}^h \right)^{1/\mu_1} \right)^{\mu_1-1}}- \frac{\exp V_{\Cr}^h}{\exp{V^h} + \exp{V_{\Cr}^h}}, \right.\notag\\
		&~~~~~~~~\left. \frac{\left( \exp{V_{\Cr}^h} \right)^{1/\mu_1}}{\left( \exp{V^h} \right)^{1/\mu_1}+\left( \exp{V_{\Cr}^h} \right)^{1/\mu_1}} - \frac{\left( \exp{V_{\Cr}^h} \right)^{1/\mu_1}}{\left( \exp{V^h} \right)^{1/\mu_1}+\left( \exp{V_{\Cr}^h} \right)^{1/\mu_1}} \right\}\label{eq6-35}
	\end{align}
	となり,$\delta_{\C}=0$ となる.
	次に{\rm (III)}の場合を考える.この場合は$\Ar ,\Br $の配置についても対称であるため演算子$\min$ がなくなり,弱妥協効果の大きさは,
	\begin{align}
		\delta_{\C}= \frac{\exp{V_{\Cr}^h}}{\frac{1}{2} \left( \left( 2\exp{V^h} \right)^{1/\mu_1} \right)^{\mu_1} + \exp{V_{\Cr}^h}}-\frac{\exp{V_{\Cr}^h}}{\exp{V^h}+\exp{V_{\Cr}^h}}\label{eq6-36}
	\end{align}
	となり,$\delta_{\C}=0$ となる.\QED
\end{proof}
\begin{figure}[t]
  \centering
  \psbox[width=1.0 \linewidth]{clip6-3}
\caption{NL モデルによる選択肢 $\{\Ar ,\Br ,\Cr\}$ のネスティング}
\label{fig:6-3}
\end{figure}
\subsection{GNL モデルによる弱妥協効果に関する性質}
本節では,前節で生起することが確かめられた,GNL モデルにおける弱妥協効果の生起に関するいくつかの興味深い結果を示す.
\subsubsection{ネスト数と効果の生起の関係}
\begin{lemma}
{\rm GNL} モデルにおいてネスト数 $N_j$ が $3$ 未満かつ,各選択肢の確定的効用関数 $V_k^h \forall k \in {\K}$ が線形の場合,弱妥協効果は生起しない.
	\begin{equation}
	N_j<3,V_k^h := V^h,~\forall k~\in ~\K~~\Longrightarrow Q_{\Cr}^h \leq Q_{\Ar}^h= Q_{\Br}^h.\label{eq6-37}
	\end{equation}
\end{lemma}
\begin{proof}\label{pr02}
$N_j=1$ のとき,{\rm GNL} モデルは {\rm MNL} モデルとなるため,その {\rm I.I.A.} 特性より自明.
$N_j=2$ のとき,選択肢 $\Ar ,\Br $ 及びネスト $1,2$ の対称性より,$\gamma_{\Ar 1}=\gamma_{\Br 2}:=\gamma_{\Ar} $,$\gamma_{\Ar 2}=\gamma_{\Br 1}=1-\gamma_{\Ar} $,$\gamma_{\Cr1}=\gamma_{\Cr2}:=\gamma_{\Cr}$,$\mu_1=\mu_2:=\mu$ を得る.
また,同様に,$V_{\Ar}^h=V_{\Br}^h=V_{\Cr}^h := V^h$ であり,さらに $\exp{V^h} := Y^h$ としよう.すると,各選択肢集合下における選択肢$\Cr$の選択確率は,
\begin{align}
Q_{\Cr|\{\Ar ,\Br ,\Cr\}}^h  &= \frac{\gamma_{\Cr}^{1/\mu}}{\gamma_{\Ar }^{1/\mu} + (1-\gamma_{\Ar })^{1/\mu} + \gamma_{\Cr}^{1/\mu}}, \label{eq6-38}\\
Q_{\Cr|\{\Ar ,\Cr\}}^h  &= \frac{\gamma_{\Cr}^{1/\mu}}{\gamma_{\Ar}^{1/\mu} + \gamma_{\Cr}^{1/\mu}} \frac{\left( \gamma_{\Ar}^{1/\mu}+\gamma_{\Cr}^{1/\mu} \right)^\mu}{\left( \gamma_{\Ar}^{1/\mu}+\gamma_{\Cr}^{1/\mu} \right)^\mu + \left( (1-\gamma_{\Ar})^{1/\mu} +\gamma_{\Cr}^{1/\mu} \right)^{\mu}} \nonumber \\
&+\frac{\gamma_{\Cr}^{1/\mu}}{(1-\gamma_{\Ar})^{1/\mu} + \gamma_{\Cr}^{1/\mu}} \frac{\left( (1-\gamma_{\Ar})^{1/\mu}+\gamma_{\Cr}^{1/\mu} \right)^\mu}{\left( \gamma_{\Ar}^{1/\mu}+\gamma_{\Cr}^{1/\mu} \right)^\mu + \left( (1-\gamma_{\Ar})^{1/\mu}+\gamma_{\Cr}^{1/\mu} \right)^\mu}\label{eq6-39}
\end{align}
と表わされる.
ここで,$0<\mu\leq 1$ であるため,
	\begin{eqnarray}
	Q_{\Cr |\{\Ar ,\Br ,\Cr \}}^h  &\leq& \frac{1}{3},\label{eq6-40}\\
	Q_{\Cr |\{\Ar ,\Cr\}}^h  &\geq& \frac{1}{3}\label{eq6-41}
	\end{eqnarray}
を得る.ここで,再び対称性より,弱妥協効果の大きさ $\delta_c$ は,
	\begin{align}
	\delta_c  =& \frac{Q_{\Cr| \{\Ar, \Br, \Cr \}}^h}{Q_{C|\{\Ar, \Br, \Cr \}}^h +\left( 1- \frac{1}{2} Q_{\Cr|\{\Ar, \Br, \Cr \}}^h \right)} - Q_{\Cr|\{\Ar, \Cr \}}^h \nonumber \\
 =& \frac{Q_{\Cr|\{\Ar ,\Br ,\Cr \}}^h }{1+\frac{1}{2} Q_{\Cr|\{\Ar ,\Br ,\Cr \}}^h} - Q_{\Cr|\{\Ar ,\Cr \}}^h \nonumber\\
 :=& \Gamma \left( Q_{\Cr|\{\Ar ,\Br ,\Cr \}}^h \right) - Q_{\Cr|\{\Ar ,\Cr \}}^h \label{eq6-42}
	\end{align}
と表わされる.$\Gamma(Q_{\Cr|\{\Ar ,\Br ,\Cr\}}^h)$ は,区間$(0,1]$ において $Q_{\Cr|\{\Ar ,\Br ,\Cr\}}^h$ に関して単調増加であるため,
	\begin{equation}
	\Gamma (Q_{\Cr|\{\Ar ,\Br ,\Cr \}}^h) \leq \frac{2}{7}=g(1) \label{eq6-43}
	\end{equation}
である.式{\rm (\ref{eq6-39}), (\ref{eq6-43})}より,
	\begin{equation}
	\delta_c \leq \frac{2}{7} - \frac{1}{3} = -\frac{1}{21} < 0\label{eq6-44}
	\end{equation}
を得る.\QED
\end{proof}
\subsubsection{弱妥協効果の最大値}
本節では,提示された GNL モデルによる弱妥協効果の最大値を示し,現実がそれを超えるかどうかを検証する.

\begin{lemma}
	選択肢数,ネスト数ともに $3$ の {\rm GNL} モデルにおいて,効用関数が線形の場合,弱妥協効果の大きさ $\delta_{\C}$ には上限が存在し,その大きさは,
	\begin{equation}
		\overline{\delta}_{\C} = \frac{1}{6}\label{eq6-45}
	\end{equation}
	となる.ここで,$\overline{\delta}_{\C}$ は弱妥協効果の最大値を表わす.
\end{lemma}
\begin{proof}
付録 {\rm \ref{sec:6-A-2}} を参照せよ.\QED
\end{proof}
Simonson and Tvesky (1992) \cite{ST92} では,表 \ref{tb:6-2}に示す四つのカテゴリーで
弱妥協効果の大きさを測定している.それぞれのカテゴリーで複数の選択肢集合を提示しているため,
その中で最も大きいものと,小さいものとを示している.
ただし,Simonson and Tvesky (1992) \cite{ST92} は集計値,つまり割合である.
本章の GNL モデルは非集計モデルであるため,確率であるため,あくまで期待値として再現できるかを比較する.
値を比較すると,ポータブル・グリル以外はいずれもこのモデルを用いて表現可能であることが分かる.
従って,これら表現可能なカテゴリーについては,直ちに効用最大化と矛盾する現象の生起とは決めつけられないだろう.

\begin{table} [t]
\begin{center}
\caption{弱妥協効果の大きさと再現可能性}  \label{tb:6-2}
\begin{tabular} {p{12.0zw}p{8.5zw}p{8.5zw}p{5zw}}
\toprule
Category & $\max \delta_{c}$ & $\min \delta_{c}$ & Our model \\
\hline\hline
{\small Camera} &$0.071$&$0.071$&possible\\
{\small Battery} &$0.170$&$0.060$&possible\\
{\small Calculator} &$0.139$&$0.134$&possible\\
{\small Portable Grill}&$0.247$&$0.200$&impossible\\
\bottomrule
\end{tabular}
\end{center}
\end{table}
\subsubsection{擬似相関係数からみた弱妥協効果の解釈}
GNL モデルにおける選択肢 $x$, $y$ 間の相関係数 $\rho_{xy}$ は,2 章で示したように,近似的に以下の式で表わされる:
\begin{equation}
\rho _{xy} = \sum_{j=1}^{N_j} \gamma_{xj}^{1/2} \gamma_{yj}^{1/2} \left( 1-\mu_j^2 \right).\label{eq6-46}
\end{equation}
類似度パラメータが小さく,同じネストに所属するアロケーション・パラメータが大きいほど相関係数が大きい.
弱妥協効果は,中央(ネスト $2$)の類似度パラメータが相対的に小さく,外側への選択肢 $\Ar $,$\Br $ の帰属度が高いほど大きくなる.
すなわち,$\Ar $,$\Br $ の相関係数が高く,$\Ar $,$\Cr$ 及び $\Br $,$\Cr$ の相関係数が低い場合,弱妥協効果が生起しやすいといえる.
相関係数が高いということは,一般的にはランダム効用における誤差項が大きいことを示し,
比較的狭い属性空間に三つの選択肢が存在していると考えられる.
このことから,比較的狭い属性空間上において,弱妥協効果は生起しやすいといえる.

\subsection{強妥協効果の数値例}
本章では,GNL モデルにおける強妥協効果の生起について述べる.
ネストの構造,選択肢数,効用関数の仮定は,弱妥協効果と同様としよう.
これらの仮定のもと,表 \ref{tb:6-3}に示すパラメータを用い,各条件のもとでの選択肢の選択確率を計算すると,
\begin{align}
P_{\Ar |\left\{\Ar ,\Br \right\}}^h &=P_{\Ar |\left\{\Ar ,\Cr\right\}}^h=P_{\Br |\left\{\Br ,\Cr\right\}}^h=0.5,\label {eq6-47}\\
P_{\Ar |\left\{\Ar ,\Br ,\Cr\right\}}^h &=P_{\Br |\left\{\Ar ,\Br ,\Cr\right\}}^h=0.321,\label {eq6-48}\\
P_{\Cr|\left\{\Ar ,\Br ,\Cr\right\}}^h &=0.358 \label {eq6-49}
\end{align}
となる.式(\ref{eq6-47})--(\ref{eq6-49})は式(\ref{eq6-11})--(\ref{eq6-13})を満たしており,GNL モデルにおいて強妥協効果が生起していることがわかる.

\begin{table} [tb5-3]
\begin{center}
\caption{強妥協効果の生起例パラメータ}  \label{tb:6-3}
\begin{tabular} {ccp{0.2zw}rp{0.2zw}ccp{0.2zw}r}
\toprule
\multicolumn{2}{c}{Variable}& && &\multicolumn{2}{c}{Variable}& &\\
\hline\hline
\multirow{6}{*}{Attribute Value}& $X_{1\Ar }$&& $3.0$ &&Similarity Parameter&$\mu_3$&&$0.614$\\
&$X_{1\Br }$&& $1.0$ &&\multirow{9}{*}{Allocation Parameter}&$\gamma_{\Ar 1}$&&$0.7$\\
&$X_{1\Cr}$&& $2.0$ &&&$\gamma_{\Ar 2}$&&$0.1$\\
&$X_{2\Ar }$&& $1.0$ &&&$\gamma_{\Ar 3}$&&$0.2$\\
&$X_{2\Br }$&& $3.0$ &&&$\gamma_{\Br 1}$&&$0.2$\\
&$X_{2\Cr}$&& $2.0$ &&&$\gamma_{\Br 2}$&&$0.1$\\
\multirow{2}{*}{Definite Utility Parameter}&$\alpha_1$&&$1.0$&&&$\gamma_{\Br 3}$&&$0.7$\\
&$\alpha_2$&&$1.0$&&&$\gamma_{\Cr1}$&&$0.1$\\
\multirow{2}{*}{Similarity Parameter}&$\mu_1$&&$0.614$&&&$\gamma_{\Cr2}$&&$0.8$\\
&$\mu_2$&&$0.5$&&&$\gamma_{\Cr3}$&&$0.1$\\
\bottomrule
\end{tabular}
\end{center}
\end{table}

ただし,表 \ref{tb:6-3}のパラメータは式(\ref{eq6-11}),(\ref{eq6-13})を満たしていないことに注意されたい.
これは,いくつかの可能性を示唆している.
まず,もし GNL モデル及び仮定(\ref{eq6-30})--(\ref{eq6-33})にしたがって消費者が意思決定を行なっているとしたら,効用関数が線形ではなく,$V_{\Cr}^h>V_{\Ar}^h=V_{\Br}^h$となる非線形性が存在する.
もしくは,走行性能,燃費以外の属性が存在し,それにより式(\ref{eq6-30}),(\ref{eq6-32})を満たさない関係が生まれている.

\section{魅力効果}
\subsection{弱魅力効果の数値例}
魅力効果についても妥協効果と同様に数値例を用いて生起することを示そう.
対象商品を妥協効果と同様に乗用車とし,属性についても走行性能と燃費の二つとする.
そして,消費者の効用関数の確定項 $V_k^h$ も式\eqref{eq6-14}に示すものと同様とする.
そして,GNL モデルのネスト数 $N_j$ を $3$ とし,それぞれに全ての選択肢 $\Ar $,$\Br $,$\Dr$ がネスティングされるものとする.具体的な構造を図\ref{fig:6-4}に示す.

\begin{table} [tb5-4]
\begin{center}
\caption{弱魅力効果の生起例パラメータ}  \label{tb:6-4}
\begin{tabular} {ccp{0.3zw}rp{0.6zw}ccp{0.3zw}r}
\toprule
\multicolumn{2}{c}{Variable}& && &\multicolumn{2}{c}{Variable}& &\\
\hline\hline
\multirow{6}{*}{Attribute Value}& $X_{1\Ar }$&& $1.0$ &&Similarity Parameter&$\mu_3$&&$0.1$\\
&$X_{1\Br }$&& $0.0$ &&\multirow{9}{*}{Allocation Parameter}&$\gamma_{\Ar 1}$&&$0.9$\\
&$X_{1\Dr}$&& $0.4$ &&&$\gamma_{\Ar 2}$&&$0.05$\\
&$X_{2\Ar }$&& $0.0$ &&&$\gamma_{\Ar 3}$&&$0.05$\\
&$X_{2\Br }$&& $1.0$ &&&$\gamma_{\Br 1}$&&$0.2$\\
&$X_{2\Dr}$&& $-0.1$&&&$\gamma_{\Br 2}$&&$0.4$\\
\multirow{2}{*}{Definite Utility Parameter}&$\alpha_1$&&$1.0$&&&$\gamma_{\Br 3}$&&$0.4$\\
&$\alpha_2$&&$1.0$&&&$\gamma_{\Dr1}$&&$0.2$\\
\multirow{2}{*}{Similarity Parameter}&$\mu_1$&&$0.7$&&&$\gamma_{\Dr2}$&&$0.4$\\
&$\mu_2$&&$1.0$&&&$\gamma_{\Dr3}$&&$0.4$\\
\bottomrule
\end{tabular}
\end{center}
\end{table}
\begin{figure}[b]
  \centering
  \psbox[width=0.45 \linewidth]{clip6-4}
\caption{弱魅力効果が生起する GNL モデルの構造}
\label{fig:6-4}
\end{figure}

以上の仮定のもと,表 \ref{tb:6-4}に示すパラメータを用い,選択肢 $\Dr$ の追加前後の選択確率を計算すると,
\begin{align}
Q_{\Ar |\left\{\Ar ,\Br \right\}}^h=&0.495,~Q_{\Br |\left\{\Ar ,\Br \right\}}^h=0.505,\label {eq6-50}\\
Q_{\Ar |\left\{\Ar ,\Br ,\Dr\right\}}^h=&0.436,~Q_{\Br |\left\{\Ar ,\Br ,\Dr\right\}}^h=0.449,\label {eq6-51}\\
Q_{\Dr|\left\{\Ar ,\Br ,\Dr\right\}}^h=&0.115 \label {eq6-52}
\end{align}
となる.式(\ref{eq6-50})--(\ref{eq6-52})は式(\ref{eq6-19})を満たしており,GNL モデルにおいて魅力効果が生起していることがわかる.
ただし,弱魅力効果の大きさは,$\delta_{\cA}=0.0019$ と,弱妥協効果と比べ相対的確率の変化は小さいことに注意されたい.

さて,弱魅力効果は NL モデルで表現できるだろうか.NL モデルでは一般的に,属性が近い選択肢を同じネストに配置する.
\begin{lemma}
{\rm NL}モデルにおいてネスティング時にオーバラップを許さず,属性によるネスティングを行なった場合,弱魅力効果を表現することはできない.
\end{lemma}
\begin{proof}
条件を満たす {\rm NL} モデルの構造は図{\rm \ref{fig:6-5}}のみとなる.このとき,$\Ar $ と $\Br $ は違うネストに属し,
{\rm I.I.A} を満たすため,
\begin{align}
\delta_{\cA}:=\frac{Q_{\Ar | \left\{ \Ar ,\Br ,\Dr \right\}}^h}{Q_{\Ar | \left\{ \Ar ,\Br ,\Dr \right\}}^h+Q_{\Br | \left\{ \Ar ,\Br ,\Dr \right\}}^h}- Q_{\Ar | \left\{ \Ar ,\Br  \right\}}^h=0 \label{eq6-53}
\end{align}
となる.式\eqref{eq6-53}は\eqref{eq6-19}を満たさないため,弱魅力効果は生起しない.\QED
\end{proof}
\begin{figure}[t]
  \centering
  \psbox[width=0.45 \linewidth]{clip6-5}
\caption{NL モデルによる選択肢 $\{\Ar ,\Br ,\Dr\}$ のネスティング}
\label{fig:6-5}
\end{figure}
\subsection{強魅力効果と GNL モデル}
強魅力効果は,強妥協効果とは異なり,GNL モデルを用いて表現することはできない.
これは,GNL モデルが RUM モデルの一種であり,Regularity を犯していないことから自明である.

\begin{lemma}
{\rm GNL}モデルにおいて強魅力効果を表現することはできない.
\end{lemma}
\begin{proof}
{\rm Rieskamp et al. (2006)} {\rm Appendix C} \cite{RBM06} を見よ.\QED
\end{proof}
%%%%%%%%%%%%%%%%%%%%%%%%%%%%%%%%%%%%%%%%%%%%%%%%%%%%%%%%%%%%%%%%%%%%%%%%%%%%%%%%%%%%%%%%%%%%%%%%%%%%%%%%%%%%%%%%%%%%%%%%%%%%%%%%%%%%%%%%%%%%%%%%%%%%%%%%%%%%%%%%%%%%%%%%%%%%%%%%
\section{6 章のまとめ}
本章では,まず数理的に曖昧さや混同のある心理的効果について,数式を用いた定義を行なった.
まず,複数の定義がある妥協効果について,二つの定義,弱妥協効果,強妥協効果を定義し,強妥協効果は弱妥協効果に内包されることを示した.
同様に魅力効果について,二つの定義,弱魅力効果,強魅力効果を定義し,強魅力効果は弱魅力効果に内包されることを示した.

次に効用最大化行動と矛盾するとされ,NL モデルでは表現することが不可能であったこれらの効果について,効用最大化と整合的なランダム効用モデルである
GNL モデルにおいて生起することを示した.
妥協効果については,弱妥協効果,強妥協効果双方が同じネスティング構造を持つ GNL モデルで表現可能であることが分かった.
また,このうち弱妥協効果については,GNL モデルにおいて表現可能な大きさの範囲を示し,実際の実験結果と比較することにより,おおよそ実測に用いることができることが可能となった.
特にこの効果の生起については,擬似的な選択肢間の相関係数により解釈することができ,そのためにネスト数が $3$ つ必要であることが分かった.
魅力効果については,弱魅力効果についてのみ,妥協効果と同じネスティング構造を持つ GNL モデルで表現可能であることが分かった.
ただし,この構造で表現可能な魅力効果の大きさは小さいものである.

GNL モデルは NL モデルを内包するため,GNL モデルでは,同じネスティング構造で,類似性効果,妥協効果,魅力効果全ての心理的効果を表現可能である.
このことから,規範的なモデルといえる GNL モデルが,心理的効果を全て表現できるという記述的側面からも,妥当であるということがいえるだろう.
GNL モデルは非集計モデルであり,対数尤度最大化により,容易に実際の POS データからパラメータ推定が可能である.
この事実は,ほとんど従来不可能であった実際の購買データからの心理的効果の測定が技術的に可能になったことを意味する.
従来研究の結果はそのほとんどがアンケートによるものであり,過度に抽象化された環境のもとでの結果であると筆者らは考えている.

今後の課題としては,今回取り上げることがなかった,幻効果等のその他の心理的効果の効用最大化と整合的な表現,非集計モデルにおける表現が挙げられる.
また,今回開発した非集計的な側面を活用した,POS データ上での実際のミクロなマーケット・データからの心理的効果の検証も必要となるだろう.
これらは別の機会に報告したい.
%%%%%%%%%%%%%%%%%%%%%%%%%%%%%%%%%%%%%%%%%%%%%%%%%%%%%%%%%%%%%%%%%%%%%%%%%%%%%%%%%%%%%%%%%%%%%%%%%%%%%%%%%%%%%%%%%%%%%%%%%%%%%%%%%%%%%%%%%%%%%%%%%%%%%%%%%%%%%%%%%%%%%%%%%%%%%%%%
\def\thesection{\thechapter.\Alph{section}}
\setcounter{section}{0} 
\section{付録}
\subsection{Roe et al. (2001) における妥協効果の定義}\label{sec:6-A-1}
Roe et al. (2001) \cite{RBT01} では,式({\ref{eq6-11})を
\begin{equation}
	Q_{\Ar | \left\{ \Ar ,\Br  \right\}}^h =Q_{\Ar | \left\{ \Ar ,\Cr \right\}}^h =Q_{\Br | \left\{ \Br ,\Cr \right\}}^h :=\check{Q}^h \label{eq6-54}
\end{equation}
のみとして,妥協効果を定義している.しかし,この定義では,$\check{Q}^h=0.3<0.5$ のような場合も許容している.
このような場合について式({\ref{eq6-12}),({\ref{eq6-13})と併せて考えてみると,単に選択肢 $\Cr$ が他の選択枝 $\Ar , \Br $ と比較し魅力的であることを述べていることとなる.つまり,提示された選択肢集合とは何ら関係がなくなっており,これはもはや妥協効果とは呼べない.
そのため,本研究では,強妥協効果の定義として Roe et al. (2001) \cite{RBT01} での定義を用いていない.

\subsection{最大化の一階条件}\label{sec:6-A-2}
証明\ref{pr02} と同様に効用関数の仮定及び弱妥協効果の選択肢 $\Ar ,\Br $ に関する対称性より,$\exp{V_{\Ar}^h}=\exp{V_{\Br}^h}=\exp{V_{\Cr}^h}:=Y^h$ を得る.
すると,弱妥協効果の大きさは,
\begin{align}
\delta_{\C}&=\frac{ \sum \limits_{j^{\prime}=1}^{3} \left( \left( \sum \limits_{k^{\prime} \in \K_{j}} \gamma_{k^{\prime} j} Y^h \right)^{\mu_{j}-1} \left( \gamma_{{\Cr} j} Y^h  \right)^{1/\mu_{j}} \right)}
{ \sum \limits_{j=1}^{3} \left( \left( \sum \limits_{k^{\prime} \in \K_{j}} \gamma_{k^{\prime}} Y^h \right)^{\mu_{j}-1} \left( \gamma_{{\Cr} j} Y^h \right)^{1/\mu_j} +
\left( \sum \limits_{k^{\prime} \in \K_{j}} \gamma_{k^{\prime} j} Y \right)^{\mu_{j}-1} \left( \gamma_{{\Cr} j} Y^h \right)^{1/\mu_{j}} \right)}\notag\\
&-\sum \limits_{j=1}^{3} \left( \frac{ \left( \sum \limits_{k^{\prime} \in \K_{j}} \left( \gamma_{k^{\prime}j} Y^h \right)^{1/\mu_j} \right)^{\mu_j}}{\sum \limits_{j^{\prime}=1}^{3} \left( \sum \limits_{k^{\prime} \in \K_{j^{\prime}}} \left( \gamma_{k^{\prime}j^{\prime}} Y^h \right)^{1/\mu_{j^{\prime}}} \right)^{\mu_{j^{\prime}}}} \frac{ \left( \gamma_{{\Cr} j} Y^h \right)^{1/\mu_j}}{\sum \limits_{j^{\prime}=1}^{3} \left( \gamma_{k^{\prime}j^{\prime}} Y^h \right)^{1/\mu_{j^{\prime}}}} \right) \label{eq6-55}
\end{align}\begin{table} [tb:AA]
\begin{center}
\caption{弱妥協効果が最大となるパラメータ}  \label{tb:6-5}
\begin{tabular} {ccp{0.8zw}rp{4.2zw}ccp{0.8zw}r}
\toprule
\multicolumn{2}{c}{Variable}&&&&\multicolumn{2}{c}{Variable}&&\\
\hline\hline
\multirow{3}{*}{Allocation Parameter}&$\gamma_1$&&$0.5$&&\multirow{2}{*}{Similarity Parameter}&$\mu$&&$0_+$ \footnotemark\\
&$\gamma_2$&&$0.0$&&&$\mu_2$&&不定\footnotemark \\
&$\gamma_{\Cr}$&&$1.0$&&&&&\\
\bottomrule
\end{tabular}
\end{center}
\end{table}
\footnotetext[2]{$0$ではない,$0$に最も近い正の実数を表わす.}
\footnotetext{今回は$\gamma_2=0.0$,$\gamma_{\Cr}=1.0$ であるため,ネスト $2$ に所属する選択肢が $\Cr$ のみであり,不定となる.}
\normalsize

ここで対称性より,
\begin{align}
\gamma_{\Ar 1}&=\gamma_{\Br 3}:=\gamma_1,\label{eq6-56}\\
\gamma_{\Ar 2}&=\gamma_{\Br 2}:=\gamma_2,\label{eq6-57}\\
\gamma_{\Ar 3}&=\gamma_{\Br 1}:=\gamma_3=1-\gamma_1-\gamma_2,\label{eq6-58}\\
\gamma_{\Cr 1}&=\gamma_{\Cr 3}:=\gamma_{\Cr},\label{eq6-59}\\
\gamma_{\Cr 2}&=1-2\gamma_{\Cr},\label{eq6-60}\\
\mu_1 &=\mu_3:=\mu \label{eq6-61}
\end{align}
 とする.ここで,制約条件 \eqref{eq2-34}--\eqref{eq2-36}を考慮したラグランジアン $\Lambda$ を以下のようにたてる:
\begin{align}
\Lambda (\gamma_1,\gamma_2,\gamma_{\Cr}, \mu,\mu_2,\lambda_{j1},\lambda_{j2},\lambda_{kj}) :=& \delta_{\C}(\gamma_1,\gamma_2,\gamma_{\Cr},\mu,\mu_2) \notag\\
- &\sum \limits_{j=1}^{3} \lambda_{j1} \mu_j - \sum \limits_{j=1}^{3} \lambda_{j2} \left( \mu_j -1 \right) - \sum_{j=1}^{3} \sum_{k^{\prime} \in j} \lambda_{k^{\prime}j} \gamma_{k^{\prime}j} \label{eq6-62}
\end{align}
ここで,$\lambda_{j1}$,$\lambda_{j2}$,$\lambda_{kj}$ はそれぞれ制約条件 \eqref{eq2-34}の下限,上限,\eqref{eq2-36}に対応するラグランジェ乗数である.
なお,式\eqref{eq2-35}については考慮済であるため含めない.また,GEV モデルの定義より,絶対的な確定的効用の値は選択確率に影響を及ぼさないため,$Y$ は変数とならない.
ここで,式\eqref{eq6-62}の KKT 条件より,最大値 $\overline{\delta}_{\C}$ を求める.すると,表 \ref{tb:6-5}に示すパラメータのとき,最大値
\begin{align}
\overline{\delta}_{\C} = \frac{1}{6} \label{eq6-63}
\end{align}
が求まる.