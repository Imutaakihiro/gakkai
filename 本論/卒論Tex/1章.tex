\chapter{はじめに}
\setcounter{page}{1}
\pagenumbering{arabic}
%%%%%%%%%%%%%%%%%%%%%%%%%%%%%%%%%%%%%%%%%%%%%%%%%%%%%%%%%%%%%%%%%%%%%%%%%%%%%%%
\section{背景}
%%%%%%%%%%%%%%%%%%%%%%%%%%%%%%%%%%%%%%%%%%%%%%%%%%%%%%%%%%%%%%%%%%%%%%%%%%%%%%%
デジタル技術の急速な進化により,私たちの日常生活は膨大なデータとともにある時代へと突入している.特に,インターネットの普及やスマートフォンの発展に伴い,個人の行動履歴や購買履歴,さらにはオンライン上での嗜好情報などがデータとして収集・蓄積されるようになっている.これらのデータは,マーケティング,医療,公共政策など,さまざまな分野において有益な活用が進められており,データ駆動型社会の発展に寄与している.

しかしその一方で,個人情報の取り扱いに関する懸念が急速に高まっている.具体的には,個人データがどのように収集・管理され,どの程度の安全性が確保されているのかが不透明であることが問題視されている.近年,データ漏洩や不適切な利用が社会問題として頻繁に取り上げられており,プライバシー保護はますます重要な社会的課題となっている.

このような状況に対応するため,データ匿名化技術が注目を集めている.データ匿名化技術は,個人情報を保護しながらデータの利便性を維持するための手段として,多様な手法が提案されている.その中でも,$k$匿名化は特に広く利用されている技術の一つである.例として,医療分野においては,患者の病歴や診断データを匿名化することで,研究者が貴重な医療情報を活用できるようになりつつも,患者のプライバシーが保護される.また,消費者行動分析においても,匿名化されたデータを利用することで,消費者の購買傾向や嗜好を把握しながら,個人を特定するリスクを低減することが可能となる.

一方で,マーケティングや行動経済学の分野では,消費者の選択行動を理解するためにブランド選択モデルが広く利用されている.ブランド選択モデルは,消費者が商品やサービスを選択する際の意思決定プロセスを数理的に表現し,効果的な広告戦略や販売促進を行うための基盤となる.近年では,機械学習や統計解析を活用した高度なブランド選択モデルが開発され,消費者の購買行動の予測精度が向上している.

しかし,これらのモデルを高精度で運用するためには,消費者個人の詳細なデータが必要不可欠である.この点が,プライバシー保護の観点から重要な課題となっている.具体的には,個人情報をそのまま利用することによる倫理的な問題や,法規制への抵触の可能性が懸念される.たとえば,消費者の詳細な行動履歴や購買履歴を無制限に活用することは,消費者に不安を与え,企業への信頼を損なう要因となり得る.

このように,プライバシー保護とデータ活用の両立は,現代社会において解決すべき重要な課題として浮かび上がっている.本研究は,この課題に対する具体的な解決策を模索し,$k$匿名化を活用した選択モデルの安全性と有用性のバランスを検討することを目的とする.
%%%%%%%%%%%%%%%%%%%%%%%%%%%%%%%%%%%%%%%%%%%%%%%%%%%%%%%%%%%%%%%%%%%%%%%%%%%%%%%
\section{研究の目的}
%%%%%%%%%%%%%%%%%%%%%%%%%%%%%%%%%%%%%%%%%%%%%%%%%%%%%%%%%%%%%%%%%%%%%%%%%%%%%%%
本研究の目的は,個人情報保護のための匿名化技術である$k$匿名化をブランド選択モデルに適用することにより,プライバシー保護とデータの有用性をどのように両立できるかを明らかにすることである.近年,消費者データの活用が進む中で,個人のプライバシーを適切に保護しながら,データの分析精度を維持することが求められている.しかし,匿名化を施すことでデータの細部が失われ,分析の精度に影響を与える可能性がある.本研究では,このトレードオフの特性を詳細に検討し,最適なバランスを探る.

具体的には,$k$匿名化を適用することによるデータの変化が,分析結果やモデルの精度に与える影響を定量的に評価する.まず,$k$匿名化を適用したデータセットと,適用前のデータセットを比較し,匿名化がデータの統計的特性に及ぼす影響を詳細に検証する.この評価においては,異なる$k$の値を設定し,それぞれの設定においてプライバシー保護の強度とデータ分析の精度にどのような変化が生じるかを確認する.

さらに,ブランド選択モデルとしてGNL(Generalized Nested Logit) モデルを採用し,$k$匿名化がモデルの推定精度やパラメータの推定値に与える影響を詳細に分析する.GNLモデルは,消費者の選択行動を説明するための強力な手法であり,選択肢間の相関を考慮できる点で広く活用されている.本研究では,匿名化がGNLモデルの識別性能や予測精度にどのような影響を及ぼすかを検討する.

また,$k$匿名化の適用によってブランド選択モデルの結果がどのように変化するのかを,消費者行動のパターンや選択確率の変動を通じて詳細に評価する.この分析を通じて,匿名化がブランド選択の意思決定プロセスに及ぼす影響を解明し,データの一般化や情報の統合がモデルの精度にどのような影響を与えるのかを明らかにする.
%%%%%%%%%%%%%%%%%%%%%%%%%%%%%%%%%%%%%%%%%%%%%%%%%%%%%%%%%%%%%%%%%%%%%%%%%%%%%%%
\section{本研究の構成}
本研究は, 全$5$章からなる. 
次章以降は次のように構成されている. 

第2章では,$k$匿名化の仕組みとその重要性について詳しく説明する.
第3章では,GNLモデルにおける選択確率や構造について概説し,集計ルールが満たすべき条件を整理するとともに,ネストと選択肢ごとの集計ルールについて詳述する.
第4章では,$k$匿名化の適用前後における最尤推定の結果を比較し,確定的効用の範囲に与える影響を評価する.
最後に,第5章では本研究の結論をまとめるとともに,今後の課題について述べる.
%%%%%%%%%%%%%%%%%%%%%%%%%%%%%%%%%%%%%%%%%%%%%%%%%%%%%%%%%%%%%%%%%%%%%%%%%%%%%%%
