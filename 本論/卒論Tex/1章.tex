\chapter{はじめに}
\setcounter{page}{1}
\pagenumbering{arabic}

%%%%%%%%%%%%%%%%%%%%%%%%%%%%%%%%%%%%%%%%%%%%%%%%%%%%%%%%%%%%%%%%%%%%%%%%%%%%%%%
\section{研究背景}
%%%%%%%%%%%%%%%%%%%%%%%%%%%%%%%%%%%%%%%%%%%%%%%%%%%%%%%%%%%%%%%%%%%%%%%%%%%%%%%

\subsection{高等教育における授業評価の位置づけ}
高等教育では教育の質保証が重要な課題であり,学生による授業評価(Student Evaluation of Teaching: SET)は教育改善の基盤として広く実施されている\cite{marsh2007,spooren2013}.授業評価は教員へのフィードバックや全学的な教育改善に用いられる.

授業評価アンケートは多段階の評価スコアと自由記述から構成されることが多い.評価スコアは授業間比較や経年変化の把握に適している一方,自由記述は評価理由や具体的な要望を把握できる.

\subsection{自由記述分析と感情分析の課題}
自由記述は非構造データであり,学期ごとに大量に収集されるため,人手による読解だけで全体傾向を把握することは困難である.また,読解者の主観により解釈がばらつく可能性がある.

教育データマイニングや学習分析の文脈では,テキスト分析を含む大規模データの自動分析が進められている\cite{romero2020}.感情分析はテキストに含まれる肯定的・否定的・中立的な感情を推定する技術であり\cite{liu2012},自由記述を数値化して評価スコアとの関係を検討する手段として有用である.

\subsection{本研究の対象データ}
本研究では,福岡工業大学における2018年度から2024年度までの7年間の授業評価データを分析対象とする.対象データの規模を表\ref{tab:data_intro}に示す.

\begin{table}[t]
    \centering
    \caption{本研究の対象データ}
    \label{tab:data_intro}
    \resizebox{0.75\textwidth}{!}{
    \begin{tabular}{l r}
        \toprule
        項目 & 値 \\
        \midrule
        対象期間 & 2018年度〜2024年度(7年間) \\
        対象学科数 & 9学科 \\
        授業数 & 3,268件 \\
        自由記述総件数 & 83,851件 \\
        平均自由記述数/授業 & 25.2件 \\
        \bottomrule
    \end{tabular}
    }
\end{table}

本研究の規模は,統計的分析と機械学習モデルの検討に十分な量である.

%%%%%%%%%%%%%%%%%%%%%%%%%%%%%%%%%%%%%%%%%%%%%%%%%%%%%%%%%%%%%%%%%%%%%%%%%%%%%%%
\section{課題の整理}
%%%%%%%%%%%%%%%%%%%%%%%%%%%%%%%%%%%%%%%%%%%%%%%%%%%%%%%%%%%%%%%%%%%%%%%%%%%%%%%

授業評価の活用には以下の課題がある.第一に,評価スコアは総合判断であり,どの要因がどの程度影響したかを直接把握できない.第二に,自由記述は非構造データであり,83,851件の記述を人手で一貫して分析することは現実的ではない.第三に,限られた教育改善資源の中で,改善優先度を客観的に判断するための定量的根拠が不足している.

%%%%%%%%%%%%%%%%%%%%%%%%%%%%%%%%%%%%%%%%%%%%%%%%%%%%%%%%%%%%%%%%%%%%%%%%%%%%%%%
\section{研究目的}
%%%%%%%%%%%%%%%%%%%%%%%%%%%%%%%%%%%%%%%%%%%%%%%%%%%%%%%%%%%%%%%%%%%%%%%%%%%%%%%

本研究の目的は,授業評価アンケートの自由記述から感情スコアを推定し,授業評価スコアとの関係性を分析することで,授業評価に影響する要因を定量的に特定することである.

具体的には,以下の3点を目的とする.

\begin{enumerate}
\item \textbf{関係性の把握}: 自由記述の感情スコアと授業評価スコアの関係を統計的に検討する.
\item \textbf{共通要因と特化要因の分離}: 感情スコアと授業評価スコアを同時に予測するモデルにより,共通要因とタスク特化要因を分離する.
\item \textbf{要因の定量化}: SHAP分析により語彙レベルの寄与度を可視化し,改善の示唆を得る.
\end{enumerate}

%%%%%%%%%%%%%%%%%%%%%%%%%%%%%%%%%%%%%%%%%%%%%%%%%%%%%%%%%%%%%%%%%%%%%%%%%%%%%%%
\section{研究仮説}
%%%%%%%%%%%%%%%%%%%%%%%%%%%%%%%%%%%%%%%%%%%%%%%%%%%%%%%%%%%%%%%%%%%%%%%%%%%%%%%

本研究では以下の仮説を設定する.

\textbf{仮説1}: 授業単位で集約した感情スコアと授業評価スコアには正の相関関係がある.

\textbf{仮説2}: 感情スコアと授業評価スコアの両方に影響する共通要因が存在する.

\textbf{仮説3}: マルチタスク学習により,共通要因と特化要因を分離できる.

%%%%%%%%%%%%%%%%%%%%%%%%%%%%%%%%%%%%%%%%%%%%%%%%%%%%%%%%%%%%%%%%%%%%%%%%%%%%%%%
\section{研究のアプローチと特徴}
%%%%%%%%%%%%%%%%%%%%%%%%%%%%%%%%%%%%%%%%%%%%%%%%%%%%%%%%%%%%%%%%%%%%%%%%%%%%%%%

本研究では,日本語の事前学習済みBERTを基盤とした感情分類モデルを構築し\cite{bert},83,851件の自由記述から感情スコアを推定する.さらに,感情スコアと授業評価スコアを同時に予測するマルチタスク学習モデルを構築し\cite{mtl},両タスクに共通する特徴と各タスクに固有の特徴の分離を図る.モデル解釈にはSHAP分析を用い\cite{shap},語彙レベルの寄与度を定量化することで,授業改善に直結しうる要因の抽出を目指す.

%%%%%%%%%%%%%%%%%%%%%%%%%%%%%%%%%%%%%%%%%%%%%%%%%%%%%%%%%%%%%%%%%%%%%%%%%%%%%%%
\section{研究の意義}
%%%%%%%%%%%%%%%%%%%%%%%%%%%%%%%%%%%%%%%%%%%%%%%%%%%%%%%%%%%%%%%%%%%%%%%%%%%%%%%

本研究の学術的意義は,感情と評価を統合的に扱う分析枠組みを提示し,授業評価における要因構造の理解を深める点にある.マルチタスク学習とSHAP分析を組み合わせることで,複数指標を同時に解釈可能な形で扱えることを示す.

実践的意義として,大規模自由記述の自動分析により,授業改善の優先順位付けに資する定量的根拠を提供できる点が挙げられる.

%%%%%%%%%%%%%%%%%%%%%%%%%%%%%%%%%%%%%%%%%%%%%%%%%%%%%%%%%%%%%%%%%%%%%%%%%%%%%%%
\section{本研究の構成}
%%%%%%%%%%%%%%%%%%%%%%%%%%%%%%%%%%%%%%%%%%%%%%%%%%%%%%%%%%%%%%%%%%%%%%%%%%%%%%%

本研究は全5章からなる.第2章では関連研究を整理し,本研究の位置づけを明確にする.第3章ではデータセット,前処理,モデル構成,評価指標を説明する.第4章では実験結果と考察を示す.第5章では結論と今後の課題を述べる.

%%%%%%%%%%%%%%%%%%%%%%%%%%%%%%%%%%%%%%%%%%%%%%%%%%%%%%%%%%%%%%%%%%%%%%%%%%%%%%%
