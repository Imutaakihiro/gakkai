\chapter{はじめに}
\setcounter{page}{1}
\pagenumbering{arabic}
%%%%%%%%%%%%%%%%%%%%%%%%%%%%%%%%%%%%%%%%%%%%%%%%%%%%%%%%%%%%%%%%%%%%%%%%%%%%%%%
\section{研究背景}
%%%%%%%%%%%%%%%%%%%%%%%%%%%%%%%%%%%%%%%%%%%%%%%%%%%%%%%%%%%%%%%%%%%%%%%%%%%%%%%
高等教育機関では,教育の質向上を目的として学生による授業評価が広く実施されている.授業評価アンケートは,多段階の評価スコアと自由記述から構成されることが多く,評価スコアは定量的な比較に適している.一方で,授業の良否を左右する要因や学生の本音は自由記述に表れることが多く,数値評価のみでは十分に把握できない.

近年,自然言語処理技術と機械学習の発展により,大規模な自由記述の分析が可能になってきた.自由記述を感情スコアとして数値化し,授業評価スコアとの関係を統計的に検討することで,授業改善に有用な知見を得られる可能性がある.

%%%%%%%%%%%%%%%%%%%%%%%%%%%%%%%%%%%%%%%%%%%%%%%%%%%%%%%%%%%%%%%%%%%%%%%%%%%%%%%
\section{課題の整理}
%%%%%%%%%%%%%%%%%%%%%%%%%%%%%%%%%%%%%%%%%%%%%%%%%%%%%%%%%%%%%%%%%%%%%%%%%%%%%%%
授業評価の活用には,以下の課題が存在する.第一に,授業評価スコアは複数要因の総合判断であり,どの要因がどの程度影響したかが直接観測できない点である.第二に,自由記述は非構造データであるため,人的読解に依存すると大規模データの分析が困難である.第三に,授業評価スコアのみでは学生の感情や満足度の差異を捉えにくく,教育改善に結び付く具体的示唆が得られにくい点が挙げられる.

%%%%%%%%%%%%%%%%%%%%%%%%%%%%%%%%%%%%%%%%%%%%%%%%%%%%%%%%%%%%%%%%%%%%%%%%%%%%%%%
\section{研究目的}
%%%%%%%%%%%%%%%%%%%%%%%%%%%%%%%%%%%%%%%%%%%%%%%%%%%%%%%%%%%%%%%%%%%%%%%%%%%%%%%
本研究の目的は,授業評価アンケートの自由記述から感情スコアを推定し,授業評価スコアとの関係性を分析することで,授業評価に影響する要因を定量的に特定することである.

具体的には,(1) 自由記述の感情分析により感情スコアを算出し,授業単位で集計した感情スコアと授業評価スコアの関係を検討する,(2) 感情スコアと評価スコアを同時に予測するモデルを構築し,共通要因(満足度要因)とそれぞれに特有の要因を分離する,(3) 解釈可能性手法を用いて要因を定量化し,授業改善に資する知見を得る,という流れで目的の達成を図る.

%%%%%%%%%%%%%%%%%%%%%%%%%%%%%%%%%%%%%%%%%%%%%%%%%%%%%%%%%%%%%%%%%%%%%%%%%%%%%%%
\section{研究仮説}
%%%%%%%%%%%%%%%%%%%%%%%%%%%%%%%%%%%%%%%%%%%%%%%%%%%%%%%%%%%%%%%%%%%%%%%%%%%%%%%
本研究では,以下の仮説を設定する.
\begin{enumerate}
\item 授業単位で集約した感情スコアと授業評価スコアには正の相関関係がある.
\item 感情スコアと授業評価スコアの両方に影響する共通要因(満足度要因)が存在する.
\item マルチタスク学習により,共通要因と特化要因を分離できる.
\end{enumerate}

%%%%%%%%%%%%%%%%%%%%%%%%%%%%%%%%%%%%%%%%%%%%%%%%%%%%%%%%%%%%%%%%%%%%%%%%%%%%%%%
\section{研究のアプローチと特徴}
%%%%%%%%%%%%%%%%%%%%%%%%%%%%%%%%%%%%%%%%%%%%%%%%%%%%%%%%%%%%%%%%%%%%%%%%%%%%%%%
本研究では,日本語の事前学習済みBERTモデルを基盤とした感情分類モデルを構築し,授業評価アンケートの自由記述から感情スコアを推定する.推定した感情スコアは授業単位で集計し,授業評価スコアとの相関関係を統計的に検討する.

さらに,感情スコアと授業評価スコアを同時に予測するマルチタスク学習モデルを構築し,両者に共通する要因と特化要因を分離する.加えて,SHAP分析を用いて単語レベルの寄与度を定量化し,授業改善に直結する要因の抽出を行う.

%%%%%%%%%%%%%%%%%%%%%%%%%%%%%%%%%%%%%%%%%%%%%%%%%%%%%%%%%%%%%%%%%%%%%%%%%%%%%%%
\section{研究の意義}
%%%%%%%%%%%%%%%%%%%%%%%%%%%%%%%%%%%%%%%%%%%%%%%%%%%%%%%%%%%%%%%%%%%%%%%%%%%%%%%
本研究は,自由記述を感情スコアとして定量化し,授業評価スコアとの関係を明確化する点に意義がある.また,マルチタスク学習とSHAP分析を組み合わせることで,単なる相関の確認にとどまらず,授業改善に資する具体的な要因を抽出できる.これにより,教育改善の意思決定をデータに基づいて行うための基盤を提供する.

%%%%%%%%%%%%%%%%%%%%%%%%%%%%%%%%%%%%%%%%%%%%%%%%%%%%%%%%%%%%%%%%%%%%%%%%%%%%%%%
\section{本研究の構成}
本研究は,全5章からなる.第2章では,授業評価研究,自然言語処理による感情分析,マルチタスク学習,解釈可能AI,順序回帰に関する関連研究を整理する.第3章では,データセットと前処理,感情分類モデル,マルチタスク学習,順序回帰モデル,および評価方法を述べる.第4章では,感情スコアと授業評価スコアの関係性分析,モデルの推定結果,SHAP分析による要因抽出の結果を示す.第5章では,本研究のまとめと今後の課題を述べる.
%%%%%%%%%%%%%%%%%%%%%%%%%%%%%%%%%%%%%%%%%%%%%%%%%%%%%%%%%%%%%%%%%%%%%%%%%%%%%%%
