%-----------------------------
\chapter{結果と考察}
%-----------------------------

本章では,授業評価アンケートの基礎統計量,感情分類モデルの性能,相関分析結果,SHAP分析による要因抽出結果を示し,得られた知見を考察する.

%%%%%%%%%%%%%%%%%%%%%%%%%%%%%%%%%%%%%%%%%%%%%%%%%%%%%%%%%%%%%%%%%%%%%%%%%%%%%%%
\section{基礎統計量}
%%%%%%%%%%%%%%%%%%%%%%%%%%%%%%%%%%%%%%%%%%%%%%%%%%%%%%%%%%%%%%%%%%%%%%%%%%%%%%%

\subsection{感情スコアと授業評価スコアの分布}
感情スコアと授業評価スコアの基本統計量を表\ref{tab:basicstats}に示す.感情スコアは授業単位で集約した値であり,−1(ネガティブ)から+1(ポジティブ)の範囲をとる.授業評価スコアは4段階(1〜4点)である.

感情スコアは平均0.001でほぼニュートラルに近く,標準偏差は0.260である.これは,多くの自由記述がニュートラルに分類されること,および授業単位で集約することでポジティブ・ネガティブが相殺されることを反映している.

授業評価スコアは平均3.459点(4点満点)で,標準偏差は0.216と小さい.第1四分位数が3.330,第3四分位数が3.600であり,多くの授業が3点台後半に集中していることが分かる.最小値が2.000であることから,極端に低い評価の授業は少ないことが示される.

\begin{table}[t]
    \centering
    \caption{感情スコアと授業評価スコアの基本統計量}
    \label{tab:basicstats}
    \resizebox{0.7\textwidth}{!}{
    \begin{tabular}{l r r}
        \toprule
        統計量 & 感情スコア & 授業評価スコア \\
        \midrule
        平均 & 0.001 & 3.459 \\
        標準偏差 & 0.260 & 0.216 \\
        最小値 & −1.000 & 2.000 \\
        第1四分位数(Q1) & −0.167 & 3.330 \\
        中央値(Q2) & 0.000 & 3.480 \\
        第3四分位数(Q3) & 0.167 & 3.600 \\
        最大値 & 1.000 & 4.000 \\
        \bottomrule
    \end{tabular}
    }
\end{table}

教師データのラベル分布を表\ref{tab:labeldist}に示す.ニュートラルが628件(62.8\%)と大半を占める一方,ネガティブは191件(19.1\%),ポジティブは180件(18.0\%)と少数である.

このクラス不均衡は,多くの学生が事実の記述や中立的な感想を述べる傾向にあることを反映している.また,日本語の自由記述では明確な感情表現を避ける傾向があることも一因と考えられる.

\begin{table}[t]
    \centering
    \caption{教師データのラベル分布(1,000件)}
    \label{tab:labeldist}
    \resizebox{0.6\textwidth}{!}{
    \begin{tabular}{l r r}
        \toprule
        ラベル & 件数 & 割合 \\
        \midrule
        ネガティブ & 191 & 19.1\% \\
        ニュートラル & 628 & 62.8\% \\
        ポジティブ & 180 & 18.0\% \\
        \midrule
        合計 & 1,000 & 100.0\% \\
        \bottomrule
    \end{tabular}
    }
\end{table}

%%%%%%%%%%%%%%%%%%%%%%%%%%%%%%%%%%%%%%%%%%%%%%%%%%%%%%%%%%%%%%%%%%%%%%%%%%%%%%%
\section{感情分類モデルの性能}
%%%%%%%%%%%%%%%%%%%%%%%%%%%%%%%%%%%%%%%%%%%%%%%%%%%%%%%%%%%%%%%%%%%%%%%%%%%%%%%

\subsection{性能評価結果と考察}
感情分類モデルの性能指標を表\ref{tab:perf}に示す.検証データ200件に対して,正解率77.0\%,マクロ平均F1スコア0.706,重み付き平均F1スコア0.770を達成した.

\begin{table}[t]
    \centering
    \caption{感情分類モデルの性能指標(検証データ200件)}
    \label{tab:perf}
    \resizebox{0.7\textwidth}{!}{
    \begin{tabular}{l r}
        \toprule
        指標 & 値 \\
        \midrule
        正解率(Accuracy) & 0.770 \\
        マクロ平均適合率(Precision) & 0.707 \\
        マクロ平均再現率(Recall) & 0.705 \\
        マクロ平均F1スコア & 0.706 \\
        重み付き平均F1スコア & 0.770 \\
        \bottomrule
    \end{tabular}
    }
\end{table}

正解率77.0\%は,教育分野の自由記述に対する感情分析として実用的な水準であると考えられる.マクロ平均F1スコア(0.706)と重み付き平均F1スコア(0.770)の差は,クラス間の性能差を反映している.

クラス別の性能指標を表\ref{tab:perf_class}に示す.ニュートラルクラスが最も高い性能(F1スコア0.833)を示す一方,ネガティブ(0.667)およびポジティブ(0.618)は相対的に低い.

\begin{table}[t]
    \centering
    \caption{クラス別の性能指標}
    \label{tab:perf_class}
    \resizebox{0.75\textwidth}{!}{
    \begin{tabular}{l r r r r}
        \toprule
        クラス & 適合率 & 再現率 & F1スコア & サポート \\
        \midrule
        ネガティブ & 0.659 & 0.675 & 0.667 & 40 \\
        ニュートラル & 0.833 & 0.833 & 0.833 & 132 \\
        ポジティブ & 0.630 & 0.607 & 0.618 & 28 \\
        \bottomrule
    \end{tabular}
    }
\end{table}

ニュートラルクラスの高い性能は,サンプル数が多いこと(132件),および中立的な記述のパターンが比較的安定していることによると考えられる.一方,ネガティブおよびポジティブクラスはサンプル数が少なく(40件,28件),表現の多様性も高いため,分類が困難であると考えられる.

混同行列を表\ref{tab:confusion}に示す.主な誤分類パターンは以下の通りである.

\begin{table}[t]
    \centering
    \caption{混同行列}
    \label{tab:confusion}
    \resizebox{0.6\textwidth}{!}{
    \begin{tabular}{l r r r}
        \toprule
        & \multicolumn{3}{c}{予測} \\
        \cmidrule(lr){2-4}
        実際 & ネガティブ & ニュートラル & ポジティブ \\
        \midrule
        ネガティブ & 27 & 12 & 1 \\
        ニュートラル & 13 & 110 & 9 \\
        ポジティブ & 1 & 10 & 17 \\
        \bottomrule
    \end{tabular}
    }
\end{table}

第一に,ネガティブとニュートラルの混同が多い(ネガティブ→ニュートラル12件,ニュートラル→ネガティブ13件).これは,批判や不満を婉曲的に表現する日本語の特性により,ネガティブな感情がニュートラルに見える場合があることを示唆する.

第二に,ポジティブとニュートラルの混同も見られる(ポジティブ→ニュートラル10件,ニュートラル→ポジティブ9件).事実の記述に肯定的なニュアンスが含まれる場合や,控えめな肯定表現がニュートラルと判定される場合があると考えられる.

第三に,ネガティブとポジティブの直接的な混同は少ない(ネガティブ→ポジティブ1件,ポジティブ→ネガティブ1件).これは,明確な極性を持つ表現は正しく分類されやすいことを示す.


%%%%%%%%%%%%%%%%%%%%%%%%%%%%%%%%%%%%%%%%%%%%%%%%%%%%%%%%%%%%%%%%%%%%%%%%%%%%%%%
\section{感情スコアと授業評価スコアの相関分析}
%%%%%%%%%%%%%%%%%%%%%%%%%%%%%%%%%%%%%%%%%%%%%%%%%%%%%%%%%%%%%%%%%%%%%%%%%%%%%%%

\subsection{相関分析の結果と解釈}
授業単位で集約した感情スコアと授業評価スコアの相関分析結果を表\ref{tab:correlation}に示す.3,268授業を対象に分析を行った.

\begin{table}[t]
    \centering
    \caption{感情スコアと授業評価スコアの相関分析結果(N=3,268)}
    \label{tab:correlation}
    \resizebox{0.75\textwidth}{!}{
    \begin{tabular}{l r r l}
        \toprule
        指標 & 相関係数 & $p$値 & 解釈 \\
        \midrule
        ピアソン相関係数 & 0.3097 & $<0.000001$ & 中程度の正の相関 \\
        スピアマン順位相関係数 & 0.2970 & $<0.000001$ & 中程度の正の相関 \\
        ケンドール順位相関係数 & 0.2042 & $<0.000001$ & 弱〜中程度の正の相関 \\
        \bottomrule
    \end{tabular}
    }
\end{table}

ピアソン相関係数は0.3097($p<0.000001$)であり,中程度の正の相関が確認された.スピアマン順位相関係数(0.2970)およびケンドール順位相関係数(0.2042)においても同様に統計的に有意な正の相関が得られた.

感情スコアと授業評価スコアの散布図を図\ref{fig:correlation}に示す.

\begin{figure}[t]
    \centering
    \includegraphics[width=0.7\textwidth]{fig/correlation_scatter.png}
    \caption{感情スコアと授業評価スコアの散布図(N=3,268)}
    \label{fig:correlation}
\end{figure}

相関係数0.3097は,社会科学の分野では「中程度」の相関として解釈される.この結果は,以下の点を示唆する.

第一に,授業レベルで集約した感情スコアと授業評価スコアには一定の関係があることが確認された.自由記述にポジティブな感情が多く表れる授業は,評価スコアも高い傾向がある.

第二に,相関係数が1に近くないことは,感情スコアと評価スコアが同一の概念を測定しているわけではないことを示す.評価スコアには感情以外の要因(授業の有用性,学習成果など)も影響していると考えられる.

第三に,複数の相関指標で一貫した結果が得られたことは,この関係が頑健であることを示す.ピアソン相関係数は線形関係を,順位相関係数は単調関係を捉えるため,両者で同様の結果が得られたことは,感情スコアと評価スコアの関係が線形的かつ単調であることを示唆する.

分析単位の違いについて補足する.
本研究では授業レベルでの相関(0.3097)を分析したが,個人レベル(個々の自由記述と授業評価スコア)での相関は約0.12と弱いことが予備分析で確認されている.

この差は,授業レベルでの集約により個人差がキャンセルされ,授業の「真の」特性がより明確に表れることによると考えられる.個人の感情は様々な要因に左右されるが,多数の学生の感情を平均することで,授業そのものの特性を反映した指標となる.

このことは,授業レベルでの分析が教育改善にとって有効であることを示唆する.

%%%%%%%%%%%%%%%%%%%%%%%%%%%%%%%%%%%%%%%%%%%%%%%%%%%%%%%%%%%%%%%%%%%%%%%%%%%%%%%
\section{単一タスクモデルのSHAP分析}
%%%%%%%%%%%%%%%%%%%%%%%%%%%%%%%%%%%%%%%%%%%%%%%%%%%%%%%%%%%%%%%%%%%%%%%%%%%%%%%

\subsection{ポジティブ・ネガティブ判定に寄与する重要語}
感情分類モデルに対するSHAP分析を行い,ポジティブ判定およびネガティブ判定に寄与する重要語を抽出した.5,000件のサンプル(ポジティブ2,500件,ネガティブ2,500件)を層化サンプリングし,出現回数5回以上の1,564語を分析対象とした.語彙はサブワード単位であるため,ここでは重要語例として提示する.

上位20語を表\ref{tab:shap_single}に示す.

\begin{table}[t]
    \centering
    \caption{ポジティブ判定に寄与する重要語例(TOP20)}
    \label{tab:shap_single}
    \resizebox{0.7\textwidth}{!}{
    \begin{tabular}{r l r r}
        \toprule
        順位 & 単語 & 平均SHAP値 & 出現回数 \\
        \midrule
        1 & やす & 0.2660 & 337 \\
        2 & 良かっ & 0.2466 & 207 \\
        3 & おもしろ & 0.2438 & 10 \\
        4 & よかっ & 0.2251 & 195 \\
        5 & 面白 & 0.2178 & 100 \\
        6 & 楽しい & 0.1959 & 67 \\
        7 & 楽しめる & 0.1876 & 6 \\
        8 & ありが & 0.1760 & 19 \\
        9 & 楽し & 0.1642 & 192 \\
        10 & 面白い & 0.1518 & 37 \\
        11 & でき & 0.1254 & 996 \\
        12 & 出来 & 0.1147 & 237 \\
        13 & かっ & 0.1099 & 539 \\
        14 & 助 & 0.1044 & 41 \\
        15 & 達成 & 0.1035 & 9 \\
        16 & 学 & 0.0962 & 263 \\
        17 & きっかけ & 0.0957 & 20 \\
        18 & 好き & 0.0947 & 32 \\
        19 & 嬉 & 0.0943 & 29 \\
        20 & 充実 & 0.0916 & 20 \\
        \bottomrule
    \end{tabular}
    }
\end{table}

ネガティブ判定に寄与する重要語例(TOP20)を表\ref{tab:shap_negative}に示す.

\begin{table}[t]
    \centering
    \caption{ネガティブ判定に寄与する重要語例(TOP20)}
    \label{tab:shap_negative}
    \resizebox{0.7\textwidth}{!}{
    \begin{tabular}{r l r r}
        \toprule
        順位 & 単語 & 平均SHAP値 & 出現回数 \\
        \midrule
        1 & ほし & −0.0443 & 5 \\
        2 & ほう & −0.0425 & 98 \\
        3 & 大 & −0.0346 & 86 \\
        4 & まじ & −0.0314 & 5 \\
        5 & 難しかっ & −0.0311 & 88 \\
        6 & 直す & −0.0264 & 6 \\
        7 & ほしい & −0.0263 & 45 \\
        8 & 欲しい & −0.0247 & 33 \\
        9 & 奥 & −0.0219 & 8 \\
        10 & 器具 & −0.0211 & 7 \\
        11 & 真面目 & −0.0193 & 33 \\
        12 & 苦手 & −0.0191 & 98 \\
        13 & 程度 & −0.0187 & 41 \\
        14 & ください & −0.0183 & 365 \\
        15 & もう & −0.0172 & 35 \\
        16 & 穴 & −0.0160 & 11 \\
        17 & 期間 & −0.0159 & 6 \\
        18 & 不足 & −0.0159 & 16 \\
        19 & 油 & −0.0156 & 5 \\
        20 & 引き & −0.0155 & 7 \\
        \bottomrule
    \end{tabular}
    }
\end{table}

ポジティブ判定とネガティブ判定に寄与する重要語の可視化を図\ref{fig:shap}に示す.

\begin{figure}[t]
    \centering
    \begin{minipage}{0.48\textwidth}
        \centering
        \includegraphics[width=\textwidth]{fig/shap_positive.png}
        \subcaption{ポジティブ判定に寄与する重要語}
    \end{minipage}
    \hfill
    \begin{minipage}{0.48\textwidth}
        \centering
        \includegraphics[width=\textwidth]{fig/shap_negative.png}
        \subcaption{ネガティブ判定に寄与する重要語}
    \end{minipage}
    \caption{SHAP分析による重要語の可視化(TOP20)}
    \label{fig:shap}
\end{figure}

SHAP分析の重要語はサブワード断片を含むため,語彙単位の断定的解釈は控え,傾向として整理する.

\textbf{理解容易性に関わる表現}: 理解のしやすさを示す表現が上位に含まれる傾向が見られる.これは,授業内容の理解容易性が満足度に関係する可能性を示唆する.

\textbf{興味・関心に関わる表現}: 面白さ・楽しさに関わる表現が上位に含まれる傾向が見られる.これは,授業への興味・関心が感情評価に関係する可能性を示唆する.

\textbf{学習成果の実感に関わる表現}: 学習成果や達成感を示す表現が上位に含まれる傾向が確認される.ただし,これらの語彙解釈は文脈依存であるため,慎重な解釈が必要である.

\textbf{要望・改善要求に関わる表現}: 要望を示す表現がネガティブ判定の重要語に含まれる傾向が見られる.これらは不満の表出である可能性もあるため,授業改善の参考情報として扱う必要がある.

\textbf{困難さに関わる表現}: 困難さを示す表現がネガティブ判定に含まれる傾向が確認される.ただし,これらは授業の質そのものではなく,学生側の学習状況を反映している可能性がある.

%%%%%%%%%%%%%%%%%%%%%%%%%%%%%%%%%%%%%%%%%%%%%%%%%%%%%%%%%%%%%%%%%%%%%%%%%%%%%%%
\section{マルチタスク学習のSHAP分析}
%%%%%%%%%%%%%%%%%%%%%%%%%%%%%%%%%%%%%%%%%%%%%%%%%%%%%%%%%%%%%%%%%%%%%%%%%%%%%%%

\subsection{要因タイプ別の結果と解釈}
マルチタスクモデルのSHAP分析により,感情スコアと授業評価スコアの両方に影響する共通要因と,各タスクに特化した要因を分離した.語彙の分類結果を表\ref{tab:shap_types}に示す.

\begin{table}[t]
    \centering
    \caption{語彙の要因タイプ別内訳(3,198語)}
    \label{tab:shap_types}
    \resizebox{0.7\textwidth}{!}{
    \begin{tabular}{l r r l}
        \toprule
        要因タイプ & 語彙数 & 割合 & 特徴 \\
        \midrule
        共通要因(満足度) & 577 & 18.0\% & 両スコアに寄与 \\
        感情特化要因 & 1,200 & 37.5\% & 感情スコアのみに寄与 \\
        評価特化要因 & 532 & 16.6\% & 評価スコアのみに寄与 \\
        低重要度要因 & 889 & 27.8\% & 両スコアへの影響小 \\
        \midrule
        合計 & 3,198 & 100.0\% & — \\
        \bottomrule
    \end{tabular}
    }
\end{table}

全体3,198語のうち,共通要因は577語(18.0\%)であり,これらは感情スコアと評価スコアの双方に寄与する「満足度要因」と解釈できる.感情特化要因が最も多く1,200語(37.5\%)を占め,評価特化要因は532語(16.6\%)である.

共通要因の上位語彙例を表\ref{tab:shap_common}に示す.語彙はサブワード単位であるため,ここでは例示として提示する.

\begin{table}[t]
    \centering
    \caption{共通要因(満足度要因)の重要語例(TOP10)}
    \label{tab:shap_common}
    \resizebox{0.65\textwidth}{!}{
    \begin{tabular}{r l r r}
        \toprule
        順位 & 単語 & 感情重要度 & 評価重要度 \\
        \midrule
        1 & 学ぶ & 0.001278 & 0.001386 \\
        2 & 理解 & 0.001073 & 0.000833 \\
        3 & 総括 & 0.000974 & 0.000952 \\
        4 & 推奨 & 0.001132 & 0.000755 \\
        5 & 人数 & 0.001195 & 0.000704 \\
        6 & 把握 & 0.000891 & 0.000682 \\
        7 & 習得 & 0.000823 & 0.000645 \\
        8 & 基礎 & 0.000756 & 0.000612 \\
        9 & 応用 & 0.000698 & 0.000589 \\
        10 & 実践 & 0.000654 & 0.000567 \\
        \bottomrule
    \end{tabular}
    }
\end{table}

共通要因は,感情スコアと評価スコアの双方に同時に寄与する語彙であり,限られた資源で授業改善を行う際の「投資効率が高い要因」と解釈できる.上位語彙例には学習成果や理解に関わる表現が含まれるため,学びの実感が満足度と評価の双方に関係する可能性が示唆される.

マルチタスク分析の語彙はサブワード単位であり,語幹や断片が混在する.そのため,本研究では語彙の詳細列挙による解釈は控え,語彙単位の議論は今後の課題とする.より厳密な傾向把握には,フレーズ単位の統合や頻度併記を行った再分析が必要である.

各要因タイプの解釈を表\ref{tab:shap_interpret}に示す.語彙例はサブワード断片を含むため,ここでは省略する.

\begin{table}[t]
    \centering
    \caption{要因タイプの解釈}
    \label{tab:shap_interpret}
    \resizebox{0.75\textwidth}{!}{
    \begin{tabular}{l l}
        \toprule
        要因タイプ & 解釈の方向性 \\
        \midrule
        共通要因(満足度) & 感情と評価の双方に影響する要因 \\
        感情特化要因 & 感情的満足に強く影響する要因 \\
        評価特化要因 & 評価スコアに特に影響する要因 \\
        低重要度要因 & 影響が限定的な要因 \\
        \bottomrule
    \end{tabular}
    }
\end{table}

感情特化要因は感情的評価に関わる表現が中心となる可能性がある一方で,評価特化要因は学習成果や有用性に関わる表現が中心となる可能性がある.ただし,語彙単位の詳細な検証は本研究の範囲外であり,今後の再分析が必要である.

マルチタスク学習によって「共通要因」「感情特化要因」「評価特化要因」を分離できる点は,授業改善の方針を選択する際に有用である.具体的には以下の戦略が考えられる.

\begin{itemize}
\item \textbf{効率重視の戦略}: 共通要因への対応を優先することで,感情と評価の両方を同時に向上させる.
\item \textbf{満足度重視の戦略}: 感情特化要因への対応により,学生の感情的満足度を高める.
\item \textbf{評価重視の戦略}: 評価特化要因への対応により,評価スコアの改善を図る.
\end{itemize}

共通要因の割合が18.0\%にとどまる点は,満足度と評価が必ずしも完全に一致するわけではないことを示している.これは,評価スコアが授業の「成果」や「有用性」を反映しやすい一方で,感情スコアは「楽しさ」や「雰囲気」といった情緒的側面を反映しやすいことによるものと解釈できる.

感情特化要因(1,200語,37.5\%)は,学生の感情的評価を強く反映する語彙群である.これらは授業の雰囲気や楽しさ,満足感に関わる表現が中心となる傾向がある.感情特化要因に含まれる語彙は,感情スコアの上昇に寄与するが,評価スコアへの影響は必ずしも大きくない.これは,感情的満足と評価スコアが必ずしも一致しない可能性を示唆する.

評価特化要因(532語,16.6\%)は,授業の有用性や学習成果,授業運営の評価に関わる語彙が中心となると考えられる.これらは評価スコアの改善に直結しやすいが,感情面の改善には限定的となる可能性がある.評価特化要因への対応は,授業評価スコアを向上させたい場合に有効であるが,学生の感情的満足度を高めるには別のアプローチが必要となる.

この区別により,「満足感を高める施策」と「評価スコアを高める施策」を分けて設計できる.例えば以下のような活用が考えられる.

\begin{enumerate}
\item \textbf{授業改善の優先順位付け}: 共通要因への対応を最優先とし,次に目的に応じて感情特化または評価特化要因に対応する.
\item \textbf{授業タイプ別の戦略}: 娯楽性を重視する授業では感情特化要因を,実践力養成を重視する授業では評価特化要因を重視する.
\item \textbf{フィードバックの解釈}: 自由記述を分析する際,感情的表現と評価的表現を区別して解釈する.
\end{enumerate}

%%%%%%%%%%%%%%%%%%%%%%%%%%%%%%%%%%%%%%%%%%%%%%%%%%%%%%%%%%%%%%%%%%%%%%%%%%%%%%%
\section{順序回帰モデルの結果}
%%%%%%%%%%%%%%%%%%%%%%%%%%%%%%%%%%%%%%%%%%%%%%%%%%%%%%%%%%%%%%%%%%%%%%%%%%%%%%%

授業評価スコアは1点から4点までの順序尺度であるため,順序回帰モデルの導入を検討した.順序回帰では,各評価段階への遷移確率を推定し,評価段階ごとの寄与要因を分析できる.

予備的な実験として,評価スコアが3点以上となる確率(P3+)と4点となる確率(P4)を個別に予測するモデルを構築した.結果の詳細は追加実験の完了後に報告する予定であるが,以下の傾向が示唆されている.

\begin{itemize}
\item 「普通(3点)」から「良い(4点)」への向上には,学習成果の実感に関わる要因が特に重要である.
\item 評価段階によって寄与する要因が異なる可能性があり,きめ細かな改善施策の設計に活用できる.
\end{itemize}

%%%%%%%%%%%%%%%%%%%%%%%%%%%%%%%%%%%%%%%%%%%%%%%%%%%%%%%%%%%%%%%%%%%%%%%%%%%%%%%
\section{総合考察}
%%%%%%%%%%%%%%%%%%%%%%%%%%%%%%%%%%%%%%%%%%%%%%%%%%%%%%%%%%%%%%%%%%%%%%%%%%%%%%%

\subsection{仮説検証と主要な発見}
本章の結果を踏まえ,研究仮説の検証を行う.

\textbf{仮説1}(感情スコアと評価スコアには正の相関がある)は,相関分析により支持された.ピアソン相関係数0.3097($p<0.000001$)の中程度の正の相関が確認され,複数の相関指標で一貫した結果が得られた.

\textbf{仮説2}(共通要因が存在する)は,SHAP分析により支持された.577語(18.0\%)の共通要因が抽出され,感情と評価の双方に寄与する語彙群が存在することが示された.

\textbf{仮説3}(マルチタスク学習により要因分離が可能)は,マルチタスクモデルのSHAP分析により支持された.共通要因・感情特化要因・評価特化要因・低重要度要因の4グループへの分類に成功した.

本研究における主要な発見は以下の通りである.

第一に,理解容易性に関わる表現がポジティブ判定の上位に含まれる傾向が確認された.これは,授業内容の理解しやすさが満足度に関係する可能性を示唆する.

第二に,興味・関心に関わる表現が上位に含まれる傾向が確認された.これは,授業設計上の興味喚起が感情的満足度に関係する可能性を示唆する.

第三に,共通要因と特化要因の分離に成功した.マルチタスク学習とSHAP分析の組み合わせにより,18.0\%の語彙が共通要因として特定され,効率的な授業改善の指針が得られた.

第四に,感情スコアと評価スコアは関連するが同一ではないことが確認された.相関係数0.3097は「中程度」であり,両者は部分的に重複しつつも異なる側面を捉えていることが示された.

本研究の結果は,先行研究の知見と以下の点で整合的である.

感情分析の精度(正解率77\%)は,教育分野のテキスト分類に関する先行研究と同程度の水準である\cite{rajput2016,sindhu2019}.教育分野の自由記述は感情表現が控えめであり,他ドメインと比較して分類が困難であることが知られているが\cite{liu2012},BERTによる微調整により実用的な精度を達成した\cite{bert}.

理解容易性や興味・関心が満足度に関係するという結果は,授業評価に関する先行研究の知見と一致する\cite{marsh2007,spooren2013}.ただし,本研究ではSHAP分析により要因を定量化し\cite{shap},語彙単位の寄与度を示した点が新規性である.

\subsection{研究の限界と教育改善への示唆}
本研究には以下の限界がある.

第一に,因果関係の検証は行っていない.相関分析やSHAP分析は関連性を示すものであり,理解容易性を高めれば評価が上がるといった因果的主張は本研究からは導けない.

第二に,データは単一大学に限定されており,他大学への一般化可能性は検証されていない.

第三に,サブワード単位の分析では,語彙の文脈依存的な意味を完全に捉えられない場合がある.

本研究の結果は,教育改善に以下の示唆を提供する.

第一に,授業内容の理解しやすさを高めることが,満足度向上の最も効果的な方策であると考えられる.説明の明確化,構造化された教材,理解度確認の機会設定などが有効であると推察される.

第二に,授業の面白さ・楽しさを高める工夫も効果的である.興味を引く導入,実践的な例示,対話的な活動などが考えられる.

第三に,共通要因への対応を優先することで,限られた資源で効率的な改善が可能である.学習成果や理解の実感に関わる要素の充実が推奨される.

\clearpage
