%-----------------------------
\chapter{結果と考察}
%-----------------------------

本章では,基礎統計量,感情分類モデルの性能,相関分析結果,SHAP分析による要因抽出結果を示し,得られた知見を考察する.

%%%%%%%%%%%%%%%%%%%%%%%%%%%%%%%%%%%%%%%%%%%%%%%%%%%%%%%%%%%%%%%%%%%%%%%%%%%%%%%
\section{基礎統計量}
%%%%%%%%%%%%%%%%%%%%%%%%%%%%%%%%%%%%%%%%%%%%%%%%%%%%%%%%%%%%%%%%%%%%%%%%%%%%%%%

\subsection{感情スコアと授業評価スコアの分布}
感情スコアと授業評価スコアの基本統計量を表\ref{tab:basicstats}に示す.感情スコアは授業単位で集約した値であり,−1(ネガティブ)から+1(ポジティブ)の範囲をとる.

\begin{table}[t]
    \centering
    \caption{感情スコアと授業評価スコアの基本統計量}
    \label{tab:basicstats}
    \resizebox{0.7\textwidth}{!}{
    \begin{tabular}{l r r}
        \toprule
        統計量 & 感情スコア & 授業評価スコア \\
        \midrule
        平均 & 0.001 & 3.459 \\
        標準偏差 & 0.260 & 0.216 \\
        最小値 & −1.000 & 2.000 \\
        第1四分位数(Q1) & −0.167 & 3.330 \\
        中央値(Q2) & 0.000 & 3.480 \\
        第3四分位数(Q3) & 0.167 & 3.600 \\
        最大値 & 1.000 & 4.000 \\
        \bottomrule
    \end{tabular}
    }
\end{table}

感情スコアは平均0.001でほぼニュートラルに近く,授業評価スコアは平均3.459点である.

\subsection{教師データのラベル分布}
教師データのラベル分布を表\ref{tab:labeldist}に示す.ニュートラルが62.8\%と大半を占め,ネガティブとポジティブは少数である.

\begin{table}[t]
    \centering
    \caption{教師データのラベル分布(1,000件)}
    \label{tab:labeldist}
    \resizebox{0.6\textwidth}{!}{
    \begin{tabular}{l r r}
        \toprule
        ラベル & 件数 & 割合 \\
        \midrule
        ネガティブ & 191 & 19.1\% \\
        ニュートラル & 628 & 62.8\% \\
        ポジティブ & 180 & 18.0\% \\
        \midrule
        合計 & 1,000 & 100.0\% \\
        \bottomrule
    \end{tabular}
    }
\end{table}

%%%%%%%%%%%%%%%%%%%%%%%%%%%%%%%%%%%%%%%%%%%%%%%%%%%%%%%%%%%%%%%%%%%%%%%%%%%%%%%
\section{感情分類モデルの性能}
%%%%%%%%%%%%%%%%%%%%%%%%%%%%%%%%%%%%%%%%%%%%%%%%%%%%%%%%%%%%%%%%%%%%%%%%%%%%%%%

\subsection{全体性能}
感情分類モデルの性能指標を表\ref{tab:perf}に示す.検証データ200件に対して正解率0.770,マクロ平均F1スコア0.706を達成した.

\begin{table}[t]
    \centering
    \caption{感情分類モデルの性能指標(検証データ200件)}
    \label{tab:perf}
    \resizebox{0.7\textwidth}{!}{
    \begin{tabular}{l r}
        \toprule
        指標 & 値 \\
        \midrule
        正解率(Accuracy) & 0.770 \\
        マクロ平均適合率(Precision) & 0.707 \\
        マクロ平均再現率(Recall) & 0.705 \\
        マクロ平均F1スコア & 0.706 \\
        重み付き平均F1スコア & 0.770 \\
        \bottomrule
    \end{tabular}
    }
\end{table}

\subsection{クラス別傾向}
クラス別の性能指標を表\ref{tab:perf_class}に示す.ニュートラルが最も高い性能を示し,ネガティブとポジティブは相対的に低い.

\begin{table}[t]
    \centering
    \caption{クラス別の性能指標}
    \label{tab:perf_class}
    \resizebox{0.75\textwidth}{!}{
    \begin{tabular}{l r r r r}
        \toprule
        クラス & 適合率 & 再現率 & F1スコア & サポート \\
        \midrule
        ネガティブ & 0.659 & 0.675 & 0.667 & 40 \\
        ニュートラル & 0.833 & 0.833 & 0.833 & 132 \\
        ポジティブ & 0.630 & 0.607 & 0.618 & 28 \\
        \bottomrule
    \end{tabular}
    }
\end{table}

%%%%%%%%%%%%%%%%%%%%%%%%%%%%%%%%%%%%%%%%%%%%%%%%%%%%%%%%%%%%%%%%%%%%%%%%%%%%%%%
\section{感情スコアと授業評価スコアの相関分析}
%%%%%%%%%%%%%%%%%%%%%%%%%%%%%%%%%%%%%%%%%%%%%%%%%%%%%%%%%%%%%%%%%%%%%%%%%%%%%%%

\subsection{相関分析の結果}
授業単位で集約した感情スコアと授業評価スコアの相関分析結果を表\ref{tab:correlation}に示す(N=3,268).

\begin{table}[t]
    \centering
    \caption{感情スコアと授業評価スコアの相関分析結果(N=3,268)}
    \label{tab:correlation}
    \resizebox{0.75\textwidth}{!}{
    \begin{tabular}{l r r l}
        \toprule
        指標 & 相関係数 & $p$値 & 解釈 \\
        \midrule
        ピアソン相関係数 & 0.3097 & $<0.000001$ & 中程度の正の相関 \\
        スピアマン順位相関係数 & 0.2970 & $<0.000001$ & 中程度の正の相関 \\
        ケンドール順位相関係数 & 0.2042 & $<0.000001$ & 弱〜中程度の正の相関 \\
        \bottomrule
    \end{tabular}
    }
\end{table}

\subsection{散布図と解釈}
感情スコアと授業評価スコアの散布図を図\ref{fig:correlation}に示す.中程度の正の相関が確認され,感情スコアは評価スコアと関連するが同一概念ではない可能性が示唆される.

\begin{figure}[t]
    \centering
    \includegraphics[width=0.7\textwidth]{fig/correlation_scatter.png}
    \caption{感情スコアと授業評価スコアの散布図(N=3,268)}
    \label{fig:correlation}
\end{figure}

%%%%%%%%%%%%%%%%%%%%%%%%%%%%%%%%%%%%%%%%%%%%%%%%%%%%%%%%%%%%%%%%%%%%%%%%%%%%%%%
\section{SHAP分析}
%%%%%%%%%%%%%%%%%%%%%%%%%%%%%%%%%%%%%%%%%%%%%%%%%%%%%%%%%%%%%%%%%%%%%%%%%%%%%%%

\subsection{単一タスクモデルの重要語}
感情分類モデルに対するSHAP分析を行い,ポジティブ判定とネガティブ判定に寄与する重要語を抽出した.上位10語を表\ref{tab:shap_single}および表\ref{tab:shap_negative}に示す.

\begin{table}[t]
    \centering
    \caption{ポジティブ判定に寄与する重要語例(TOP10)}
    \label{tab:shap_single}
    \resizebox{0.7\textwidth}{!}{
    \begin{tabular}{r l r r}
        \toprule
        順位 & 単語 & 平均SHAP値 & 出現回数 \\
        \midrule
        1 & やす & 0.2660 & 337 \\
        2 & 良かっ & 0.2466 & 207 \\
        3 & おもしろ & 0.2438 & 10 \\
        4 & よかっ & 0.2251 & 195 \\
        5 & 面白 & 0.2178 & 100 \\
        6 & 楽しい & 0.1959 & 67 \\
        7 & 楽しめる & 0.1876 & 6 \\
        8 & ありが & 0.1760 & 19 \\
        9 & 楽し & 0.1642 & 192 \\
        10 & 面白い & 0.1518 & 37 \\
        \bottomrule
    \end{tabular}
    }
\end{table}

\begin{table}[t]
    \centering
    \caption{ネガティブ判定に寄与する重要語例(TOP10)}
    \label{tab:shap_negative}
    \resizebox{0.7\textwidth}{!}{
    \begin{tabular}{r l r r}
        \toprule
        順位 & 単語 & 平均SHAP値 & 出現回数 \\
        \midrule
        1 & ほし & −0.0443 & 5 \\
        2 & ほう & −0.0425 & 98 \\
        3 & 大 & −0.0346 & 86 \\
        4 & まじ & −0.0314 & 5 \\
        5 & 難しかっ & −0.0311 & 88 \\
        6 & 直す & −0.0264 & 6 \\
        7 & ほしい & −0.0263 & 45 \\
        8 & 欲しい & −0.0247 & 33 \\
        9 & 奥 & −0.0219 & 8 \\
        10 & 器具 & −0.0211 & 7 \\
        \bottomrule
    \end{tabular}
    }
\end{table}

重要語はサブワード断片を含むため,語彙の解釈は傾向として整理する必要がある.理解容易性や興味・関心に関わる表現が上位に含まれる傾向が確認される.

\subsection{マルチタスク学習の要因タイプ}
マルチタスクモデルのSHAP分析により,語彙を4つの要因タイプに分類した結果を表\ref{tab:shap_types}に示す.

\begin{table}[t]
    \centering
    \caption{語彙の要因タイプ別内訳(3,198語)}
    \label{tab:shap_types}
    \resizebox{0.7\textwidth}{!}{
    \begin{tabular}{l r r l}
        \toprule
        要因タイプ & 語彙数 & 割合 & 特徴 \\
        \midrule
        共通要因(満足度) & 577 & 18.0\% & 両スコアに寄与 \\
        感情特化要因 & 1,200 & 37.5\% & 感情スコアのみに寄与 \\
        評価特化要因 & 532 & 16.6\% & 評価スコアのみに寄与 \\
        低重要度要因 & 889 & 27.8\% & 両スコアへの影響小 \\
        \midrule
        合計 & 3,198 & 100.0\% & — \\
        \bottomrule
    \end{tabular}
    }
\end{table}

共通要因の重要語例(TOP5)を表\ref{tab:shap_common}に示す.

\begin{table}[t]
    \centering
    \caption{共通要因(満足度要因)の重要語例(TOP5)}
    \label{tab:shap_common}
    \resizebox{0.6\textwidth}{!}{
    \begin{tabular}{r l r r}
        \toprule
        順位 & 単語 & 感情重要度 & 評価重要度 \\
        \midrule
        1 & 学ぶ & 0.001278 & 0.001386 \\
        2 & 理解 & 0.001073 & 0.000833 \\
        3 & 総括 & 0.000974 & 0.000952 \\
        4 & 推奨 & 0.001132 & 0.000755 \\
        5 & 人数 & 0.001195 & 0.000704 \\
        \bottomrule
    \end{tabular}
    }
\end{table}

共通要因は感情と評価の双方に寄与する語彙群であり,改善の優先順位付けに活用できる可能性がある.

%%%%%%%%%%%%%%%%%%%%%%%%%%%%%%%%%%%%%%%%%%%%%%%%%%%%%%%%%%%%%%%%%%%%%%%%%%%%%%%
\section{総合考察}
%%%%%%%%%%%%%%%%%%%%%%%%%%%%%%%%%%%%%%%%%%%%%%%%%%%%%%%%%%%%%%%%%%%%%%%%%%%%%%%

\subsection{仮説検証}
仮説1は,相関分析により中程度の正の相関が確認されたため支持された.仮説2は,共通要因が抽出されたことから支持された.仮説3は,共通要因と特化要因の分離が可能であったことから支持された.

\subsection{主要な示唆}
理解容易性や興味・関心に関わる語彙が重要語として上位に現れる傾向は,授業評価研究の知見と整合的である\cite{marsh2007,spooren2013}.また,共通要因の割合が18.0\%にとどまる点は,感情スコアと評価スコアが部分的に重複しつつも異なる側面を捉えている可能性を示唆する.本結果はSHAPによって語彙寄与度を定量化した点に特徴がある\cite{shap}.

\clearpage
