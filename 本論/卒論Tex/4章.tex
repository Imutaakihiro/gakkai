%-----------------------------
\chapter{結果}
本節では,$k$匿名化統合前後での最尤推定された結果の比較と選択肢を集計した場合の確定的効用の範囲について述べる.

\section{ブランド統合前の結果}
まず,ブランド統合前の結果について,最尤推定によって得られた各パラメータの推定値を基に,その傾向を示す.
\begin{table}
    \centering
    \caption{ブランド統合前の最尤推定されたパラメータ}
    \resizebox{1.0\textwidth}{!}{
    \begin{tabular}{c r c r c r c r c r c r c r} % 各列を中央揃え (c) に設定
       \toprule
       \multicolumn{2}{c}{1day}&\multicolumn{2}{c}{2week}&\multicolumn{2}{c}{1month}&\multicolumn{2}{c}{1day}&\multicolumn{2}{c}{2week}&\multicolumn{2}{c}{1month}\\
       \hline
       \multicolumn{1}{c}{Para.} & \multicolumn{1}{c}{Value} & \multicolumn{1}{c}{Para.} & \multicolumn{1}{c}{Value} & \multicolumn{1}{c}{Para.} & \multicolumn{1}{c}{Value} & \multicolumn{1}{c}{Para.} & \multicolumn{1}{c}{Value} & \multicolumn{1}{c}{Para.} & \multicolumn{1}{c}{Value} & \multicolumn{1}{c}{Para.} & \multicolumn{1}{c}{Value} \\ % ヘッダー部分
        \hline\hline
        $ \alpha_1 $ & $0.6809$ & $ \alpha_1 $ & $-1.4033$ & $ \alpha_1 $ & $1.8754$ & $ \gamma_{55} $ & $0.1843$ & $ \gamma_{55} $ & $0.3693$ & $ \gamma_{55} $ & $0.3880$ \\ 
        $ \alpha_2 $ & $1.5179$ & $ \alpha_2 $ & $0.2870$ & $ \alpha_2 $ & $0.0018$ & $ \gamma_{56} $ & $0.1830$ & $ \gamma_{56} $ & $0.6201$ & $ \gamma_{56} $ & $0.0348$ \\ 
        $ \alpha_3 $ & $1.8319$ & $ \alpha_3 $ & $0.3260$ & $ \alpha_3 $ & $-0.8181$ & $ \gamma_{64} $ & $0.3997$ & $ \gamma_{64} $ & $0.0549$ & $ \gamma_{64} $ & $0.2994$ \\    
        $ \alpha_4 $ & $-0.0258$ & $ \alpha_4 $ & $-0.0389$ & $ \alpha_4 $ & $-0.0377$ & $ \gamma_{65} $ & $0.4183$ & $ \gamma_{65} $ & $0.2292$ & $ \gamma_{65} $ & $0.6687$ \\ 
        $ \alpha_5 $ & $0.0090$ & $ \alpha_5 $ & $0.0084$ & $ \alpha_5 $ & $0.0081$ & $ \gamma_{66} $ & $0.1820$ & $ \gamma_{66} $ & $0.7159$ & $ \gamma_{66} $ & $0.0319$ \\
        $ \alpha_6 $ & $0.4410$ & $ \alpha_6 $ & $0.8195$ & $ \alpha_6 $ & $-6.9543$ & $ \gamma_{73} $ & $0.0415$ & $ \gamma_{73} $ & $0.0000$ & $ \gamma_{73} $ & $0.0414$ \\
        $ \mu_1 $ & $0.7983$ & $ \mu_1 $ & $0.0001$ & $ \mu_1 $ & $0.0253$ & $ \gamma_{74} $ & $0.0282$ & $ \gamma_{74} $ & $0.0000$ & $ \gamma_{74} $ & $0.0000$ \\ 
        $ \mu_2 $ & $0.5509$ & $ \mu_2 $ & $0.0001$ & $ \mu_2 $ & $0.0027$ & $ \gamma_{75} $ & $0.1850$ & $ \gamma_{75} $ & $1.0000$ & $ \gamma_{75} $ & $0.9428$ \\ 
        $ \mu_3 $ & $0.5864$ & $ \mu_3 $ & $0.9999$ & $ \mu_3 $ & $0.7843$ & $ \gamma_{76} $ & $0.7453$ & $ \gamma_{76} $ & $0.0000$ & $ \gamma_{76} $ & $0.0158$ \\    
        $ \mu_4 $ & $0.0631$ & $ \mu_4 $ & $0.1472$ & $ \mu_4 $ & $0.8909$ & $ \gamma_{83} $ & $0.1820$ & $ \gamma_{83} $ & $0.9989$ & $ \gamma_{83} $ & $0.3258$ \\ 
        $ \mu_5 $ & $0.6148$ & $ \mu_5 $ & $0.9979$ & $ \mu_5 $ & $0.7366$ & $ \gamma_{84} $ & $0.1268$ & $ \gamma_{84} $ & $0.0011$ & $ \gamma_{84} $ & $0.0074$ \\
        $ \gamma_{11} $ & $0.0022$ & $ \gamma_{11} $ & $0.0000$ & $ \gamma_{11} $ & $0.0425$ & $ \gamma_{85} $ & $0.1880$ & $ \gamma_{85} $ & $0.0000$ & $ \gamma_{85} $ & $0.0025$ \\   
        $ \gamma_{14} $ & $0.0086$ & $ \gamma_{14} $ & $0.0000$ & $ \gamma_{14} $ & $0.8331$ & $ \gamma_{86} $ & $0.5031$ & $ \gamma_{86} $ & $0.0000$ & $ \gamma_{86} $ & $0.6643$ \\   
        $ \gamma_{15} $ & $0.0033$ & $ \gamma_{15} $ & $0.9980$ & $ \gamma_{15} $ & $0.0670$ & $ \gamma_{93} $ & $0.1159$ & $ \gamma_{93} $ & $0.0000$ & $ \gamma_{93} $ & $0.0133$ \\   
        $ \gamma_{16} $ & $0.0049$ & $ \gamma_{16} $ & $0.0020$ & $ \gamma_{16} $ & $0.0574$ & $ \gamma_{94} $ & $0.5497$ & $ \gamma_{94} $ & $0.0000$ & $ \gamma_{94} $ & $0.0087$ \\   
        $ \gamma_{21} $ & $0.3499$ & $ \gamma_{21} $ & $0.0000$ & $ \gamma_{21} $ & $0.3288$ & $ \gamma_{95} $ & $0.1396$ & $ \gamma_{95} $ & $1.0000$ & $ \gamma_{95} $ & $0.1116$ \\   
        $ \gamma_{24} $ & $0.3168$ & $ \gamma_{24} $ & $0.0000$ & $ \gamma_{24} $ & $0.6692$ & $ \gamma_{96} $ & $0.1949$ & $ \gamma_{96} $ & $0.0000$ & $ \gamma_{96} $ & $0.8664$ \\   
        $ \gamma_{25} $ & $0.2901$ & $ \gamma_{25} $ & $0.2560$ & $ \gamma_{25} $ & $0.0014$ & $ \gamma_{104} $ & $0.3390$ & $ \gamma_{104} $ & $0.0000$ & $ \gamma_{104} $ & $0.0001$ \\   
        $ \gamma_{26} $ & $0.0433$ & $ \gamma_{26} $ & $0.7440$ & $ \gamma_{26} $ & $0.0006$ & $ \gamma_{105} $ & $0.6240$ & $ \gamma_{105} $ & $0.0000$ & $ \gamma_{105} $ & $0.9951$ \\   
        $ \gamma_{32} $ & $0.1129$ & $ \gamma_{32} $ & $0.0000$ & $ \gamma_{32} $ & $0.0062$ & $ \gamma_{106} $ & $0.0370$ & $ \gamma_{106} $ & $1.0000$ & $ \gamma_{106} $ & $0.0048$ \\   
        $ \gamma_{34} $ & $0.7800$ & $ \gamma_{34} $ & $0.1723$ & $ \gamma_{34} $ & $0.9805$ & $ \gamma_{114} $ & $0.2236$ & $ \gamma_{114} $ & $0.0000$ & $ \gamma_{114} $ & $0.6452$ \\   
        $ \gamma_{35} $ & $0.1009$ & $ \gamma_{35} $ & $0.8195$ & $ \gamma_{35} $ & $0.0124$ & $ \gamma_{115} $ & $0.7054$ & $ \gamma_{115} $ & $1.0000$ & $ \gamma_{115} $ & $0.1148$ \\   
        $ \gamma_{36} $ & $0.0063$ & $ \gamma_{36} $ & $0.0082$ & $ \gamma_{36} $ & $0.0010$ & $ \gamma_{116} $ & $0.0710$ & $ \gamma_{116} $ & $0.0000$ & $ \gamma_{116} $ & $0.2400$ \\   
        $ \gamma_{42} $ & $0.0713$ & $ \gamma_{42} $ & $0.0000$ & $ \gamma_{42} $ & $0.1225$ & $ \gamma_{124} $ & $0.5980$ & $ \gamma_{124} $ & $0.0000$ & $ \gamma_{124} $ & $0.0128$ \\   
        $ \gamma_{44} $ & $0.4635$ & $ \gamma_{44} $ & $0.0014$ & $ \gamma_{44} $ & $0.0264$ & $ \gamma_{125} $ & $0.2121$ & $ \gamma_{125} $ & $0.0020$ & $ \gamma_{125} $ & $0.9630$ \\   
        $ \gamma_{45} $ & $0.3807$ & $ \gamma_{45} $ & $0.9986$ & $ \gamma_{45} $ & $0.4901$ & $ \gamma_{126} $ & $0.1899$ & $ \gamma_{126} $ & $0.9980$ & $ \gamma_{126} $ & $0.0243$ \\   
        $ \gamma_{46} $ & $0.0845$ & $ \gamma_{46} $ & $0.0000$ & $ \gamma_{46} $ & $0.3611$ & $ \gamma_{134} $ & $0.1917$ & $ \gamma_{134} $ & $1.0000$ & $ \gamma_{134} $ & $0.0124$ \\   
        $ \gamma_{52} $ & $0.0533$ & $ \gamma_{52} $ & $0.0000$ & $ \gamma_{52} $ & $0.5338$ & $ \gamma_{135} $ & $0.8083$ & $ \gamma_{135} $ & $0.0000$ & $ \gamma_{135} $ & $0.9876$ \\   
        $ \gamma_{54} $ & $0.5794$ & $ \gamma_{54} $ & $0.0106$ & $ \gamma_{54} $ & $0.0434$ & $ $ & $ $ & $ $ & $ $ & $ $ & $ $ \\   
      \bottomrule  
    \end{tabular}
    }
\end{table}

効用関数のパラメータを見ると,期間別で一貫した傾向が見られるものの,パラメータごとに顕著なばらつきが存在する.
$\alpha_2$では一貫して正の値を示し,$\alpha_4$では一貫して負の値を示しており,期間別で見ても同じような傾向が示唆される.
$\alpha_1$と$\alpha_6$では,期間別でパラメータの符号に変化があり,ブランド選択の時間依存性や消費者の選好変化を反映している可能性を示唆している.

非類似度パラメータについては,期間ごとに大きなばらつきが見られる.
特に,$\mu_3$および$\mu_5$の2週間および1か月の推定値が0.9999や0.9979と非常に高い値を示していることから,特定のブランド間で強い相関があることが示唆される.
これは,消費者が特定のブランドを選択する際に,代替選択肢としての競争関係が顕著であることを示している.
一方で,$\mu_1$および$\mu_2$は期間によって極端に低い値を示す場合があり,短期間では選択行動がより分散している可能性を示唆している.

アロケーションパラメータに関しては,期間ごとに異なるブランドの分布が観察された.
例えば,$\gamma_{55}$や$\gamma_{56}$は比較的安定しているが,$\gamma_{75}$や$\gamma_{76}$のようなパラメータは大きく変動し特定の期間で極端な値を示している.
また,0の値を示すパラメータが存在することは,ある特定のブランドやカテゴリーが消費者の選択肢から事実上排除されている可能性を示唆する.

\section{ブランド統合後の結果}
次に,ブランド統合後の結果について,最尤推定によって得られた各パラメータの推定値を基に,その傾向を示す.
\begin{table}[t]
    \centering
    \caption{ブランド統合後の最尤推定されたパラメータ}
    \resizebox{0.99\textwidth}{!}{
    \begin{tabular}{c r c r c r c r c r c r c r} % 各列を中央揃え (c) に設定
       \toprule
       \multicolumn{2}{c}{1day}&\multicolumn{2}{c}{2week}&\multicolumn{2}{c}{1month}&\multicolumn{2}{c}{1day}&\multicolumn{2}{c}{2week}&\multicolumn{2}{c}{1month}\\
       \hline
       \multicolumn{1}{c}{Para.} & \multicolumn{1}{c}{Value} & \multicolumn{1}{c}{Para.} & \multicolumn{1}{c}{Value} & \multicolumn{1}{c}{Para.} & \multicolumn{1}{c}{Value} & \multicolumn{1}{c}{Para.} & \multicolumn{1}{c}{Value} & \multicolumn{1}{c}{Para.} & \multicolumn{1}{c}{Value} & \multicolumn{1}{c}{Para.} & \multicolumn{1}{c}{Value} \\ % ヘッダー部分
        \hline\hline
        $ \alpha_1 $ & $8.2714$ & $ \alpha_1 $ & $4.3831$ & $ \alpha_1 $ & $3.8316$ & $ \gamma_{15} $ & $0.0000$ & $ \gamma_{15} $ & $0.0000$ & $ \gamma_{15} $ & $0.0001$ \\ 
        $ \alpha_2 $ & $8.5423$ & $ \alpha_2 $ & $0.3986$ & $ \alpha_2 $ & $-15.6204$ & $ \gamma_{16} $ & $0.0061$ & $ \gamma_{16} $ & $0.0000$ & $ \gamma_{16} $ & $0.9999$ \\ 
        $ \alpha_3 $ & $2.2577$ & $ \alpha_3 $ & $19.7751$ & $ \alpha_3 $ & $19.2397$ & $ \gamma_{22} $ & $0.7100$ & $ \gamma_{22} $ & $0.0003$ & $ \gamma_{22} $ & $1.0000$ \\    
        $ \alpha_4 $ & $-1.0814$ & $ \alpha_4 $ & $7.3926$ & $ \alpha_4 $ & $-6.8555$ & $ \gamma_{24} $ & $0.0001$ & $ \gamma_{24} $ & $0.0000$ & $ \gamma_{24} $ & $0.0000$ \\ 
        $ \alpha_5 $ & $0.6440$ & $ \alpha_5 $ & $15.1747$ & $ \alpha_5 $ & $14.0522$ & $ \gamma_{25} $ & $0.2899$ & $ \gamma_{25} $ & $0.0000$ & $ \gamma_{25} $ & $0.0000$ \\
        $ \alpha_6 $ & $-12.3034$ & $ \alpha_6 $ & $3.7961$ & $ \alpha_6 $ & $17.1754$ & $ \gamma_{26} $ & $0.0000$ & $ \gamma_{26} $ & $0.9997$ & $ \gamma_{26} $ & $0.0000$ \\
        $ \mu_1 $ & $1.0000$ & $ \mu_1 $ & $0.0001$ & $ \mu_1 $ & $0.0001$ & $ \gamma_{33} $ & $0.0002$ & $ \gamma_{33} $ & $0.2505$ & $ \gamma_{33} $ & $0.0000$ \\ 
        $ \mu_2 $ & $0.5274$ & $ \mu_2 $ & $0.0001$ & $ \mu_2 $ & $0.1520$ & $ \gamma_{34} $ & $0.0014$ & $ \gamma_{34} $ & $0.5569$ & $ \gamma_{34} $ & $0.0000$ \\ 
        $ \mu_3 $ & $0.9998$ & $ \mu_3 $ & $1.0000$ & $ \mu_3 $ & $0.0001$ & $ \gamma_{35} $ & $0.9983$ & $ \gamma_{35} $ & $0.0001$ & $ \gamma_{35} $ & $0.6972$ \\    
        $ \mu_4 $ & $0.4338$ & $ \mu_4 $ & $0.9996$ & $ \mu_4 $ & $0.8737$ & $ \gamma_{36} $ & $0.0000$ & $ \gamma_{36} $ & $0.1925$ & $ \gamma_{36} $ & $0.3028$ \\ 
        $ \mu_5 $ & $0.0001$ & $ \mu_5 $ & $1.0000$ & $ \mu_5 $ & $0.0039$ & $ \gamma_{44} $ & $0.9873$ & $ \gamma_{44} $ & $1.0000$ & $ \gamma_{44} $ & $1.0000$ \\
        $ \gamma_{11} $ & $0.9939$ & $ \gamma_{11} $ & $0.9193$ & $ \gamma_{11} $ & $0.0000$ & $ \gamma_{45} $ & $0.0127$ & $ \gamma_{45} $ & $0.0000$ & $ \gamma_{45} $ & $0.0000$ \\   
        $ \gamma_{14} $ & $0.0000$ & $ \gamma_{14} $ & $0.0807$ & $ \gamma_{14} $ & $0.0000$ & $ $ & $ $ & $ $ & $ $ & $ $ & $ $ \\   
        \bottomrule  
    \end{tabular}
    }
\end{table}
\begin{figure}
効用関数のパラメータに関しては,統合前と比較して大きな変動が見られる.
特に,$\alpha_1$から$\alpha_6$の値が,期間(1日,2週間,1か月)ごとに大きく異なり,選択肢の統合により消費者のブランド選好が変化していることが示唆される.
例えば,$\alpha_1$の推定値は統合後に8.2714(1日),4.3831(2週間),3.8316(1か月)と全体的に高い正の値を示しており,統合されたブランドに対する効用が一貫して高いことがわかる.
一方で,$\alpha_2$は1か月後に-15.6204と極端な負の値を示し,特定の統合ブランドに対する選好が大きく低下していることが確認される.
さらに,$\alpha_6$の推定値も期間ごとに変動が激しく,1日には-12.3034,1か月には17.1754と大きく変化していることから,匿名化とともに消費者の選択に大きな影響を与えていることが示唆される.

次に,非類似度パラメータを考察すると,ブランド統合後の市場におけるブランド間の競争関係の変化が明確に示されている.
統合前に比べて,$\mu_1$および$\mu_5$の値が1.0000(1日)または0.0001(2週間,1か月)と極端な値を示しており,ブランド間の類似性が大きく変化していることが確認できる.

アロケーションパラメータについては,ブランド統合後の選択肢の分布が変化していることが確認できる.特に,$\gamma_{15}$,$\gamma_{16}$,$\gamma_{24}$,$\gamma_{25}$の値が一部の期間で0.0000を示しており,特定の選択肢が完全に排除されるか,非常に低い選択確率を持つようになったことが示されている.一方で,$\gamma_{22}$や$\gamma_{26}$などは期間によって大きな変動を示しておる.また,$\gamma_{44}$が1.0000を示していることから,一部の選択肢は完全に市場での優位性を持つようになったと考えられる.

\section{ブランド統合による影響}
\centering
\includegraphics[scale=0.67]{1day.png}
\caption{ブランド統合前後(1day) でのGNL構造}
\end{figure}
\begin{figure}
\centering
\includegraphics[scale=0.67]{2week.png}
\caption{ブランド統合前後(2week) でのGNL構造}
\end{figure}
\begin{figure}
\centering
\includegraphics[scale=0.67]{1month.png}
\caption{ブランド統合前後(1month) でのGNL構造}
\end{figure}
ブランド統合前後の結果を比較すると,統合が消費者のブランド選択に及ぼす影響が明確に示唆される.
統合前の段階では,効用関数のパラメータに一定の傾向が見られたものの,パラメータごとにばらつきが存在し,消費者の選好が時間とともに変動していることが示唆される.特定のブランドに対する選好は一貫して正または負の傾向を示す一方で,一部のブランドでは時間経過とともに選択のされ方が変化していることが確認できる.
ブランド統合後は,効用関数のパラメータが大きく変動し,消費者のブランド選好が再編されたことが確認できる.統合されたブランドの影響力が増す一方で,一部のブランドでは選択が大幅に減少しており,ブランド統合が消費者の意思決定に直接的な影響を与えたことが示唆される.また,統合によって特定の選択肢の効用が顕著に変化し,ブランド間の競争関係が大きく変容したことが確認できる.

非類似度パラメータの変化を考察すると,ブランド統合前後でブランド間の競争構造が大きく変化していることが分かる.統合前は,特定のブランド間で強い競争関係が見られたが,統合後は一部のブランド間の類似性が高まり,消費者の選択行動にも影響を与えている.特に,統合後は特定のブランドが市場での競争優位性を確立する一方で,一部の選択肢間の競争が極端に弱まる場面も確認できる.これにより,統合がブランド間の競争力の再配置に寄与していることが示唆される.

アロケーションパラメータの変化についても,ブランド統合が選択肢の分布に与える影響が顕著であることが確認できる.統合前はブランドごとの分布が比較的安定していたが,統合後は特定のブランドが市場での優位性を高める一方で,完全に選択肢から排除されるブランドも存在することが示唆される.このことから,ブランド統合は単なる選択肢の統合にとどまらず,消費者の行動や競争環境の力学を大きく変化させる要因となっていることが分かる.それぞれのブランド統合前後のGNLモデル構造については,図4.1から図4.3に示している.


ブランド統合前後の比較結果から,統合が消費者の選択行動,ブランド間の競争関係,選択肢の分布に大きな影響を与えることが示唆される.統合によって特定のブランドの影響力が増し,消費者の選択に対する影響力が高まる一方で,他のブランドの選択確率が大幅に低下し,消費者の選択行動が変容する傾向にある.

\section{選択肢を統計した場合の確定的効用の範囲}
次に,選択肢を統計した場合の確定的効用の範囲を示す.
\begin{table}[b]
    \centering
    \caption{確定的効用の範囲}
    \resizebox{0.8\textwidth}{!}{
    \begin{tabular}{c r r r r r r r r r} % 各列を中央揃え (c) に設定
       \toprule
       \multicolumn{1}{c}{}&\multicolumn{3}{c}{1day}&\multicolumn{3}{c}{2week}&\multicolumn{3}{c}{1month}\\
       \hline
       \multicolumn{1}{c}{DU.} & \multicolumn{1}{c}{min} & \multicolumn{1}{c}{$\gamma_{Kj}\bar V_k$} & \multicolumn{1}{c}{max} & \multicolumn{1}{c}{min} & \multicolumn{1}{c}{$\gamma_{Kj}\bar V_k$} & \multicolumn{1}{c}{max} & \multicolumn{1}{c}{min} & \multicolumn{1}{c}{$\gamma_{Kj}\bar V_k$} & \multicolumn{1}{c}{max} \\ % ヘッダー部分
        \hline\hline
        $ \gamma_{11}\bar V_1 $ & $0.0106$ & $1.5718$ & $1.3073$ & $0.0000$ & $-$ & $0.0000$ & $0.1227$ & $-$ & $1.2541$ \\ 
        $ \gamma_{14}\bar V_1 $ & $0.0415$ & $-$ & $1.1836$ & $0.0000$ & $-$ & $0.0000$ & $0.9883$ & $-$ & $5.5152$ \\ 
        $ \gamma_{15}\bar V_1 $ & $0.0159$ & $-$ & $1.0839$ & $-0.4541$ & $-$ & $3.5674$ & $0.0021$ & $-$ & $0.4435$ \\ 
        $ \gamma_{16}\bar V_1 $ & $0.0237$ & $-$ & $0.1618$ & $-1.3195$ & $-$ & $0.4749$ & $0.0009$ & $0.0336$ & $0.3800$ \\ 
        $ \gamma_{22}\bar V_2 $ & $0.2067$ & $5.2924$ & $0.9635$ & $0.0000$ & $-$ & $0.0000$ & $0.0363$ & $1.4272$ & $4.6642$ \\ 
        $ \gamma_{24}\bar V_2 $ & $1.3435$ & $-$ & $6.6563$ & $0.0041$ & $-$ & $1.4704$ & $0.0765$ & $-$ & $8.3673$ \\ 
        $ \gamma_{25}\bar V_2 $ & $0.5903$ & $6.0340$ & $1.8205$ & $2.1978$ & $-$ & $6.9934$ & $0.0725$ & $-$ & $3.3902$ \\ 
        $ \gamma_{26}\bar V_2 $ & $0.0369$ & $-$ & $1.5990$ & $0.0000$ & $1.3905$ & $5.4182$ & $0.0059$ & $-$ & $1.7267$ \\ 
        $ \gamma_{33}\bar V_3 $ & $0.0779$ & $-$ & $1.1736$ & $0.0000$ & $1.3834$ & $8.4067$ & $0.0078$ & $-$ & $0.9382$ \\ 
        $ \gamma_{34}\bar V_3 $ & $0.0529$ & $-$ & $2.3808$ & $0.0000$ & $-0.9163$ & $5.0784$ & $0.0000$ & $-$ & $7.2836$ \\ 
        $ \gamma_{35}\bar V_3 $ & $0.3473$ & $4.5316$ & $0.5192$ & $1.8775$ & $-$ & $2.8063$ & $1.7701$ & $0.3026$ & $2.6458$ \\ 
        $ \gamma_{36}\bar V_3 $ & $1.3993$ & $-$ & $2.0915$ & $0.000$ & $-$ & $0.0000$ & $0.0297$ & $1.1220$ & $0.0443$ \\ 
        $ \gamma_{44}\bar V_4 $ & $0.0129$ & $4.8405$ & $2.7517$ & $0.0000$ & $2.0060$ & $4.1496$ & $0.0003$ & $2.2320$ & $4.5750$ \\ 
        $ \gamma_{45}\bar V_4 $ & $0.0046$ & $-$ & $5.0019$ & $0.0000$ & $-$ & $7.0908$ & $0.0208$ & $-$ & $5.0535$ \\ 
        $ \gamma_{46}\bar V_4 $ & $0.0046$ & $-$ & $1.2530$ & $0.0000$ & $-$ & $5.0784$ & $0.0162$ & $-$ & $4.3439$ \\ 
        \bottomrule  
    \end{tabular}
    }
\end{table}
ブランド統合前後のブランド選択モデルにおける確定的効用の範囲を比較することで,統合が消費者の選択行動に与える影響が明確に示唆される.統合前は,各ブランド選択肢の確定的効用に一定のばらつきが見られるものの,消費者の選好は比較的安定しており,大きな変動は生じていない.特定のブランドに対する選好が一貫して高い水準を維持している一方で,その他のブランド間では競争が均等に分散している傾向が確認できる.これは,統合前の市場環境において,ブランド選択が安定していたことを示唆している.

一方で,ブランド統合後には特定のブランドの確定的効用が顕著に変化し,消費者のブランド選好の再編が進んだことが確認できる.特に,ブランド統合直後の短期間では,消費者の選択行動に大きな変化は見られず,統合の影響は限定的である.統合後の中期においては,ブランド間の競争関係が変化し,消費者の選択肢の分布が再編成される傾向が見られ,特定のブランドへの選好が集中し始める傾向にある.さらに,長期間の分析では,統合の影響がより顕著になり,一部のブランドは消費者の選好において強く支持されるようになった一方で,競争力を失ったブランドの選択確率が大幅に低下し,事実上市場から淘汰されるブランドも見られる傾向にある.

このように,ブランド統合が消費者の選択行動に与える影響は,短期間では限定的であるが,中長期的には選択肢の競争力や市場の構造を大きく変化させることが示唆される.特定のブランドの確定的効用が上昇することにより,消費者の選択肢がより限定的になり,一部のブランドが市場での支配力を強める結果となる.確定的効用の値を安定させるための手法として,初期値を変更することや効用関数のパラメータの値を固定することなどが挙げられる.また,アロケーションパラメータの値の偏りが2週間よりも1ヶ月の方が安定しており,セール期間などによる購買機会に大きな変動が生じる可能性があるため,購入日時の統合を「3週間」とすることで,より安定した分析が可能になるかもしれない.

\clearpage