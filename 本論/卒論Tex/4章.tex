%-----------------------------
\chapter{結果と考察}
本章では,授業評価アンケートの基礎統計量,相関分析結果,SHAP分析による要因抽出結果を示し,得られた知見を考察する.

\section{基礎統計量}
感情スコアと授業評価スコアの基本統計量を表\ref{tab:basicstats}に示す.感情スコアは平均0.001でほぼニュートラルに近く,授業評価スコアは平均3.459点(4点満点)で比較的高い評価が多いことが分かる.

\begin{table}[t]
    \centering
    \caption{感情スコアと授業評価スコアの基本統計量}
    \label{tab:basicstats}
    \resizebox{0.6\textwidth}{!}{
    \begin{tabular}{l r r}
        \toprule
        統計量 & 感情スコア & 授業評価スコア \\
        \midrule
        平均 & 0.001 & 3.459 \\
        標準偏差 & 0.260 & 0.216 \\
        最小値 & -1.000 & 2.000 \\
        第1四分位数(Q1) & -0.167 & 3.330 \\
        中央値(Q2) & 0.000 & 3.480 \\
        第3四分位数(Q3) & 0.167 & 3.600 \\
        最大値 & 1.000 & 4.000 \\
        \bottomrule
    \end{tabular}
    }
\end{table}

教師データのラベル分布を表\ref{tab:labeldist}に示す.ニュートラルが大半を占める一方,ネガティブ・ポジティブは少数であり,クラス不均衡が存在する.

\begin{table}[t]
    \centering
    \caption{教師データのラベル分布(1,000件)}
    \label{tab:labeldist}
    \resizebox{0.6\textwidth}{!}{
    \begin{tabular}{l r}
        \toprule
        ラベル & 件数 \\
        \midrule
        ネガティブ & 191 \\
        ニュートラル & 628 \\
        ポジティブ & 180 \\
        \bottomrule
    \end{tabular}
    }
\end{table}

\section{感情スコアと授業評価スコアの相関分析}
授業単位で集約した感情スコアと授業評価スコアの相関分析結果を表\ref{tab:correlation}に示す.ピアソン相関係数は0.3097($p<0.000001$)であり,中程度の正の相関が確認された.順位相関係数でも同様に正の相関が得られ,複数の指標で一貫した結果となった.

\begin{table}[t]
    \centering
    \caption{感情スコアと授業評価スコアの相関分析結果}
    \label{tab:correlation}
    \resizebox{0.75\textwidth}{!}{
    \begin{tabular}{l r r}
        \toprule
        指標 & 相関係数 & $p$値 \\
        \midrule
        ピアソン相関係数 & 0.3097 & $<0.000001$ \\
        スピアマン順位相関係数 & 0.2970 & $<0.000001$ \\
        ケンドール順位相関係数 & 0.2042 & $<0.000001$ \\
        \bottomrule
    \end{tabular}
    }
\end{table}

\section{感情分類モデルの性能}
感情分類モデルの性能指標を表\ref{tab:perf}に示す.詳細な数値は最終的な学習条件の確定後に追記する.

\begin{table}[t]
    \centering
    \caption{感情分類モデルの性能指標}
    \label{tab:perf}
    \resizebox{0.6\textwidth}{!}{
    \begin{tabular}{l r}
        \toprule
        指標 & 値 \\
        \midrule
        正解率 & --- \\
        F1スコア & --- \\
        \bottomrule
    \end{tabular}
    }
\end{table}

\section{単一タスクモデルのSHAP分析}
感情分類モデルに対するSHAP分析を行い,ポジティブ判定に寄与する重要語を抽出した.上位10語を表\ref{tab:shap_single}に示す.「やす」「良かっ」「おもしろ」「面白」など,理解のしやすさや授業の面白さを示す語彙が上位に現れた.

\begin{table}[t]
    \centering
    \caption{ポジティブ判定に寄与する重要語TOP10}
    \label{tab:shap_single}
    \resizebox{0.75\textwidth}{!}{
    \begin{tabular}{r l r r}
        \toprule
        順位 & 単語 & 平均SHAP値 & 出現回数 \\
        \midrule
        1 & やす & 0.2660 & 337 \\
        2 & 良かっ & 0.2466 & 207 \\
        3 & おもしろ & 0.2438 & 10 \\
        4 & よかっ & 0.2251 & 195 \\
        5 & 面白 & 0.2178 & 100 \\
        6 & 楽しい & 0.1959 & 67 \\
        7 & 楽しめる & 0.1876 & 6 \\
        8 & ありが & 0.1760 & 19 \\
        9 & 楽し & 0.1642 & 192 \\
        10 & 面白い & 0.1518 & 37 \\
        \bottomrule
    \end{tabular}
    }
\end{table}

理解のしやすさや面白さに関連する語彙が上位に位置することから,学生の満足度には内容理解と興味喚起が大きく関与することが示唆される.
特に「やす」は「分かりやすい」「理解しやすい」などに共通する語幹であり,授業理解のしやすさが肯定的評価の中心的要因である可能性を示す.また,「面白」「おもしろ」「楽しい」といった語彙が複数ランクインしている点は,授業内容の興味・関心が感情評価に与える影響が大きいことを示唆する.
これらの結果は,評価スコアが高い授業ほど「理解しやすさ」と「面白さ」が同時に高まるという仮説と整合的であり,授業改善においては内容構成の明確化や関心を引く説明の設計が有効な方向性となる.
一方で,単一タスクのSHAP分析は感情スコアに対する寄与を示すものであり,評価スコアへの直接的な影響を意味するものではない点に留意が必要である.

\section{マルチタスク学習のSHAP分析}
マルチタスクモデルのSHAP分析により,感情スコアと授業評価スコアの両方に影響する共通要因と,各タスクに特化した要因を分離した.語彙の分類結果を表\ref{tab:shap_types}に示す.共通要因は577語(18.0\%)であり,満足度要因として授業改善への示唆が得られる.

\begin{table}[t]
    \centering
    \caption{語彙の要因タイプ別内訳}
    \label{tab:shap_types}
    \resizebox{0.6\textwidth}{!}{
    \begin{tabular}{l r r}
        \toprule
        要因タイプ & 語彙数 & 割合 \\
        \midrule
        共通要因(満足度) & 577 & 18.0\% \\
        感情特化要因 & 1,200 & 37.5\% \\
        評価特化要因 & 532 & 16.6\% \\
        低重要度要因 & 889 & 27.8\% \\
        \bottomrule
    \end{tabular}
    }
\end{table}

共通要因の上位語彙を表\ref{tab:shap_common}に示す.「学ぶ」「理解」「総括」「推奨」など,学習成果に関わる語彙が上位に現れている.

\begin{figure}[t]
\centering
\fbox{\rule{0pt}{40mm}\rule{0.75\textwidth}{0pt}}
\caption{要因タイプの割合(概念図)}
\label{fig:shap_ratio}
\end{figure}

\begin{table}[t]
    \centering
    \caption{要因タイプの解釈例}
    \label{tab:shap_interpret}
    \resizebox{0.6\textwidth}{!}{
    \begin{tabular}{l l l}
        \toprule
        要因タイプ & 解釈の方向性 & 代表語例 \\
        \midrule
        共通要因(満足度) & 感情と評価の双方に影響 & 学ぶ,理解,総括 \\
        感情特化要因 & 感情の高低を強く反映 & 楽しい,面白い \\
        評価特化要因 & 評価スコアに特に関与 & 推奨,人数 \\
        低重要度要因 & 影響が限定的 & (該当語彙多数) \\
        \bottomrule
    \end{tabular}
    }
\end{table}

\begin{table}[t]
    \centering
    \caption{共通要因(満足度要因)TOP5}
    \label{tab:shap_common}
    \resizebox{0.6\textwidth}{!}{
    \begin{tabular}{r l r r}
        \toprule
        順位 & 単語 & 感情重要度 & 評価重要度 \\
        \midrule
        1 & 学ぶ & 0.001278 & 0.001386 \\
        2 & 理解 & 0.001073 & 0.000833 \\
        3 & 総括 & 0.000974 & 0.000952 \\
        4 & 推奨 & 0.001132 & 0.000755 \\
        5 & 人数 & 0.001195 & 0.000704 \\
        \bottomrule
    \end{tabular}
    }
\end{table}

共通要因は,感情スコアと評価スコアの双方に同時に寄与する語彙であり,限られた資源で授業改善を行う際の「投資効率が高い要因」と解釈できる.例えば「学ぶ」「理解」「総括」といった語彙は,授業内容の理解度や学習成果の実感を示唆するため,学生にとって「学びが得られた」という感覚が満足度と評価の双方に影響している可能性が高い.\n
感情特化要因には「楽しい」「面白い」など情緒的評価を反映する語彙が多く含まれると考えられ,これらは満足感の形成には強く寄与するが,評価スコアへの影響は必ずしも大きくない可能性がある.一方で,評価特化要因は授業設計や運営の実務的な要素に関わる語彙が含まれる可能性が高く,評価スコアの改善には直接的に寄与するが,感情面の改善には限定的となることが示唆される.

このように,マルチタスク学習によって「共通要因」「感情特化要因」「評価特化要因」を分離できる点は,授業改善の方針を選択する際に有用である.具体的には,満足度と評価の両方を高めたい場合は共通要因に対する改善を優先し,感情的満足を高めたい場合は感情特化要因を,評価指標を改善したい場合は評価特化要因を重点的に改善する戦略が考えられる.

さらに,共通要因の割合が18.0\%にとどまる点は,満足度と評価が必ずしも完全に一致するわけではないことを示している.これは,評価スコアが授業の「成果」や「有用性」を反映しやすい一方で,感情スコアは「楽しさ」や「雰囲気」といった情緒的側面を反映しやすいことによるものと解釈できる.そのため,評価スコアのみでは把握しきれない学生の感情的側面を補う指標として,感情スコアを活用する意義がある.

以上より,マルチタスク学習とSHAP分析の組み合わせは,授業改善における具体的な指針を提供するだけでなく,評価スコアと感情スコアの関係性をより精緻に理解するための有効な枠組みであるといえる.

\section{感情特化要因・評価特化要因の示唆}
感情特化要因は,学生の感情的評価を強く反映する語彙群であり,授業の雰囲気や楽しさ,満足感に関わる表現が中心となる傾向がある.これらは感情スコアの上昇に寄与するが,評価スコアへの影響は必ずしも大きくない可能性がある.

一方,評価特化要因は授業の有用性や学習成果,授業運営の評価に関わる語彙が中心となると考えられる.これらは評価スコアの改善に直結しやすいが,感情面の改善には限定的となる可能性がある.

この区別により,「満足感を高める施策」と「評価スコアを高める施策」を分けて設計できる点が,本研究の結果の実践的意義である.

\section{順序回帰モデルの結果(追加実験)}
順序回帰モデルの結果は,追加実験として現在検証中である.本節では,評価段階ごとの寄与語やP2(中低評価確率),P4(高評価確率)の差異を中心に整理する予定である.結果の確定後に表・図を追加する.

\section{総合考察}
相関分析により,授業レベルでは感情スコアと授業評価スコアに統計的に有意な正の相関が確認された.これは,学生の自由記述に表れる感情が授業評価と一定の関係を持つことを示している.

SHAP分析から,理解しやすさや授業の面白さに関わる語彙がポジティブ判定に強く寄与することが示された.また,共通要因では「学ぶ」「理解」「総括」など学習成果に関わる語彙が上位にあり,感情と評価の双方を高める可能性が示唆された.

一方,本研究の結果は相関関係に基づくものであり,因果関係を主張するものではない.今後は実験的検証や介入研究により,因果的な関係の検証が必要である.

\clearpage
