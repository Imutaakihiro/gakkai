\documentclass[10.5bp, jafontscale=1.00]{ltjsarticle}
\usepackage{stix,amsmath,cite,citesort,eclbkbox,listings,jlisting}
\usepackage{graphicx}
\usepackage{titlesec}
%\usepackage[morisawa-pro]{luatexja-preset}
\usepackage[no-math,expert,haranoaji]{luatexja-preset}
\setmainfont{Times New Roman}[BoldFont=TimesNewRomanPS-BoldMT]
\setsansfont{Times New Roman}[BoldFont=TimesNewRomanPS-BoldMT]
\setlength{\oddsidemargin}{-0.in}
\setlength{\evensidemargin}{-0.in}
\setlength{\topmargin}{-0.in}
\setlength{\headheight}{0cm}
\setlength{\headsep}{0mm}
\setlength{\textheight}{43.5\baselineskip}
%\setlength{\textwidth}{50zw}
\setlength{\columnsep}{9mm}
\setlength{\topsep}{0pt}
\setlength{\partopsep}{0pt}
\setlength{\parsep}{0pt}
\titleformat*{\section}{\normalsize\bfseries}
\titleformat*{\subsection}{\normalsize\bfseries}
\titleformat*{\subsubsection}{\normalsize\bfseries}
\makeatletter % 「@」を含む場合は\makeatletterと\makeatotherで囲んだ範囲内で処理をする
\renewcommand\section{%
  \@startsection{section}% name
    {1}% level
    {0pt}% indent
    {-0\baselineskip plus -1ex minus -.2ex}% beforeskip
    {0\baselineskip plus.2ex}% afterskip
    {\normalfont\normalsize\bfseries}% style
  }
\renewcommand{\subsection}{% \newcommand から \renewcommand に変更すること
  \@startsection{subsection}% 区切りコマンドの名前(section, subsection等)
    {2}% 深さ(sectionが1, subsectionは2等)
    {0pt}% 左のインデント量
    {-0\baselineskip plus -1ex minus -0.2ex}% 前アキ 見出し上のスペース
	{0\baselineskip plus 0.2ex}% 後アキ 見出し下のスペース 負にすると見出し後のスペース
    {\normalsize\bfseries}% 見出しのフォント
}
\renewcommand{\subsubsection}{% \newcommand から \renewcommand に変更すること
  \@startsection{subsubsection}% 区切りコマンドの名前(section, subsection等)
    {3}% 深さ(sectionが1, subsectionは2等)
    {0pt}% 左のインデント量
    {-0.\baselineskip plus -1ex minus -0.2ex}% 前アキ 見出し上のスペース
	{0.\baselineskip plus 0.2ex}% 後アキ 見出し下のスペース 負にすると見出し後のスペース
    {\normalsize\bfseries}% 見出しのフォント
}
\def\linesperpage#1{%
\baselineskip=\textheight
\divide\baselineskip by #1}

\def\fitwidth#1#2{\leavevmode{%
  \setbox0=\hbox{#2}%
  \ifdim\wd0>#1 \resizebox{#1}{\height}{#2}%
  \else
    \def\@tempa{#2}%
    \ifx\@tempa\empty\hbox to#1{\hss}%
    \else\relax\hbox to#1\bgroup\hfil\@fitwidth#2{}\end@fitwidth\fi
  \fi}}
\def\@fitwidth#1#2\end@fitwidth{\def\@tempa{#2}#1%
  \ifx\@tempa\empty\let\next\end@fitwidth
  \else\hfill\def\next{\@fitwidth#2\end@fitwidth}\fi
  \next}
\def\end@fitwidth{\hfil\egroup}

\makeatother
\begin{document}
\linesperpage{48}
\pagestyle{empty}
\twocolumn[
\fontsize{14bp}{\baselineskip}\selectfont
\begin{flushleft}
{\bf Aー10}%セッション番号を入れる
\end{flushleft}
\vspace*{-\baselineskip}
\begin{center}
\textbf{卒研題目} % 卒研題目を入れる
\end{center}
\vspace*{0.5\baselineskip}
\fontsize{10.5bp}{\baselineskip}\selectfont
\begin{center}
22M1XXX 福工 太郎
\end{center}

\begin{flushright}
指導教員 小林 稔
\end{flushright}
\hspace*{1cm}
\vspace*{-0.\baselineskip}

]
\fontsize{10.5bp}{\baselineskip}\selectfont

\section{はじめに}
本研究では・・を行い,・・を明らかにする.

\section{理論}

\section{予稿集の書き方}
\subsection{原稿のサイズ}
原稿はA4サイズで作成し,この見本を参考にすること.フォントを埋め込んだPDFファイルをmyFITにて提出する.
\subsection{文章量}
1編あたりの本文(表題,教員名,氏名は含まない)は,全角22字×42行×2段を原則とするが,表題,氏名欄が増えた場合,行数減少は可とする.
\subsection{文字のと色と大きさ}
黒色,明朝体10.5ポイントを原則とする.英数字は見出しを除き半角英数字とし,Times系のTimes New Romanフォント10.5ポイントの使用を原則とする.句点は「.」,読点は「,」を用いる.
\subsection{配置}
表題:原稿用紙の第1行に割り当てられた発表番号を14ポイントで書く.続けて中央に表題を(長ければ2行)に14ポイントで書く.
学籍番号と氏名:表題からポイント数を10.5に下げて1行空け,学籍番号順に並べて中央に書く.
指導教員名:その次の行に“指導教員 ○○○○”と右寄せで書く.
本文:1行空けて2段組とし,本文を10.5ポイントで書く.
\subsection{図表}
図表は,原則としてPDF形式もしくは拡張メタファイルやSVGなどのベクター形式で貼り付ける. 
\subsection{原稿の余白}
天地左右のマージンは25mmとする.
\section{本文内容}
はじめに,本文,おわりに等の見出しはゴシック体10.5ptで数字は半角英数の太字({\bf Bold体})とする.

見出しの次は行を空けずに本文を書く.

各段落の文頭1字下げる.

数式には数式番号を行末に付ける.

数式には数式番号を行末に付ける.

数式には数式番号を行末に付ける.

\section{文字数の確認}
1234567890123456789011234567890123456789012

 4

 5

 6

 7

 8

 9

10

11

12

13

14

15

16

17

18

19

20

21

22

23

24

25

26

27

28

29

30行目

\section{おわりに}
本研究では,・・・・・・し,・・・であることを明らかにした.
\begin{thebibliography}{9}%←文献の編数が10以上となる場合には99に変更する
\bibitem{美人} ナンシー・エトコフ, なぜ美人ばかりが得をするのか, 285p., 草思社 (2000年)
\bibitem{品質} 小林稔, 村松健児, \lq\lq 生産の計画立案技術の品質\rq\rq, 生産管理, Vol. 20, No. 2, pp. 29-37 (2014年)
\end{thebibliography}
\vspace*{-0.5\baselineskip}

40行目

41行目

42行目

42行目
\end{document}