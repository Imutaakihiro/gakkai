\chapter{はじめに}
\setcounter{page}{1}
\pagenumbering{arabic}

%%%%%%%%%%%%%%%%%%%%%%%%%%%%%%%%%%%%%%%%%%%%%%%%%%%%%%%%%%%%%%%%%%%%%%%%%%%%%%%
\section{研究背景}
%%%%%%%%%%%%%%%%%%%%%%%%%%%%%%%%%%%%%%%%%%%%%%%%%%%%%%%%%%%%%%%%%%%%%%%%%%%%%%%

高等教育機関では,教育の質保証が重要な課題であり,授業評価は教育改善のための主要な手段として広く実施されている\cite{marsh2007,spooren2013}.授業評価アンケートは一般に,評価スコアと自由記述から構成される.評価スコアは集計・比較が容易である一方,評価の理由や具体的な改善点は直接観測できない.

自由記述には授業への具体的な意見が含まれるが,非構造テキストであるため大規模分析には困難が伴う.そのため,近年は自然言語処理技術を用いた自動分析が進められている\cite{gottipati2018,hujala2020,bert}.また,感情分析は学生の満足度や不満を把握する方法として活用されている\cite{liu2012}.

本研究の対象データは,福岡工業大学における2018年度から2024年度までの7年間の授業評価データである.対象データの規模を表\ref{tab:data_intro}に示す.

\begin{table}[t]
    \centering
    \caption{本研究の対象データ}
    \label{tab:data_intro}
    \resizebox{0.75\textwidth}{!}{
    \begin{tabular}{l r}
        \toprule
        項目 & 値 \\
        \midrule
        対象期間 & 2018年度〜2023年度(6年間) \\
        対象学科数 & 9学科 \\
        授業数 & 3,268件 \\
        自由記述総件数 & 83,851件 \\
        平均自由記述数/授業 & 25.2件 \\
        \bottomrule
    \end{tabular}
    }
\end{table}

%%%%%%%%%%%%%%%%%%%%%%%%%%%%%%%%%%%%%%%%%%%%%%%%%%%%%%%%%%%%%%%%%%%%%%%%%%%%%%%
\section{課題の整理}
%%%%%%%%%%%%%%%%%%%%%%%%%%%%%%%%%%%%%%%%%%%%%%%%%%%%%%%%%%%%%%%%%%%%%%%%%%%%%%%

授業評価の活用には,(1) 評価スコアから評価要因の内訳が直接把握できない,(2) 自由記述は大規模かつ表現が多様であり手作業の分析が困難である,(3) 感情的側面と評価スコアの関係が明確でない,(4) 改善資源の優先順位付けに客観的根拠が不足する,という課題がある.これらは授業改善に必要な要因の抽出と整理を難しくしている.

%%%%%%%%%%%%%%%%%%%%%%%%%%%%%%%%%%%%%%%%%%%%%%%%%%%%%%%%%%%%%%%%%%%%%%%%%%%%%%%
\section{研究目的と仮説}
%%%%%%%%%%%%%%%%%%%%%%%%%%%%%%%%%%%%%%%%%%%%%%%%%%%%%%%%%%%%%%%%%%%%%%%%%%%%%%%

本研究の目的は,自由記述から感情スコアを推定し,授業評価スコアとの関係を分析した上で,共通要因と特化要因を定量的に整理することである.これにより,授業評価スコアの高さと関連する要因を明確化し,改善施策の検討に資する知見を得ることを目指す.

本研究では,以下の2点を仮説として設定する.
\begin{enumerate}
\item \textbf{仮説1}: 授業単位で集約した感情スコアと授業評価スコアには正の相関がある.
\item \textbf{仮説2}: 感情スコアと授業評価スコアの双方に関連する共通要因が存在する.
\end{enumerate}

%%%%%%%%%%%%%%%%%%%%%%%%%%%%%%%%%%%%%%%%%%%%%%%%%%%%%%%%%%%%%%%%%%%%%%%%%%%%%%%
\section{研究のアプローチ}
%%%%%%%%%%%%%%%%%%%%%%%%%%%%%%%%%%%%%%%%%%%%%%%%%%%%%%%%%%%%%%%%%%%%%%%%%%%%%%%

本研究では,BERTを基盤とした感情分類により自由記述の感情スコアを推定し\cite{bert,cl-tohoku},感情スコアと授業評価スコアを同時に扱うマルチタスク学習モデルを構築する\cite{mtl}.さらに,SHAP分析により語彙単位の寄与度を算出し\cite{shap},共通要因と特化要因を整理する.

%%%%%%%%%%%%%%%%%%%%%%%%%%%%%%%%%%%%%%%%%%%%%%%%%%%%%%%%%%%%%%%%%%%%%%%%%%%%%%%
\section{本研究の構成}
%%%%%%%%%%%%%%%%%%%%%%%%%%%%%%%%%%%%%%%%%%%%%%%%%%%%%%%%%%%%%%%%%%%%%%%%%%%%%%%

本研究は全5章からなる.第2章では関連研究を整理し,第3章でデータと手法を述べる.第4章で結果を示し,第5章でまとめと今後の課題を述べる.
