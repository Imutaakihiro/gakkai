\chapter{データと手法}

本章では,本研究で使用したデータセットの概要,前処理手順,モデル構成,SHAP分析の設定,評価指標を述べる.

%%%%%%%%%%%%%%%%%%%%%%%%%%%%%%%%%%%%%%%%%%%%%%%%%%%%%%%%%%%%%%%%%%%%%%%%%%%%%%%
\section{データセット}
%%%%%%%%%%%%%%%%%%%%%%%%%%%%%%%%%%%%%%%%%%%%%%%%%%%%%%%%%%%%%%%%%%%%%%%%%%%%%%%

\subsection{データセットの概要}
本研究では,福岡工業大学の授業評価システムにおける2018年度から2023年度までの6年間のデータを使用した.対象は9学科にわたり,授業数は3,268件である.自由記述の総件数は83,851件であり,各授業に対して平均25.2件の自由記述が付随している.

データセットの概要を表\ref{tab:dataset}に示す.

\begin{table}[t]
    \centering
    \caption{データセット概要}
    \label{tab:dataset}
    \resizebox{0.75\textwidth}{!}{
    \begin{tabular}{l r}
        \toprule
        項目 & 値 \\
        \midrule
        対象期間 & 2018年度〜2023年度(6年間) \\
        対象学科数 & 9 \\
        授業数 & 3,268 \\
        自由記述総件数 & 83,851 \\
        平均自由記述数/授業 & 25.2 \\
        自由記述の平均文字数 & 約41文字 \\
        \bottomrule
    \end{tabular}
    }
\end{table}

\subsection{授業評価アンケートの構成}
授業評価アンケートは,(1) 授業評価スコア(1〜4点),(2) 自由記述の2種類の情報から構成される.自由記述は以下の2つの質問からなる.

\begin{enumerate}
\item 先生に向けてこの授業の感想や学んだこと,意見や要望を記述してください
\item 次期履修者に向けて,この授業についてのアドバイスを記述してください
\end{enumerate}

授業評価スコアの平均値は3.459点(標準偏差0.216)であり,比較的高い評価に集中する傾向がある.自由記述は授業単位で複数件存在するため,授業単位で集約して分析する.

\subsection{データの特性}
本データセットには以下の特性がある.第一に,授業評価スコアは順序尺度である.第二に,自由記述の長さにばらつきがあり,短い記述から長い記述まで存在する.第三に,自由記述には授業内容への感想,教員への要望,履修者へのアドバイスなど多様な内容が含まれる.

%%%%%%%%%%%%%%%%%%%%%%%%%%%%%%%%%%%%%%%%%%%%%%%%%%%%%%%%%%%%%%%%%%%%%%%%%%%%%%%
\section{前処理}
%%%%%%%%%%%%%%%%%%%%%%%%%%%%%%%%%%%%%%%%%%%%%%%%%%%%%%%%%%%%%%%%%%%%%%%%%%%%%%%

\subsection{テキストの簡易整形}
自由記述の前処理は最小限に留め,以下のみ実施した.

\begin{enumerate}
\item \textbf{空白・改行の整理}: 連続する空白や改行を整形し,空欄を除外
\item \textbf{最大長制限}: モデル入力長の上限(512トークン)に合わせて切り詰め
\end{enumerate}

%%%%%%%%%%%%%%%%%%%%%%%%%%%%%%%%%%%%%%%%%%%%%%%%%%%%%%%%%%%%%%%%%%%%%%%%%%%%%%%
\section{教師データ}
%%%%%%%%%%%%%%%%%%%%%%%%%%%%%%%%%%%%%%%%%%%%%%%%%%%%%%%%%%%%%%%%%%%%%%%%%%%%%%%

\subsection{ラベリング手順}
感情分類モデルの構築のため,全83,851件の自由記述からランダムに1,000件を抽出し,手動でラベリングを行った.ラベルはネガティブ(−1),ニュートラル(0),ポジティブ(+1)の3クラスとした.

\begin{itemize}
\item \textbf{ポジティブ}: 授業に対する肯定的評価,満足感,感謝の表明を含む記述
\item \textbf{ネガティブ}: 授業に対する否定的評価,不満,改善要望を含む記述
\item \textbf{ニュートラル}: 事実の記述,中立的な感想,感情を含まない記述
\end{itemize}

\subsection{ラベル分布}
教師データのラベル分布を表\ref{tab:labeldist-method}に示す.ニュートラルが全体の62.8\%を占める一方,ネガティブ(19.1\%)とポジティブ(18.0\%)は少数である.

\begin{table}[t]
    \centering
    \caption{教師データのラベル分布(1,000件)}
    \label{tab:labeldist-method}
    \resizebox{0.6\textwidth}{!}{
    \begin{tabular}{l r r}
        \toprule
        ラベル & 件数 & 割合 \\
        \midrule
        ネガティブ & 191 & 19.1\% \\
        ニュートラル & 628 & 62.8\% \\
        ポジティブ & 180 & 18.0\% \\
        \midrule
        合計 & 1,000 & 100.0\% \\
        \bottomrule
    \end{tabular}
    }
\end{table}

\subsection{各クラスの語彙的特徴}
各感情クラスの語彙的特徴を把握するため,クラスごとのワードクラウドを作成した.図\ref{fig:wordcloud}に,ポジティブ,ニュートラル,ネガティブの各クラスにおける出現頻度の高い語彙を示す.

\begin{figure}[t]
    \centering
    \begin{minipage}{0.32\textwidth}
        \centering
        \includegraphics[width=\textwidth]{fig/POSITIVE_min3_wordcloud.png}
        \subcaption{ポジティブ}
    \end{minipage}
    \hfill
    \begin{minipage}{0.32\textwidth}
        \centering
        \includegraphics[width=\textwidth]{fig/NEUTRAL_min3_wordcloud.png}
        \subcaption{ニュートラル}
    \end{minipage}
    \hfill
    \begin{minipage}{0.32\textwidth}
        \centering
        \includegraphics[width=\textwidth]{fig/NEGATIVE_min3_wordcloud.png}
        \subcaption{ネガティブ}
    \end{minipage}
    \caption{各感情クラスの語彙的特徴(ワードクラウド)}
    \label{fig:wordcloud}
\end{figure}

語彙傾向はクラス間で差があり,感情分類モデルが語彙パターンを学習できる可能性を示唆する.

\subsection{データ分割}
教師データは訓練用と検証用に分割した.訓練データは800件(80\%),検証データは200件(20\%)とし,層化抽出によりラベル分布を維持した.

%%%%%%%%%%%%%%%%%%%%%%%%%%%%%%%%%%%%%%%%%%%%%%%%%%%%%%%%%%%%%%%%%%%%%%%%%%%%%%%
\section{モデル構成}
%%%%%%%%%%%%%%%%%%%%%%%%%%%%%%%%%%%%%%%%%%%%%%%%%%%%%%%%%%%%%%%%%%%%%%%%%%%%%%%

\subsection{BERTベース感情分類モデル}
感情分類には日本語BERTを基盤とするモデルを用いた\cite{bert,cl-tohoku}.BERTエンコーダの出力から[CLS]トークンのベクトルを取得し,3クラスの確率分布を出力する分類ヘッドを接続した.

\subsection{マルチタスク学習モデル}
感情スコア予測と授業評価スコア予測を同時に学習するマルチタスク学習モデルを構築した\cite{mtl}.BERTエンコーダを共有表現として用い,感情分類ヘッドと評価スコア予測ヘッドを分岐させる構成とした.損失関数は感情分類損失と評価スコア予測損失の重み付き和とし,$\alpha=\beta=0.5$とした.

\begin{equation}
\mathcal{L}_{\text{total}} = \alpha \cdot \mathcal{L}_{\text{sentiment}} + \beta \cdot \mathcal{L}_{\text{score}}
\label{eq:mtl_loss}
\end{equation}

マルチタスク学習モデルの構成を図\ref{fig:mtl_model}に示す.

\begin{figure}[t]
\centering
\begin{tabular}{c}
\hline
\textbf{マルチタスク学習モデルの構成} \\
\hline
入力テキスト \\
$\downarrow$ \\
BERTエンコーダ(共有) \\
$\downarrow$ \\
{[}CLS{]}トークンの出力ベクトル(768次元) \\
$\swarrow$ \hspace{2cm} $\searrow$ \\
感情分類ヘッド \hspace{1cm} 評価スコア予測ヘッド \\
$\downarrow$ \hspace{2.5cm} $\downarrow$ \\
3クラス確率 \hspace{1.5cm} 評価スコア(回帰値) \\
\hline
\end{tabular}
\caption{マルチタスク学習モデルのアーキテクチャ}
\label{fig:mtl_model}
\end{figure}

%%%%%%%%%%%%%%%%%%%%%%%%%%%%%%%%%%%%%%%%%%%%%%%%%%%%%%%%%%%%%%%%%%%%%%%%%%%%%%%
\section{授業単位集約と相関分析}
%%%%%%%%%%%%%%%%%%%%%%%%%%%%%%%%%%%%%%%%%%%%%%%%%%%%%%%%%%%%%%%%%%%%%%%%%%%%%%%

感情分類モデルにより全83,851件の自由記述に対して感情スコアを推定した後,授業単位で感情スコアを集約した.各授業の感情スコアは,その授業に属する自由記述の感情スコアの算術平均として算出した.

\begin{equation}
\bar{S}_j = \frac{1}{n_j}\sum_{i=1}^{n_j} s_{ij}
\label{eq:agg}
\end{equation}

授業単位の感情スコアと授業評価スコアの関係性を検討するため,ピアソン相関係数,スピアマン順位相関係数,ケンドール順位相関係数を算出した.

%%%%%%%%%%%%%%%%%%%%%%%%%%%%%%%%%%%%%%%%%%%%%%%%%%%%%%%%%%%%%%%%%%%%%%%%%%%%%%%
\section{SHAP分析}
%%%%%%%%%%%%%%%%%%%%%%%%%%%%%%%%%%%%%%%%%%%%%%%%%%%%%%%%%%%%%%%%%%%%%%%%%%%%%%%

SHAP分析は計算コストが高いため,層化サンプリングにより5,000件のサンプルを抽出して分析を行った.また,出現回数が5回未満の低頻度語は除外し,最終的に3,198語を分析対象とした.

分析対象の設定を表\ref{tab:shap_setting}に示す.

\begin{table}[t]
    \centering
    \caption{SHAP分析の設定}
    \label{tab:shap_setting}
    \resizebox{0.7\textwidth}{!}{
    \begin{tabular}{l r}
        \toprule
        項目 & 値 \\
        \midrule
        分析サンプル数 & 5,000件 \\
        サンプリング手法 & 層化サンプリング \\
        最小出現回数閾値 & 5回 \\
        分析対象語彙数 & 3,198語 \\
        \bottomrule
    \end{tabular}
    }
\end{table}

マルチタスクモデルのSHAP分析では,感情スコアと評価スコアへの寄与度に基づき,語彙を共通要因・感情特化要因・評価特化要因・低重要度要因の4グループに分類した.

%%%%%%%%%%%%%%%%%%%%%%%%%%%%%%%%%%%%%%%%%%%%%%%%%%%%%%%%%%%%%%%%%%%%%%%%%%%%%%%
\section{評価指標}
%%%%%%%%%%%%%%%%%%%%%%%%%%%%%%%%%%%%%%%%%%%%%%%%%%%%%%%%%%%%%%%%%%%%%%%%%%%%%%%

感情分類モデルの評価にはAccuracyとF1スコア(マクロ平均・重み付き平均)を用いた.授業評価スコア予測の評価には$R^2$,RMSE,MAEを用いた.

%%%%%%%%%%%%%%%%%%%%%%%%%%%%%%%%%%%%%%%%%%%%%%%%%%%%%%%%%%%%%%%%%%%%%%%%%%%%%%%
\section{分析フロー}
%%%%%%%%%%%%%%%%%%%%%%%%%%%%%%%%%%%%%%%%%%%%%%%%%%%%%%%%%%%%%%%%%%%%%%%%%%%%%%%

本研究の分析フローを図\ref{fig:flow}に示す.

\begin{figure}[t]
\centering
\begin{tabular}{|l|}
\hline
\textbf{【データ収集】} \\
授業評価アンケート(3,268授業,83,851件自由記述) \\
\hline
$\downarrow$ \\
\hline
\textbf{【前処理】} \\
簡易整形,入力長の調整 \\
\hline
$\downarrow$ \\
\hline
\textbf{【教師データ作成】} \\
1,000件の手動ラベリング(3クラス) \\
\hline
$\downarrow$ \\
\hline
\textbf{【モデル構築】} \\
感情分類モデル(BERT + 分類ヘッド) \\
マルチタスク学習モデル(BERT + 2ヘッド) \\
\hline
$\downarrow$ \\
\hline
\textbf{【感情スコア推定】} \\
全自由記述に対する感情スコア推定 \\
\hline
$\downarrow$ \\
\hline
\textbf{【授業単位集約】} \\
授業ごとの感情スコア平均を算出 \\
\hline
$\downarrow$ \\
\hline
\textbf{【相関分析】} \\
感情スコアと授業評価スコアの相関を検証 \\
\hline
$\downarrow$ \\
\hline
\textbf{【SHAP分析】} \\
5,000件サンプル,3,198語を対象に要因分析 \\
要因を4グループに分類 \\
\hline
\end{tabular}
\caption{分析フローの概略}
\label{fig:flow}
\end{figure}

本章では,データセットの概要,前処理手順,BERTを基盤としたモデル構成,SHAP分析の設定,および評価指標について述べた.次章では,これらの手法を用いて得られた結果を報告する.
