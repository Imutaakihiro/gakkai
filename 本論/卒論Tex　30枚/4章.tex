%-----------------------------
\chapter{結果と考察}
%-----------------------------

本章では,授業評価アンケートの基礎統計量,相関分析結果,SHAP分析による要因抽出結果を示し,得られた知見を考察する.

%%%%%%%%%%%%%%%%%%%%%%%%%%%%%%%%%%%%%%%%%%%%%%%%%%%%%%%%%%%%%%%%%%%%%%%%%%%%%%%
\section{基礎統計量}
%%%%%%%%%%%%%%%%%%%%%%%%%%%%%%%%%%%%%%%%%%%%%%%%%%%%%%%%%%%%%%%%%%%%%%%%%%%%%%%

\subsection{感情スコアと授業評価スコアの分布}
感情スコアと授業評価スコアの基本統計量を表\ref{tab:basicstats}に示す.感情スコアは授業単位で集約した値であり,−1から+1の範囲をとる.授業評価スコアは4段階(1〜4点)である.

感情スコアは平均0.001(標準偏差0.260)であり,授業単位集約によりポジティブ・ネガティブが相殺されやすい傾向が見られる.授業評価スコアは平均3.459(標準偏差0.216)で,分布は3点台後半に集中している.

\begin{table}[t]
    \centering
    \caption{感情スコアと授業評価スコアの基本統計量}
    \label{tab:basicstats}
    \resizebox{0.7\textwidth}{!}{
    \begin{tabular}{l r r}
        \toprule
        統計量 & 感情スコア & 授業評価スコア \\
        \midrule
        平均 & 0.001 & 3.459 \\
        標準偏差 & 0.260 & 0.216 \\
        最小値 & −1.000 & 2.000 \\
        第1四分位数(Q1) & −0.167 & 3.330 \\
        中央値(Q2) & 0.000 & 3.480 \\
        第3四分位数(Q3) & 0.167 & 3.600 \\
        最大値 & 1.000 & 4.000 \\
        \bottomrule
    \end{tabular}
    }
\end{table}

\subsection{教師データのラベル分布}
教師データのラベル分布を表\ref{tab:labeldist}に示す.ニュートラルが628件(62.8\%)と大半を占め,ネガティブ191件(19.1\%),ポジティブ180件(18.0\%)である.

\begin{table}[t]
    \centering
    \caption{教師データのラベル分布(1,000件)}
    \label{tab:labeldist}
    \resizebox{0.6\textwidth}{!}{
    \begin{tabular}{l r r}
        \toprule
        ラベル & 件数 & 割合 \\
        \midrule
        ネガティブ & 191 & 19.1\% \\
        ニュートラル & 628 & 62.8\% \\
        ポジティブ & 180 & 18.0\% \\
        \midrule
        合計 & 1,000 & 100.0\% \\
        \bottomrule
    \end{tabular}
    }
\end{table}

%%%%%%%%%%%%%%%%%%%%%%%%%%%%%%%%%%%%%%%%%%%%%%%%%%%%%%%%%%%%%%%%%%%%%%%%%%%%%%%
\section{感情スコアと授業評価スコアの相関分析}
%%%%%%%%%%%%%%%%%%%%%%%%%%%%%%%%%%%%%%%%%%%%%%%%%%%%%%%%%%%%%%%%%%%%%%%%%%%%%%%

\subsection{相関分析の結果}
授業単位で集約した感情スコアと授業評価スコアの相関分析結果を表\ref{tab:correlation}に示す.3,268授業を対象に分析を行った.

\begin{table}[t]
    \centering
    \caption{感情スコアと授業評価スコアの相関分析結果(N=3,268)}
    \label{tab:correlation}
    \resizebox{0.75\textwidth}{!}{
    \begin{tabular}{l r r l}
        \toprule
        指標 & 相関係数 & $p$値 & 解釈 \\
        \midrule
        ピアソン相関係数 & 0.3097 & $<0.000001$ & 中程度の正の相関 \\
        スピアマン順位相関係数 & 0.2970 & $<0.000001$ & 中程度の正の相関 \\
        ケンドール順位相関係数 & 0.2042 & $<0.000001$ & 弱〜中程度の正の相関 \\
        \bottomrule
    \end{tabular}
    }
\end{table}

ピアソン相関係数は0.3097($p<0.000001$)であり,中程度の正の相関が確認された.スピアマン順位相関係数(0.2970)およびケンドール順位相関係数(0.2042)においても同様に統計的に有意な正の相関が得られた.

感情スコアと授業評価スコアの散布図を図\ref{fig:correlation}に示す.

\begin{figure}[tb]
    \centering
    \includegraphics[width=0.8\textwidth]{fig/correlation_scatter.png}
    \caption{感情スコアと授業評価スコアの散布図(N=3,268)}
    \label{fig:correlation}
\end{figure}

\subsection{相関の解釈}
相関係数0.31は中程度の正の相関と解釈され,感情スコアと評価スコアには一定の関係があることが示唆される.一方で,相関係数が1に近くないことは,評価スコアが感情以外の要因も反映している可能性を示す.複数の相関指標で一貫した結果が得られたことから,関係性は一定の頑健性を持つと考えられる.

%%%%%%%%%%%%%%%%%%%%%%%%%%%%%%%%%%%%%%%%%%%%%%%%%%%%%%%%%%%%%%%%%%%%%%%%%%%%%%%
\section{単一タスクモデルのSHAP分析}
%%%%%%%%%%%%%%%%%%%%%%%%%%%%%%%%%%%%%%%%%%%%%%%%%%%%%%%%%%%%%%%%%%%%%%%%%%%%%%%

\subsection{ポジティブ判定に寄与する重要語}
感情分類モデルに対するSHAP分析を行い,ポジティブ判定に寄与する重要語を抽出した.5,000件のサンプル(ポジティブ2,500件,ネガティブ2,500件)を層化サンプリングし,出現回数5回以上の3,198語を分析対象とした.

上位10語を表\ref{tab:shap_single}に示す.「やす」「良かっ」「おもしろ」「面白」など,理解のしやすさや授業の面白さを示す語彙が上位に現れた.

\begin{table}[t]
    \centering
    \caption{ポジティブ判定に寄与する重要語TOP10}
    \label{tab:shap_single}
    \resizebox{0.75\textwidth}{!}{
    \begin{tabular}{r l r r l}
        \toprule
        順位 & 単語 & 平均SHAP値 & 出現回数 & 解釈 \\
        \midrule
        1 & やす & 0.2660 & 337 & 分かりやすい,理解しやすい \\
        2 & 良かっ & 0.2466 & 207 & 良かった(過去形) \\
        3 & おもしろ & 0.2438 & 10 & 面白い \\
        4 & よかっ & 0.2251 & 195 & 良かった(ひらがな) \\
        5 & 面白 & 0.2178 & 100 & 面白い \\
        6 & 楽しい & 0.1959 & 67 & 楽しい \\
        7 & 楽しめる & 0.1876 & 6 & 楽しむことができる \\
        8 & ありが & 0.1760 & 19 & ありがとう \\
        9 & 楽し & 0.1642 & 192 & 楽しい(語幹) \\
        10 & 面白い & 0.1518 & 37 & 面白い \\
        \bottomrule
    \end{tabular}
    }
\end{table}

\subsection{ネガティブ判定に寄与する重要語}
ネガティブ判定に寄与する重要語TOP10を表\ref{tab:shap_negative}に示す.「ほしい」「難しかっ」「苦手」など,要望や困難さを示す語彙が上位に現れた.

\begin{table}[t]
    \centering
    \caption{ネガティブ判定に寄与する重要語TOP10}
    \label{tab:shap_negative}
    \resizebox{0.75\textwidth}{!}{
    \begin{tabular}{r l r r l}
        \toprule
        順位 & 単語 & 平均SHAP値 & 出現回数 & 解釈 \\
        \midrule
        1 & ほし & −0.0443 & 5 & 〜してほしい \\
        2 & ほう & −0.0425 & 98 & 〜した方がいい \\
        3 & 大 & −0.0346 & 86 & 大変,大きい \\
        4 & まじ & −0.0314 & 5 & まじめ,まじで \\
        5 & 難しかっ & −0.0311 & 88 & 難しかった \\
        6 & 直す & −0.0264 & 6 & 直してほしい \\
        7 & ほしい & −0.0263 & 45 & 〜してほしい \\
        8 & 欲しい & −0.0247 & 33 & 〜が欲しい \\
        9 & 奥 & −0.0219 & 8 & 奥が深い等 \\
        10 & 器具 & −0.0211 & 7 & 実験器具等 \\
        \bottomrule
    \end{tabular}
    }
\end{table}

ポジティブ判定とネガティブ判定に寄与する重要語の可視化を図\ref{fig:shap}に示す.

\begin{figure}[tb]
    \centering
    \begin{minipage}{0.49\textwidth}
        \centering
        \includegraphics[width=\textwidth]{fig/shap_positive.png}
        \subcaption{ポジティブ判定に寄与する重要語}
    \end{minipage}
    \hfill
    \begin{minipage}{0.49\textwidth}
        \centering
        \includegraphics[width=\textwidth]{fig/shap_negative.png}
        \subcaption{ネガティブ判定に寄与する重要語}
    \end{minipage}
\caption{SHAP分析による重要語の可視化(TOP10)}
    \label{fig:shap}
\end{figure}

\subsection{重要語の解釈と考察}
ポジティブ判定では理解のしやすさや面白さを示す語彙が多く,感情評価に関連する可能性が示唆される.一方,ネガティブ判定では「ほしい」「難しかっ」など要望・困難さを示す語彙が現れ,改善要望として解釈できる.

%%%%%%%%%%%%%%%%%%%%%%%%%%%%%%%%%%%%%%%%%%%%%%%%%%%%%%%%%%%%%%%%%%%%%%%%%%%%%%%
\section{マルチタスク学習のSHAP分析}
%%%%%%%%%%%%%%%%%%%%%%%%%%%%%%%%%%%%%%%%%%%%%%%%%%%%%%%%%%%%%%%%%%%%%%%%%%%%%%%

\subsection{要因タイプ別の分類結果}
マルチタスクモデルのSHAP分析により,感情スコアと授業評価スコアの両方に影響する共通要因と,各タスクに特化した要因を分離した.語彙の分類結果を表\ref{tab:shap_types}に示す.

\begin{table}[t]
    \centering
    \caption{語彙の要因タイプ別内訳(3,198語)}
    \label{tab:shap_types}
    \resizebox{0.7\textwidth}{!}{
    \begin{tabular}{l r r l}
        \toprule
        要因タイプ & 語彙数 & 割合 & 特徴 \\
        \midrule
        共通要因(満足度) & 577 & 18.0\% & 両スコアに寄与 \\
        感情特化要因 & 1,200 & 37.5\% & 感情スコアのみに寄与 \\
        評価特化要因 & 532 & 16.6\% & 評価スコアのみに寄与 \\
        低重要度要因 & 889 & 27.8\% & 両スコアへの影響小 \\
        \midrule
        合計 & 3,198 & 100.0\% & — \\
        \bottomrule
    \end{tabular}
    }
\end{table}

共通要因は577語(18.0\%)であり,これらは感情スコアと評価スコアの双方に寄与する「満足度要因」と解釈できる.感情特化要因が最も多く1,200語(37.5\%)を占め,評価特化要因は532語(16.6\%)である.

\subsection{共通要因(満足度要因)の分析}
共通要因の上位語彙を表\ref{tab:shap_common}に示す.「学ぶ」「理解」「総括」「推奨」「人数」など,学習成果に関わる語彙が上位に現れている.

\begin{table}[t]
    \centering
    \caption{共通要因(満足度要因)TOP10}
    \label{tab:shap_common}
    \resizebox{0.75\textwidth}{!}{
    \begin{tabular}{r l r r l}
        \toprule
        順位 & 単語 & 感情重要度 & 評価重要度 & 解釈 \\
        \midrule
        1 & 学ぶ & 0.001278 & 0.001386 & 学びがあった \\
        2 & 理解 & 0.001073 & 0.000833 & 理解できた \\
        3 & 総括 & 0.000974 & 0.000952 & 総括・まとめ \\
        4 & 推奨 & 0.001132 & 0.000755 & 推奨したい \\
        5 & 人数 & 0.001195 & 0.000704 & 人数が適切 \\
        6 & 把握 & 0.000891 & 0.000682 & 内容を把握 \\
        7 & 習得 & 0.000823 & 0.000645 & スキル習得 \\
        8 & 基礎 & 0.000756 & 0.000612 & 基礎が身についた \\
        9 & 応用 & 0.000698 & 0.000589 & 応用力 \\
        10 & 実践 & 0.000654 & 0.000567 & 実践的 \\
        \bottomrule
    \end{tabular}
    }
\end{table}

共通要因は,感情スコアと評価スコアの双方に同時に寄与する語彙であり,限られた資源で授業改善を行う際の「投資効率が高い要因」と解釈できる.例えば「学ぶ」「理解」「総括」といった語彙は,学習成果の実感に関わる可能性がある.また,感情特化要因には情緒的な語彙が,評価特化要因には有用性や運営に関わる語彙が含まれる可能性がある.

\subsection{要因分離の意義}
マルチタスク学習によって共通要因と特化要因を分離できる点は,授業改善の方針選択に有用である.共通要因の割合が18.0\%にとどまることは,満足度と評価が必ずしも一致しない可能性を示し,評価スコアと感情スコアが異なる側面を反映することを示唆する.

%%%%%%%%%%%%%%%%%%%%%%%%%%%%%%%%%%%%%%%%%%%%%%%%%%%%%%%%%%%%%%%%%%%%%%%%%%%%%%%
\section{感情特化要因・評価特化要因の示唆}
%%%%%%%%%%%%%%%%%%%%%%%%%%%%%%%%%%%%%%%%%%%%%%%%%%%%%%%%%%%%%%%%%%%%%%%%%%%%%%%

感情特化要因(1,200語,37.5\%)は楽しさや満足感に関わる語彙が中心であり,評価特化要因(532語,16.6\%)は有用性や授業運営に関わる語彙が中心となる可能性がある.両者は役割が異なるため,改善目的に応じた解釈が必要である.

感情特化要因・評価特化要因の代表的な語彙例(意味のある単語のみ)を表\ref{tab:shap_specific_examples}に示す.

\begin{table}[t]
    \centering
    \caption{感情特化要因・評価特化要因の代表語彙(抜粋)}
    \label{tab:shap_specific_examples}
    \resizebox{0.8\textwidth}{!}{
    \begin{tabular}{l l}
        \toprule
        要因タイプ & 代表語彙(抜粋) \\
        \midrule
        感情特化要因 & 感謝,ありがとう,面白い,楽しい,嬉しい,対話 \\
        評価特化要因 & 設計,回路,関数,人材,調整,符号 \\
        \bottomrule
    \end{tabular}
    }
\end{table}

\subsection{実践的活用の方向性}
この区別により,改善の優先順位付けや授業タイプ別の施策設計が可能となる.共通要因への対応を優先しつつ,目的に応じて感情特化要因または評価特化要因を重視する戦略が考えられる.

%%%%%%%%%%%%%%%%%%%%%%%%%%%%%%%%%%%%%%%%%%%%%%%%%%%%%%%%%%%%%%%%%%%%%%%%%%%%%%%
\section{総合考察}
%%%%%%%%%%%%%%%%%%%%%%%%%%%%%%%%%%%%%%%%%%%%%%%%%%%%%%%%%%%%%%%%%%%%%%%%%%%%%%%

\subsection{研究仮説の検証}
本章の結果を踏まえ,研究仮説の検証を行う.

\textbf{仮説1}は,相関分析により中程度の正の相関が確認された.\textbf{仮説2}は,SHAP分析により共通要因が抽出されたことから支持された.

\subsection{主要な発見のまとめ}
本研究では,理解しやすさや面白さに関わる語彙が感情評価に強く関連すること,感情スコアと評価スコアが中程度の相関を持つこと,共通要因と特化要因を語彙レベルで分離できることが示された.

\subsection{先行研究との比較}
本研究の結果は,先行研究の知見と以下の点で整合的である.

理解しやすさや面白さが満足度に寄与するという結果は,授業評価に関する先行研究の知見と一致する\cite{marsh2007,spooren2013}.ただし,本研究ではSHAP分析により要因を定量化し\cite{shap},語彙レベルでの寄与度を明らかにした点が新規性である.

\subsection{研究の限界}
本研究には以下の限界がある.

第一に,相関分析やSHAP分析は関連性を示すものであり,因果関係は検証していない.第二に,データは単一大学に限定されているため,一般化可能性には限界がある.

\subsection{教育改善への示唆}
本研究の結果は,教育改善に以下の示唆を提供する.

理解しやすさや学習成果の実感に関わる要因への対応は,満足度と評価の双方に寄与する可能性が高い.また,授業の面白さや雰囲気に関わる要因は感情的満足度に影響し得るため,改善目的に応じた施策設計が有効と考えられる.

\clearpage
