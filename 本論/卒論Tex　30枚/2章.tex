\chapter{関連研究}

本章では,授業評価研究,自由記述分析,感情分析,BERT,マルチタスク学習,SHAPに関する先行研究を概観し,本研究の位置づけを示す.

%%%%%%%%%%%%%%%%%%%%%%%%%%%%%%%%%%%%%%%%%%%%%%%%%%%%%%%%%%%%%%%%%%%%%%%%%%%%%%%
\section{授業評価研究}
%%%%%%%%%%%%%%%%%%%%%%%%%%%%%%%%%%%%%%%%%%%%%%%%%%%%%%%%%%%%%%%%%%%%%%%%%%%%%%%

授業評価は教育改善のための重要な指標であり,評価の信頼性・妥当性やバイアスに関する議論がある\cite{marsh2007,spooren2013}.評価スコアは比較が容易である一方,評価の理由を説明できない点が課題とされる.

%%%%%%%%%%%%%%%%%%%%%%%%%%%%%%%%%%%%%%%%%%%%%%%%%%%%%%%%%%%%%%%%%%%%%%%%%%%%%%%
\section{自由記述分析と感情分析}
%%%%%%%%%%%%%%%%%%%%%%%%%%%%%%%%%%%%%%%%%%%%%%%%%%%%%%%%%%%%%%%%%%%%%%%%%%%%%%%

自由記述は授業評価の理由や具体的な改善要望を含むが,大規模分析には自然言語処理が必要である\cite{gottipati2018,hujala2020}.感情分析はテキストから肯定・否定・中立を推定する手法であり\cite{liu2012},評価スコアとの統合的な分析は限定的である.

%%%%%%%%%%%%%%%%%%%%%%%%%%%%%%%%%%%%%%%%%%%%%%%%%%%%%%%%%%%%%%%%%%%%%%%%%%%%%%%
\section{BERTと言語モデル}
%%%%%%%%%%%%%%%%%%%%%%%%%%%%%%%%%%%%%%%%%%%%%%%%%%%%%%%%%%%%%%%%%%%%%%%%%%%%%%%

BERTは双方向の文脈情報を考慮できる事前学習済み言語モデルであり\cite{bert,transformer},日本語モデルも公開されている\cite{cl-tohoku}.自由記述のような多様な表現を含むテキストに対して,高い表現力を持つ点が特長である.

%%%%%%%%%%%%%%%%%%%%%%%%%%%%%%%%%%%%%%%%%%%%%%%%%%%%%%%%%%%%%%%%%%%%%%%%%%%%%%%
\section{マルチタスク学習}
%%%%%%%%%%%%%%%%%%%%%%%%%%%%%%%%%%%%%%%%%%%%%%%%%%%%%%%%%%%%%%%%%%%%%%%%%%%%%%%

マルチタスク学習は関連タスクを同時に学習し,共通表現を共有する枠組みである\cite{mtl}.評価スコアと感情スコアの同時学習に適用可能である.

%%%%%%%%%%%%%%%%%%%%%%%%%%%%%%%%%%%%%%%%%%%%%%%%%%%%%%%%%%%%%%%%%%%%%%%%%%%%%%%
\section{解釈可能AIとSHAP}
%%%%%%%%%%%%%%%%%%%%%%%%%%%%%%%%%%%%%%%%%%%%%%%%%%%%%%%%%%%%%%%%%%%%%%%%%%%%%%%

予測根拠を説明する解釈可能AIが注目されており,SHAPは協力ゲーム理論に基づく特徴量重要度の算出手法として広く利用されている\cite{shap}.テキストでは語彙単位の寄与度を提示できる点が利点である.

%%%%%%%%%%%%%%%%%%%%%%%%%%%%%%%%%%%%%%%%%%%%%%%%%%%%%%%%%%%%%%%%%%%%%%%%%%%%%%%
\section{本研究の位置づけ}
%%%%%%%%%%%%%%%%%%%%%%%%%%%%%%%%%%%%%%%%%%%%%%%%%%%%%%%%%%%%%%%%%%%%%%%%%%%%%%%

既存研究では,自由記述分析や感情分析は行われているが,評価スコアとの統合的分析や要因の定量化は十分ではない\cite{gottipati2018,hujala2020}.本研究は,BERTによる感情分類とマルチタスク学習を組み合わせ,SHAP分析により共通要因・特化要因を語彙レベルで整理する点に特徴がある.

既存研究との差異を表\ref{tab:comparison}に示す.

\begin{table}[t]
    \centering
    \caption{既存研究との比較}
    \label{tab:comparison}
    \resizebox{0.9\textwidth}{!}{
    \begin{tabular}{l c c c c}
        \toprule
        研究 & 評価スコア分析 & 自由記述分析 & 統合分析 & 要因の定量化 \\
        \midrule
        Gottipati et al. (2018) & − & ○ & − & − \\
        Hujala et al. (2020) & ○ & ○ & △ & − \\
        \textbf{本研究} & \textbf{○} & \textbf{○} & \textbf{○} & \textbf{○} \\
        \bottomrule
    \end{tabular}
    }
\end{table}

この位置づけに基づき,評価スコアの高さと関連する要因を定量的に示すことを目的とする.
