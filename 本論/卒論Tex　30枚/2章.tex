\chapter{関連研究}

本章では,授業評価研究と自由記述分析の先行研究を概観し,本研究の位置づけを示す.授業評価に関する議論の流れを整理することで,本研究が扱う課題の背景を明確にする.

%%%%%%%%%%%%%%%%%%%%%%%%%%%%%%%%%%%%%%%%%%%%%%%%%%%%%%%%%%%%%%%%%%%%%%%%%%%%%%%
\section{授業評価研究}
%%%%%%%%%%%%%%%%%%%%%%%%%%%%%%%%%%%%%%%%%%%%%%%%%%%%%%%%%%%%%%%%%%%%%%%%%%%%%%%

授業評価は教育改善のための重要な指標であり,評価の信頼性・妥当性やバイアスに関する議論がある\cite{marsh2007,spooren2013}.評価スコアは比較が容易である一方,評価の理由を説明できない点が課題とされる.

一方で,授業評価は回答率の低さや選択バイアスの影響を受ける可能性が指摘されている.例えば,大規模データを用いた研究では,非回答による選択バイアスが評価スコアやランキングに影響し得ることが示されている\cite{goos2017}.

%%%%%%%%%%%%%%%%%%%%%%%%%%%%%%%%%%%%%%%%%%%%%%%%%%%%%%%%%%%%%%%%%%%%%%%%%%%%%%%
\section{授業評価の自由記述分析と本研究の位置づけ}
%%%%%%%%%%%%%%%%%%%%%%%%%%%%%%%%%%%%%%%%%%%%%%%%%%%%%%%%%%%%%%%%%%%%%%%%%%%%%%%

数値評価だけでは評価理由や改善要望が把握しにくいため,自由記述の分析が重要である.国内研究では,授業アンケートの自由記述から評価情報や要望を抽出するためにテキストマイニングを用いた事例が報告されている\cite{yuasa2012}.

また,海外でも自由記述を対象とした分析が進められており,教育改善に資する示唆を得る取り組みが行われている\cite{gottipati2018,hujala2020}.これらの研究は,数値評価と自由記述の両方を活用する必要性を示唆する.

本研究は,授業評価スコアと自由記述の関係を定量的に捉えることで,授業改善に資する要因の整理を目指す.分析手法の詳細は第3章で述べ,結果と考察は第4章で示す.
