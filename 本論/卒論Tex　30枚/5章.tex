\chapter{おわりに}

本章では,本研究の成果を簡潔にまとめ,研究の限界と今後の課題を述べる.まとめと課題を順に整理する.

%%%%%%%%%%%%%%%%%%%%%%%%%%%%%%%%%%%%%%%%%%%%%%%%%%%%%%%%%%%%%%%%%%%%%%%%%%%%%%%
\section{まとめ}
%%%%%%%%%%%%%%%%%%%%%%%%%%%%%%%%%%%%%%%%%%%%%%%%%%%%%%%%%%%%%%%%%%%%%%%%%%%%%%%

本研究では,2018年度〜2024年度の授業評価データ(3,268授業,83,851件自由記述)を対象に,自由記述の感情スコア推定と授業評価スコアとの関係分析を行った.感情スコアと授業評価スコアには正の相関(ピアソン0.3097/スピアマン0.2970/ケンドール0.2042)が確認され,両者が一定の関係を持つ可能性が示唆された.

さらに,BERTを基盤としたマルチタスク学習とSHAP分析により,3,198語を共通要因(18.0\%),感情特化要因(37.5\%),評価特化要因(16.6\%),低重要度要因(27.8\%)に分類した.これにより,授業評価スコアと感情スコアの双方に関わる語彙を定量的に整理し,授業改善の優先順位付けに資する基盤を提示した.整理結果は以降の実践的示唆の前提として位置づける.

\begin{table}[t]
    \centering
    \caption{主要結果の要約}
    \label{tab:summary}
    \resizebox{0.8\textwidth}{!}{
    \begin{tabular}{l l}
        \toprule
        項目 & 値 \\
        \midrule
        対象期間 & 2018年度〜2024年度(7年間) \\
        授業数 / 自由記述数 & 3,268 / 83,851 \\
        相関係数(ピアソン/スピアマン/ケンドール) & 0.3097 / 0.2970 / 0.2042 \\
        共通要因の割合 & 18.0\%(577語) \\
        \bottomrule
    \end{tabular}
    }
\end{table}

%%%%%%%%%%%%%%%%%%%%%%%%%%%%%%%%%%%%%%%%%%%%%%%%%%%%%%%%%%%%%%%%%%%%%%%%%%%%%%%
\section{研究の限界と今後の課題}
%%%%%%%%%%%%%%%%%%%%%%%%%%%%%%%%%%%%%%%%%%%%%%%%%%%%%%%%%%%%%%%%%%%%%%%%%%%%%%%

第一に,本研究は相関関係と寄与度の分析に基づくため,因果関係は検証していない.第二に,対象が単一大学に限定されており,一般化可能性には限界がある.第三に,教師データは1,000件であり,クラス不均衡の影響が残る可能性がある.また,2018年度〜2024年度の7年間を一括して分析しているため,時間的変化を詳細には捉えていない.

本研究の結果は福岡工業大学のデータに基づくため,一般化には限界がある.他大学データでの再現性確認と,教師データ拡充やモデル改良による精度向上が今後の課題である.
